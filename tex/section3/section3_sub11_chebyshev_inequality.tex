\subsection{Chebyshev's Inequality}
\begin{theorem}
If $X$ is.a real-valued random variable, and $E[X] = \mu$ and $V[x] =
\sigma^2$, then
\[
	P(\vert X - \mu \vert \geq x \cdot \sigma) \leq \frac{1}{x^2} 
	\qquad \forall x > 0
\]
\end{theorem}
\noindent \textbf{Interpretation}: The theorem simply says that $X$ is
concentrated around its mean with high probability, i.e. with probability at
least $1 - \frac{1}{x^2}$, $X$ falls in the interval $(\mu - x \sigma,
\mu + x \sigma)$.
\begin{proof}
\[
	  P(\vert X - \mu \vert) \geq x \sigma
	= P(\vert X - \mu \vert^2 \geq x^2 \sigma^2)
	\leq \frac{E[x - \mu]^2}{x^2 \sigma^2}
	= \frac{\sigma^2}{x^2 \sigma^2}
	= \frac{1}{x^2}
\]
\end{proof}
