\subsection{Markov's Inequality}
\begin{theorem}
    If $X$ is a real-valued nonnegative random variable, then
    \[ P(X \geq a) \leq \frac{E[X]}{a} \qquad a > 0                          \]
\end{theorem}
\begin{proof}
\begin{align*}
   &X = X \cdot 1_{\lbrace x \geq a \rbrace} + X \cdot 1_{\lbrace x < a\rbrace}
      \geq a \cdot 1_{\lbrace x \geq a \rbrace} + 0
      = a \cdot 1_{\lbrace x \geq a \rbrace}                                 \\
   &\Rightarrow \exists X \geq E[a \cdot 1_{\lbrace x \geq a \rbrace}]
                          = a P(X \geq a)                                    \\
   &\Rightarrow P(X \geq a) \leq \frac{E[X]}{a}
\end{align*}
\end{proof}

\paragraph{A simple consequence of Markov's inequality}
\begin{lemma}
If $X$ is a real-valued random variable and $E[\vert X \vert^\alpha]$ for some
$\alpha > 0$, then $\forall a > 0$,
\[
	P(\vert x \vert > a) = P(\vert x \vert^\alpha > a^\alpha)
	                     \leq \frac{E[\vert X \vert^\alpha]}{a^\alpha}
\]
by applying Markov's inequality to the random variable $\vert X
\vert^\alpha$.
\end{lemma}
\note These are simple examples of what is called `tail-bounds' for random
variables.