\section{Discrete Random Variables (RVs)}
\begin{definition}
    A random variable is a function defined on some sample space that takes
values in $\mathbb{R}^d$, i.e., if $S$ is a sample space and $X : S
\rightarrow \mathbb{R}^d$, then $X$ is an $\mathbb{R}^d$-valued random
variable. If $d \geq 2$, these are also called random vectors.
\end{definition}

\begin{example} \quad                                                        \\
\begin{enumerate}[noitemsep, topsep=0em]
    \item \label{sec3-example-1}
          \textbf{Experiment}: roll a fair die                               \\
          \textbf{Sample Space}: $S = \lbrace 1, 2, \dots, 6 \rbrace$        \\
          Then $X_1 : S \rightarrow \mathbb{R}^2$ defined as
          \begin{equation*}
              X_1(i) = (i^2, 3 \cdot i), i \in \lbrace 1, 2, \dots, 6 \rbrace
          \end{equation*}
          is a random variable.
    \item \label{sec3-example-2}
          \textbf{Experiment}: select a person from a city and measure his/her
          height                                                             \\
          \textbf{Sample Space}: $S = (0, \infty)$. 
          Then $X_1 : S \rightarrow \mathbb{R}$ defined as
          \begin{equation*}
              X_2(u) = u, u \in S
          \end{equation*}
          is a random variable defined on $S$.
\end{enumerate}
\end{example}

\begin{definition}
    An random variable $X$ is discrete if it can assume only a finite or
countably infinite number of distinct values.
\end{definition}

\begin{example}
    In (\ref{sec3-example-1}) above, $X_1$ is a discrete, where in
(\ref{sec3-example-2}) above, $X_2$ is not.
\end{example}

If $X$ is an $\mathbb{R}^d$-valued random variable defined on a sample space
$S$, then for any $A \subseteq \mathbb{R}^d$, $X^{-1}(A) = \lbrace w \in S
\vert X(w) \in A \rbrace$ is an event. The event $X^{-1}(A)$ is written simply
as $\lbrace X \in A \rbrace$. If $O$ is a probability on $S$, then we write
$P(X \in A)$ to mean $P(\lbrace w \in S \vert X(w) \in A \rbrace)$.

\begin{definition}
    If $X$ is a discrete random variable defined on a sample space $S$ with
associated probability $P$ and the set of all possible values of $X$ are $x_1,
x_2, x_3, \dots$, then the probability distribution of $X$ is represented by
the collection $p(x_i) = P(X = x_i), i = 1, 2, 3, \dots$
\end{definition}

To compute the probability distribution of a discrete random variable $X$, one
can follow these steps:
\begin{itemize}[noitemsep, topsep=0em]
    \item List all possible values of $X$ as $x_1, x_2, x_3, \dots$
    \item For each $i$, compute $P(X = x_i)$. Then the distribution of $X$ will
          be given by $p(x_i) = P(X = x_i), i = 1, 2, 3, \dots$ 
\end{itemize}

\begin{example}
    Toss a fair coin twice, let $X = \text{ number of tails }$. Find the
distribution of $X$.
\end{example}
\begin{solution}
    The possible values of $X$ are $0$, $1$ and $2$. Thus the distribution of
$X$ is given by, 
\begin{align*}
    p(0) &= P(X = 0) = P(HH) = \frac{1}{4}                                   \\
    p(1) &= P(X = 1) = P(HT) + P(TH) = \frac{1}{4}+\frac{1}{4} = \frac{1}{2} \\
    p(2) &= P(X = 2) = P(TT) = \frac{1}{4}
\end{align*}
\end{solution}

\begin{example}
    Roll a fair die until $6$ appears. Let $X$ equal the number of required
tosses. Find the distribution $X$.
\end{example}
\begin{solution}
    Possible values of $X$ are $1, 2, 3, \dots$, the distribution of $X$ is
given by, 
\begin{align*}
    p(i) &= P(X = i)                                                         \\
         &= P(\bigcap_{k=1}^{i-1} \lbrace \text{not $6$ in $k$-th toss} \rbrace
            \cap \lbrace \text{$6$ in $i$-th toss} \rbrace)                  \\
         &= \prod_{k=1}^{i-1} P(\text{not $6$ in $k$-th toss}) \cdot
            P(\text{$6$ in $i$-th toss})                                     \\
         &= (\frac{5}{6})^{i-1} \cdot \frac{1}{6}          & i = 1, 2, 3, \dots
\end{align*}
\end{solution}

\begin{example}
    Let $S$ be a sample space and $E$ be an event. Define the random variable
$1_E$ as, 
\begin{equation*}
    1_E(w) = \begin{cases}
        0 & w \notin E                                                      \\
        1 & w \in E 
    \end{cases}
\end{equation*}
find the distribution of $1_E$
\end{example}
\begin{solution}
    Possible values of $1_E$ are $0$ and $1$. Thus the distribution of $1_E$ is
given by, 
\begin{align*}
    p(0) &= P(1_E = 0)
          = P(\lbrace w \vert 1_E(w) = 0\rbrace)
          = P(E^\complement) 
          = 1 - P(E)                                                         \\
    p(1) &= P(1_E = 1) = P(E)
\end{align*}
\end{solution}

\begin{example}
    Toss a fair coin and roll a fair die independently. Let,
    \begin{equation*}
        X_1 = \begin{cases}
            0 & \text{toss results in tails}                                 \\
            1 & \text{toss results in heads}
        \end{cases}
    \end{equation*}
    and, 
    \begin{equation*}
        X_2 = \text{the number on the uppermost face of the die}
    \end{equation*}.
    Let $X = (X_1, X_2)$ (so $X$ is a $\mathbb{R}^2$-valued random variable).
What is the distribution of $X$?
\end{example}
\begin{solution}
    possible values of $X$ are $\langle i, j\rangle$, where $i \in \lbrace 0, 1
\rbrace$ and $j \in \lbrace 1, 2, \dots, 6 \rbrace$, and for any such $i, j$,
\begin{equation*}
    p(\langle i,j \rangle) = P(X = \langle i,j \rangle)
                           = \frac{1}{2} \cdot \frac{1}{6}
                           = \frac{1}{12}
\end{equation*}
This specifies the distribution of $X$.
\end{solution}

\begin{theorem}
    If $X$ is a discrete random variable with possible values $x_1, x_2,
\dots$ and $p(x_i), i = 1, 2, \dots$ is the probability distribution of $X$,
then,
\begin{enumerate}[noitemsep, topsep=0em]
    \item \label{sec3-thm-1}
          $0 \leq p(x_i) \leq 1$, and,
    \item \label{sec3-thm-2}
          $\sum_{i = 1, 2, \dots} p(x_i) = 1$
\end{enumerate}
\end{theorem}
\begin{proof}
    As $p(x_i) = P(X = x_i)$ by definition. This immediately proves
(\ref{sec3-thm-1}). Next, since the events $\lbrace X = x_i \rbrace, i = 1, 2,
\dots$ are mutually disjoint,
\begin{equation*}
      \sum_{i = 1, 2, \dots} p(x_i) = \sum_{i = 1, 2, \dots} P(X = x_i)
    = P(X \in \lbrace x_1, x_2, \dots \rbrace)
\end{equation*}
    Now, $\lbrace x_1, x_2, \dots \rbrace$ is the set of all possible values of
$X$. Hence the event $\lbrace w \in S \vert X(w) \in \lbrace x_1, x_2, \dots
\rbrace \rbrace$ is the entire sample space $S$. Therefore $\sum_{i = 1, 2,
\dots} p(x_i) = P(S) = 1$, as claimed in part (\ref{sec3-thm-2}).
\end{proof}

\begin{example}
    A single cell can either die, with probability $0.1$, or split into two
cells, with probability $0.9$ producing a new generation of cells. Each cel in
the new generation dies or splits independently with the same probabilities. If
you start with one cell (generation zero), and let it split/die to produce
cells in generation one, find the distribution of the number of cells in
generation two. 
\end{example}
\begin{solution}
    Let $X_1 = \text{number of cells in generation one}$, and $X_2 =
\text{number of cells in generation two}$. Then, 
\begin{align*}
    P(X_1 = 0) &= 0.1                                                        \\
    P(X_2 = 2) &= 0.9
\end{align*}
The possible values $X_2$ can take one from $0$, $2$ and $4$.
\begin{align*}
    P(X_2 = 0 \vert X_1 = 0) &= 1                                            \\
    P(X_2 = 2 \vert X_1 = 0) &= 0                                            \\
    P(X_2 = 4 \vert X_1 = 0) &= 0                                            \\
    P(X_2 = 0 \vert X_1 = 2) &= (0.1)^2                                      \\
    P(X_2 = 2 \vert X_1 = 2) &= 0.1 \cdot 0.9 + 0.1 \cdot 0.9
                              = 2 \cdot 0.1 \cdot 0.9                        \\
    P(X_2 = 4 \vert X_1 = 2) &= (0.9)^2
\end{align*}
Hence the distribution of $X_2$ is given by,
\begin{align*}
    P(X_2 = 0) &= P(X_2 = 0 \vert X_1 = 0) \cdot P(X_1 = 0) +
                  P(X_2 = 0 \vert x_1 = 2) \cdot P(X_1 = 2)
                = 1 \cdot 0.1 + (0.1)^2 \cdot 0.9
                = 0.109                                                     \\
    P(X_2 = 2) &= P(X_2 = 2 \vert X_1 = 0) \cdot P(X_1 = 0) +
                  P(X_2 = 2 \vert X_1 = 2) \cdot P(X_1 = 2)
                = 0 + 2 \cdot 0.1 \cdot 0.9^2
                = 0.162                                                     \\
    P(X_2 = 4) &= P(X_2 = 4 \vert X_1 = 0) \cdot P(X_1 = 0) +
                  P(X_2 = 4 \vert X_1 = 2) \cdot P(X_1 \ 2)
                = 0 + 0.9*3
                = 0.729
\end{align*}
\end{solution}

\subsection{Functions of Random Variable}
If $X : S \rightarrow \mathbb{R}^m$ is an random variable and $g : \mathbb{R}^m
\rightarrow \mathbb{R}^n$ is a function, then $g(x)$ is an
$\mathbb{R}^n$-valued random variable. If the possible values of $X$ are $x_1,
x_2, \dots$ and $p_X(x_i), i = 1, 2, \dots$ is the distribution of $X$, then
the distribution of $Y = g(x)$ can be computed as follows:
\begin{enumerate}[noitemsep, topsep=0em]
    \item list all possible values $y_1, y_2, \dots$ of $Y$, 
    \item for each $j$, find the set 
          \[ E_j = \lbrace x \vert x \in \lbrace x_1, x_2, \dots \rbrace, 
                                   g(x) = y_i \rbrace,                       \]
    \item the distribution of $Y$ is given by,
          \begin{equation*}
              p_Y(y_j) = \sum_{x \in E_j} p_X(x), \quad j = 1, 2, 3, \dots
          \end{equation*}
\end{enumerate}

\begin{example}
    The distribution of $X$ is given by,
    \begin{align*}
        &p_X(-3) = 1 / 16                    &p_X(-2) = 2 / 16              \\
        &p_X(-1) = 1 / 16                    &p_X(0)  = 3 / 16              \\
        &p_X(1)  = 4 / 16                    &p_X(2)  = 3 / 16              \\
        &p_X(4)  = 2 / 16                    &\quad
    \end{align*}
\end{example}
\begin{solution}
    The possible values of $Y$ are $0, 1, 4, 9, 16$,
    \begin{align*}
        p_Y(0) &= p_X(X = 0) = 3 / 16                                      \\
        p_Y(1) &= p_X(X = -1) + p_X(X = 1) = 1 / 16 + 4 / 16 = 5 / 16      \\
        p_Y(4) &= p_X(X = -2) + p_X(X = 2) = 2 / 16 + 3 / 16 = 5 / 16      \\
        p_Y(9) &= p_X(X = -3) = 1 / 16                                     \\
        p_Y(16) &= p_X(X = 4) = 2 / 16
    \end{align*}
\end{solution}
\subsection{Expectation of a Discrete Random Variable}
\begin{definition}
        If $X$ is a discrete random variable with probability distribution
    $p(x_i), i = 1, 2, \dots$, then the expectation of $X$ is given by,
    \begin{equation*}
        E[X] = \sum_{i = 1, 2, \dots} x_i \cdot p(x_i)
    \end{equation*}
        If $X$ is $\mathbb{R}^d$-valued, then $E[X]$ is a number in
    $\mathbb{R}^d$.
\end{definition}

\noindent
\textbf{Note}: If $X$ is $\mathbb{R}$-valued and $x_1^+, x_2^+, \dots$ are the
positive values assumed by, and $x_1^-, x_x^-, \dots$ are the negative values
assumed by $X$, then $E[X]$ is only defined when either $\sum_{i = 1, 2,
\dots} x_i^+ \cdot p(x_i^+) < \infty$ or $\sum_{i=1, 2, \dots} x_i^- \cdot
p(x_i^-) < \infty$. If both of these quantities are infinite, then we say
expectation of $X$ does not exist. Similar consideration apply to
$\mathbb{R}^d$-valued random variables.

\begin{example}
    Roll a fair die until $6$ appears and let $X$ be the number of required
tosses. We have already computed the distribution of $X$ as $p(i) =
(\frac{5}{6})^{i-1} \cdot \frac{1}{6}$, $i = 1, 2, \dots$. Hence, 
\begin{equation*}
    E[X] = \sum_{i \geq 1} i \cdot (\frac{5}{6})^{i-1} \cdot \frac{1}{6}    
         = \frac{1}{6} \cdot (1 - \frac{5}{6})^{-2}                         
         = \frac{1}{6} \cdot 36
         = 6
\end{equation*}
\end{example}

\begin{example}
    If $E$ is an event and $1_E$ is the random variable defined by,
    \begin{equation*}
        1_E(w) = \begin{cases}
            1 & w \in E                                                      \\
            0 & w \notin E
        \end{cases}
    \end{equation*}
    then, 
    \begin{equation*}
        E[1_E] = 1 \cdot P(E) + 0 \cdot P(E^\complement) 
               = P(E)
    \end{equation*}
\end{example}

\begin{example}
    $k$ buses carrying a total of $n_1 + n_2 + \dots + n_k$ students set off on
a trip. The $j$-th bus is carrying $n_j$ many students.
\begin{enumerate}[noitemsep, topsep=0em]
    \item 
    Upon arrival, one student is chosen at random, and every student on the
same bus as the selected student is given a candy. If each candy costs $\$1$
and $X$ denotes the total cost of the candy, find the distribution of $X$ and
compute $E[X]$.
    \item
    Upon arrival, one buses out of $k$-bus is chosen at random and each student
on that bus is given a candy. If each candy costs $\$1$ and $Y$ denotes the
cost of the candy, find the distribution of $Y$ and compute $E[Y]$.
    \item
    Show that $E[X] \geq E[Y]$.
\end{enumerate}
    Assume $n_i \neq n_j, \quad \forall i \neq j$.
\end{example}
\begin{solution} \quad                                                       \\
    \begin{enumerate}[noitemsep, topsep=0em]
        \item 
        possible values of $X$ are $n_1, n_2, \dots, x_k$ (By assumption these
        are all distinct).
        \begin{equation*}
            p_X(n_i) = P(X = n_i) = \frac{n_i}{n_1 + \dots +n_k} 
            \quad i = 1, 2, \dots, k
        \end{equation*}
        Hence, 
        \begin{equation*}
            E[X] = \sum_{i = 1}^k n_i \cdot p_X(n_i)
                 = \sum_{i = 1}^k n_i \cdot \frac{n_i}{n_1 + \dots + n_k}
                 = \sum_{i = 1}^k \frac{(n_i)^2}{n_1 + \dots + n_k}
                 = \frac{\sum_{i = 1}^k (n_i)^2}{\sum_{i = 1}^k n_i}
        \end{equation*}
        
        \item
        possible values of $Y$ are $n_1, n_2, \dots, n_k$ and,
        \begin{equation*}
            p_Y(n_i) = P(Y = n_i) = \frac{1}{k} \quad i = 1, 2, \dots, k
        \end{equation*}
        Hence,
        \begin{equation*}
            E[Y] = \sum_{i = 1}^k n_i \cdot p_Y(n_i)
                 = \frac{\sum_{i = 1}^k n_i}{k}
        \end{equation*}
        
        \item
        Since $\sum_{i=1}^k (n_i)^2 \geq \frac{(\sum_{i=1}^k n_i)^2}{k}$ by
        Cauchy-Schwarz inequality. It follows that, 
        \begin{equation*}
            E[X] \geq E[Y]
        \end{equation*}
    \end{enumerate}
\end{solution}

\begin{theorem}
    Let $X$ be on $\mathbb{R}^m$-valued discrete random variable and $g :
\mathbb{R}^m \rightarrow \mathbb{R}^n$ be a mapping function. If the set of
possible values of $X$ are $x_1, x_2, \dots$ and the distribution of $X$ is
given by $p_X(x_i)$, $i = 1, 2, \dots$, then
\begin{equation*}
    E[g(X)] = \sum_{i \geq 1} g(x_i) \cdot p_X(x_i)
\end{equation*}
\end{theorem}
\begin{proof}
    If the possible values of $Y = g(X)$ are $y_1, y_2, \dots$, then the
probability distribution of $Y$ is given by,
\begin{equation*}
    p_Y(y_j) = \sum_{i : g(x_i) = y_j} p_X(x_i) \quad j = 1, 2, \dots
\end{equation*}
Hence,
\begin{align*}
    E[g(X)] &= E[Y]                                                          \\
            &= \sum_{j \geq 1} y_j \cdot p_Y(y_j)                            \\
            &= \sum_{j \geq 1} \sum_{i : g(x_i) = y_j} g(x_i) \cdot p_X(x_i) \\
            &= \sum_{i \geq 1} g(x_i) \cdot p_X(x_i)
\end{align*}
as claimed.
\end{proof}

\begin{example}
    The distribution of $X$ is given by,
    \begin{align*}
        &p_X(-2) = 1 / 10                     &p_X(-1) = 2 / 10              \\
        &p_X(0) = 2 / 10                      &p_X(1) = 3 / 10               \\
        &p_X(2) = 2 / 10                      &
    \end{align*}
    Then,
    \begin{align*}
        E[X^2] &= (-2)^2 \cdot \frac{1}{10} +
                  (-1)^2 \cdot \frac{2}{10} +
                  0^2 \cdot \frac{2}{10} +
                  1^2 \cdot \frac{3}{10} +
                  2^2 \cdot \frac{2}{10}                                     \\
               &= \frac{4 + 2 + 0 + 3 + 8}{10}                               \\
               &= \frac{17}{10}
    \end{align*}
    This is often easier than finding the distribution of $X^2$ explicitly and
then computing $E[X^2]$.
\end{example}

\begin{theorem}
    If $X$ is an $\mathbb{R}^m$-valued discrete random variable and $g_i :
\mathbb{R}^m \rightarrow \mathbb{R}^n$, for $i = 1, 2, \dots, k$, then,
\begin{equation*}
    E[g_1(X) + g_2(X) + \dots + g_k(X)]
    = E[g_1(X)] + E[g_2(X)] + \dots + E[g_k(X)]
\end{equation*}
\end{theorem}
\begin{proof}
    \begin{align*}
           E[g_1(X) + g_2(X) + \dots + g_k(X)]
        &= \sum_{x} (g_1(x) + g_2(x) + \dots + g_k(x)) \cdot p_X(x)          \\
        &= \sum_{i = 1}^k \sum_{x} g_i(x) \cdot p_X(x)                       \\
        &= \sum_{i = 1}^k E[g_i(X)]
    \end{align*}
\end{proof}

\begin{theorem}[linearity of expectation]
    If $X$ is an $\mathbb{R}^d$ valued random variable, $b \in \mathbb{R}^d$,
and $a \in \mathbb{R}$, then 
\begin{equation*}
    E[aX + b] = aE[X] + b
\end{equation*}
\end{theorem}
\begin{solution} \quad                                                       \\
    \begin{equation*}
        E[aX+b] = E[aX] + E[b]
    \end{equation*}
    Then the expectation of the constant random variable $Y = b$ is 
    \begin{equation*}
        \begin{cases}
            E(Y) = b \cdot P(Y = b) = b                                      \\
            E(aX) = \sum_{x} a \cdot x \cdot p_X(x) 
                  = a \sum_{x} x \cdot p_X(x)
                  = a E[X]
        \end{cases}
        \Rightarrow
        E[aX + b] = E[aX] + E[b] = aE[X] + b
    \end{equation*}
    Thus the claim follows.
\end{solution}

\begin{example}
    If $X$ is a real-valued discrete random variable, then
    \begin{equation*}
          E[X^2 + 2 \sin{X} + 3]
        = E[X^2] + 2 E[\sin{X}] + 3
    \end{equation*}
\end{example}

\begin{theorem}
    If $X_1, X_2, \dots, X_k$ are $\mathbb{R}^d$-valued discrete randome
variables defined on the same sample space $S$ then,
\begin{equation*}
    E[X_1 + X_2 + \dots + X_k] = E[X_1] + E[X_2] + \dots + E[X_k]   
\end{equation*}
\end{theorem}
\begin{proof}
    Let $X = \langle x_1, x_2, \dots, x_k \rangle : S \rightarrow
\mathbb{R}^{kd}$ be a discrete random variable. Further, letting $g_i(X) =
X_i$, we see that,
\begin{align*}
    E[X_1 + X_2 + \dots + X_k] &= E[g_1(X) + g_2(X) + \dots + g_k(X)]        \\
                               &= \sum_{i = 1}^k E[g_i(X)]                   \\
                               &= \sum_{i = 1}^k E[X_i]
\end{align*}
\end{proof}
This is an \textbf{extremely useful fact}. Note that this theorem \textbf{does
not} require any assumption of independence.

\begin{example} \quad                                                        \\
\begin{itemize}[noitemsep, topsep=0em]
    \item
    If $X$ is a discrete real-valued random variable that is nonnegative (i.e.
the values assumed by $X$ are nonnegative), then $E[X] \geq 0$.
    \item
    If $X$ and $Y$ are two discrete real-valued random variables defined on the
same sample space and $X \geq Y$, then $E[X] \geq E[Y]$.
\end{itemize}
\end{example}
\begin{solution} \quad                                                       \\
\begin{itemize}[noitemsep, topsep=0em]
    \item 
    If the possible value of $X$ are $x_1, x_2, \dots$, then $x_i \geq 0 \quad
\forall i \geq 1$. Hence $E[X] = \sum_{i \geq 1} x_i \cdot p_X(x_i) \geq 0$.
    \item
    Consider $E[X- Y] = E[X] + E[-Y] = E[X] - E[Y]$. Since, $X \geq Y$, then
for any possible out come $x_1, x_2, \dots$ and $y_1, y_2, \dots$, we have
$\forall i \geq 1 \quad x_i \geq y_i$. Thus $X - Y$ is a nonnegative discrete
random variable, and by previous example, we have $E[X - Y] \geq 0 \Rightarrow
E[X] - E[Y] \geq 0 \Rightarrow E[X] \geq E[Y]$.
\end{itemize}
\end{solution}

\begin{example}
    Let $X$ be the number of times heads appears when a coin is tossed $n$
times. Assume that probability of landing heads in a single toss is $p$. Find
$E[X]$.
\end{example}
\begin{solution}
    Define random variable $X_i$ as below,
    \begin{equation*}
        X_i = \begin{cases}
            1   & i\text{-th toss results in heads}                          \\
            0   & \text{otherwise}
        \end{cases}
    \end{equation*}
    Then $E[X_i] = P(\text{heads in }i\text{-th toss}) = p$. Since $X = X_1 +
X_2 + \dots + X_n$,
    \begin{equation*}
        E[X] = E[X_1 + X_2 + \dots + X_n]
             = E[X_1] + E[X_2] + \dots + E[X_n]
             = n \cdot p
    \end{equation*}
\end{solution}
\note Assume the tosses to be independent one could write down the
distribution of $X$ and then compute the expectation. We will do it later.

\begin{theorem}
    If $X_1, X_2, \dots$ are $\mathbb{R}$-valued random variables defined on
the same sample space and each $X_i$ is nonnegative, then
    \begin{equation*}
        E[X_1 + X_2 + \dots] = \sum_{i = 1}^\infty E[X_i]
    \end{equation*}
\end{theorem}
\note Since $X_i \geq 0 \quad \forall i$, the series $\sum_{i = 1}^\infty$
always makes sense. It can converge to a finite real value or it can be
infinite. Same is true for $\sum_{i = 1}^\infty E[X_i]$.
\subsection{Variance of an Random Variable}
\begin{definition}
    If $X$ is an $\mathbb{R}$-valued random variable with $E[X^2] < \infty$,
then the variance of $X$ is defined as,
    \begin{equation*}
        V[X] = E[(X - E(X))^2]
    \end{equation*}
    It is also often denoted by $\sigma_X^2$. The standard deviation of $X$ is
defined as $\sqrt{V[X]} = \sigma_X$.
\end{definition}
Variance of an random variable is a measure of how dispensed it is around its
mean. 

\begin{theorem}
    \begin{equation*}
        V[X] = E[X^2] - (E[X])^2
    \end{equation*}
\end{theorem}
\begin{proof}
    \begin{align*}
        V[X] &= E[(X - E[X])^2]                                              \\
             &= E[X^2 - 2X \cdot E[X] + (E[X])^2]                            \\
             &= E[X^2] - 2 \cdot E[X] \cdot E[X] + (E[X])^2                  \\
             &= E[X^2] - (E[X])^2
    \end{align*}
\end{proof}

\begin{theorem}
    If $a, b \in \mathbb{R}$, then,
    \begin{equation*}
        V[aX + b] = a^2 V[X]
    \end{equation*}
\end{theorem}
\begin{proof}
    Let $Y = aX + b$, then, 
    \begin{align*}
                     E[Y] = aE[X] + b
        &\Rightarrow Y - E[Y] = a(X - E[X])                                  \\
        &\Rightarrow V[Y] = E[(Y - E[Y])^2]
                          = E[a^2 \cdot (X - E[X])^2]
                          = a^2 \cdot E[(X - E[X])^2]
                          = a^2 \cdot V[X]
    \end{align*}
\end{proof}

\begin{example}
    If $X$ has discrete uniform distribution on $\lbrace 1, 2, \dots, n
\rbrace$, i.e., 
    \begin{equation*}
        P(X = i) = \frac{1}{n} \quad i = 1, 2, \dots, n
    \end{equation*}
    Find $E[X]$ and $\sigma_X$.
\end{example}
\begin{solution}
    \begin{align*}
        & E[X] = \sum_{i = 1}^n \frac{i}{n} 
               = \frac{n \cdot (n+1)}{2n} 
               = \frac{n+1}{2}                                               \\
        & E[X^2] = \sum_{i = 1}^n \frac{i^2}{n}
                 = \frac{n \cdot (n_1) \cdot (2n+1)}{6n}
                 = \frac{(n+1) \cdot (2n+1)}{6}                              \\
        & V[X] = E[X^2] - (E[X])^2
               = \frac{(n+1) \cdot (2n+1)}{6} - (\frac{n+1}{2})^2
               = \frac{n+1}{12} \cdot (4n + 2 - 3n -3)
               = \frac{n^2 - 1}{12}                                          \\
        & \sigma_X = \sqrt{v[X]} = \sqrt{\frac{n^2 - 1}{12}}
    \end{align*}
\end{solution}

\begin{example}
    If $X$ denotes the number of the uppermost face while rolling a fair die,
then $X$ follows a discrete uniform distribution on $\lbrace 1, 2,
\dots, 6 \rbrace$. Find $E[X], \sigma_X$.
\end{example}
\begin{solution}
    \begin{equation*}
        E[X] = \frac{6 + 1}{2} = \frac{7}{2} \qquad
        V[X] = \frac{6^2 - 1}{12} = \frac{35}{12} \qquad
        \sigma_X = \frac{35}{12}
    \end{equation*}
\end{solution}
\subsection{Binomial Distribution}
\begin{definition}
    Let $n \geq 1$ and $p \in [0, 1]$. An random variable $X$ follows a
binomial distribution with parameters $n$ and $p$, in short $X \sim
\binomialdist{n}{p}$, if the probability mass function (pmf) of $X$ is given
by,
    \begin{equation*}
        p_X(k) = P(X = k) = {n \choose k}p^k \cdot (1-p)^{n-k}
        \quad
        k = 0, 1, \dots, n
    \end{equation*}
\end{definition}
\noindent
\textbf{Interpretation}: Suppose an experiment consists of $n$ identical trials
such that, 
\begin{itemize}[noitemsep, topsep=0em]
\item
    each trial results in one of two possible outcomes: success($S$) or failure
    ($F$);
\item
    possibility of success in each trial is $p$ ( and hence probability of
    failure is $q = 1 - p$);
\item
    the trials are independent.
\end{itemize}
Then the number of successes in the experiment follows $\binomialdist{n}{p}$
distribution.

\subsubsection*{Deriving the Binomial Probability Mass Function}
Suppose probability of landing head while tossing a coin $\frac{1}{4}$. Toss
the coin $3$ times independently. The sample space will be $S = \lbrace HHH,
HHT, HTH, THH, HTT, THT, TTH, TTT \rbrace$. Then, 
\begin{align*}
    P(HHH) &= \frac{1}{4}^3                                                  \\
    P(HHT) &= P(HTH) = P(THH) = (\frac{1}{4})^2 \cdot \frac{3}{4}            \\
    P(HTT) &= P(THT) = P(TTH) = \frac{1}{4} \cdot (\frac{3}{4})^2            \\
    P(TTT) &= (\frac{3}{4})^3
\end{align*}
Thus, probability of any sample point that has $k$ heads and $3-k$ tails (in
any order) is $(\frac{1}{4})^k \cdot (\frac{3}{4})^{3-k}$. Let $X =
\#\text{heads in three tosses}$. For $k = 0, 1, 2, 3$, 
\begin{align*}
    P(X = k) &= (\frac{1}{4})^k \cdot (\frac{3}{4})^{3 - k} \cdot
                (\text{the number of sample points that have $k$ heads and
                $3-k$ tails})                                                \\
             &= (\frac{1}{4})^k \cdot (\frac{3}{4})^{3 - k} \cdot
                {3 \choose k}
\end{align*}
Same reasoning applies to general $n$ and $p$.

\begin{example}
    $100$ couples live in a community. Each couple can have either $0$, $1$ or
$2$ children with probability $\frac{1}{3}$ each. Assume that the couples have
children independently. Find the distribution of $X = \text{number of couples
that have at least one child}$.
\end{example}
\begin{solution}
    \[ X \sim \binomialdist{100}{\frac{2}{3}} \]
\end{solution}

\subsubsection*{Bernoulli Distribution}
$\binomialdist{1}{p}$ distribution is called the Bernoulli distribution with
parameter $p$. We will write it as $\bernoullidist{p}$.Thus, if $Y \sim
\bernoullidist{p}$, then $P(Y = 1) = p$, and $P(Y = 0) = 1 - p$.

The total number of success $X$ in $n$ independent identical trials each
having probability of of success $p$, follows $\binomialdist{n}{p}$. Let $Y_i =
1_{\lbrace i\text{-th trial results in success} \rbrace}$. Then $Y_i \sim
\bernoullidist{p} \quad i = 1, 2, \dots, n$, and $X = \sum_{i = 1}^n Y_i$.

\begin{theorem}
    If $X \sim \binomialdist{n}{p}$, then,
    \begin{enumerate}[noitemsep, topsep=0em]
        \item $E[X] = n \cdot p$,
        \item $V[X] = n \cdot p \cdot (1 - p)$.
    \end{enumerate}
\end{theorem}
\begin{proof}
    As $X = \sum_{i = 1}^n Y_i$, where $Y_i \sim \bernoullidist{p}$,
    \[ E[Y_i] = 1 \cdot p + 0 \cdot (1 - p) = p \]
    \[ \Rightarrow 
       E[X] = E[\sum_{i = 1}^n Y_i] = \sum_{i = 1}^n E[Y_i] = n \cdot p \]
\end{proof}
\begin{proof}
    \begin{align*}
        E[X] &= \sum_{k = 0}^n k \cdot P(X = k)                              \\
             &= \sum_{k = 0}^n k \cdot {n \choose k} \cdot p^k \cdot 
                (1-p)^{n-k}                                                  \\
             &= \sum_{k = 1}^n k \cdot \frac{n!}{k!(n-k)!} \cdot p^k \cdot
                (1-p)^{n-k}                                                  \\
             &= \sum_{k = 1}^n \frac{n!}{(k-1)! (n-k)!} \cdot p^k \cdot 
                (1-p)^{n-k}                                                  \\
             &= n \cdot p \cdot \sum_{k = 1}^n \frac{(n - 1)!}{(k-1)!(n-k)!}
                \cdot p^{k - 1} \cdot (1-p)^{n-k}                            \\
             &= n \cdot p \cdot (p + (1 - p))^{n-1}                          \\
             &= n \cdot p
    \end{align*}
    \begin{align*}
        E[X(X - 1)] &= \sum_{k = 0}^n k \cdot (k - 1) \cdot P(X = k)        \\
                    &= \sum_{k = 2}^n k \cdot (k - 1) \cdot
                       {n \choose k} \cdot p^k \cdot (1-p)^{n-k}            \\
                    &= \sum_{k = 2}^n \frac{n!}{(k - 2)!(n - k)!} \cdot
                       p^k \cdot (1 - p)^{n - k}                            \\
                    &= n(n-1)p^2 \sum_{k = 2}^n \frac{(n-2)!}{(k-2)!(n-k)!}
                       \cdot p^{k - 2} \cdot (1-p)^{n - k}                  \\
                    &= n \cdot (n-1) \cdot p^2 \cdot (p + (1 - p))^{n - 2}  \\
                    &= n \cdot (n-1) \cdot p^2
    \end{align*}
    \begin{equation*}
        E[X^2] = E[X(X - 1)] + E[X] 
               = E[X(X - 1)] + n \cdot p
               = n \cdot (n - 1) \cdot p^2 + n \cdot p
    \end{equation*}
    \begin{align*}
        V[X] &= E[X^2] - (E[X])^2                                           \\
             &= n \cdot (n - 1) \cdot p^2 + n \cdot p - n^2 \cdot p^2       \\
             &= n^2p^2 - np^2 + np - n^2p^2                                 \\
             &= np(1-p)
    \end{align*}
\end{proof}

\begin{example}[exercise 3.58 of textbook]
    Let $X$ denote the number of defectives among four items selected randomly
from a large set that is known to contain $10\%$ defectives. If the repair cost
for the defectives is given by $C = 3X^2 + X + 2$, find $E[C]$.
\end{example}
\begin{solution}
    $X \sim \binomialdist{n}{p}$ where $n = 1$, and $p = 0.1$.Then, 
    \begin{align*}
        E[C] &= E[3X^2 + X + 2]                                             \\
             &= E[3X^2] + E[X] + 2                                          \\
             &= 3 (V[X] + (E[X])^2) + E[X] + 2                              \\
             &= 3 (np(1-p) + n^2p^2) + np + 2                               \\
             &= 3n^2p^2 - 3np^2 + 4np + 2                                   \\
             &= 3.96
    \end{align*}
\end{solution}

\begin{example}
    A communication system consists of $n$ components, each of which will
independently function with probability $p$. The system operates effectively if
at least half of its components function. For what values of $p$ is a
$5$-component system more likely to operate effectively than a $3$-component
system?
\end{example}
\begin{solution}
    The number of functioning components $\sim \binomialdist{n}{p}$.
Probability that a $5$-component system will be effective is,
\[ {5 \choose 3} p^3 (1-p)^2 + {5 \choose 4} p^4 (1 - p) + p^5 \]
Probability that a $3$-component system operates effectively is,
\[ {3 \choose 2} p^2 (1-p) + p^3 \]
A $5$-component system is more likely to operate effectively if and only if
\begin{align*}
    {5 \choose 3} p^3 (1-p)^2 + {5 \choose 4} p^4 (1 - p) + p^5 &>
    {3 \choose 2} p^2 (1-p) + p^3                                           \\
    10p^3(1-p)^2 + 5p^4(1-p) + p^5 &> 3p^2(1-p) + p^3 + p^3                 \\
    (1-p)(10p - 10p^2 + 5p^2 - 3 - p - p^2) &> 0                            \\
    (1-p)(9p - 6p^2 - 3) = (1-p)(3p - 2p^2 - 1) = (1-p)(1-p)(2p-1) &> 0     \\
\end{align*}
Hence, $p > \frac{1}{2}$
\end{solution}
\subsection{Poisson Distribution}
\begin{definition}
    An random variable $X$ is said to follow Poisson distribution with
parameter $\lambda$ ($\lambda > 0$), in short $X \sim \poissondist{\lambda}$,
if the pmf of $X$ is given by, 
\begin{equation*}
    p_X(k) = P(X = k) = e^{-\lambda} \cdot \frac{\lambda^k}{k!}
    \quad k = 0, 1, 2, \dots
\end{equation*}
\end{definition}
A Poisson random variable approximates the total number of successes in a large
number of independent trials each of which has a very small probability of
success. 

\begin{example}
    If probability of hooking a fish at each attempt is $0.04$, then the number
of fish caught after $50$ attempts follows $\binomialdist{50}{0.04} \approx
\poissondist{2}$.
\end{example}

\begin{theorem}
    Let $X_n \sim \binomialdist{n}{p_n}$ where $n \cdot p_n \xrightarrow{n
\rightarrow \infty} \lambda$ for some $\lambda \in (0, \infty)$. Then for each
$k = 0, 1, 2, \dots$,
    \[ P(X_n = k) \rightarrow   e^{-\lambda} \cdot \frac{\lambda^k}{k!}
                              = P(Y = k)                                     \]
    where $Y \sim \poissondist{\lambda}$.
\end{theorem}
\begin{proof}
\begin{align*}
    P(X_n = k) &= {n \choose k} p_n^k (1 - p_n)^{n - k}                      \\
               &= \frac{n(n-1) \cdot (n - k + 1)}{k!} \cdot p_n^k \cdot 
                  (1 - p_n)^n \cdot (1 - p_n)^{-k}                           \\
               &\xrightarrow{n \rightarrow \infty} 
                  \frac{1}{k!} \cdot 1 \cdot \lambda^k \cdot e^{-\lambda} 
                  \cdot 1                                                    \\
               &= e^{-\lambda} \frac{\lambda^k}{k!}  
\end{align*}
Note that $n \cdot p_n \xrightarrow{n \rightarrow \infty} \lambda \in (0,
\infty)$, and $p_n \xrightarrow{n \rightarrow \infty} 0$.
\end{proof}

\begin{theorem}
    If $X \sim \poissondist{\lambda}$, then,
    \begin{equation*}
        E[X] = \lambda \qquad V[X] = \lambda
    \end{equation*}
\end{theorem}
\begin{proof}
    \begin{align*}
        & E[X] = \sum_{k = 0}^\infty k \cdot P(X = k)                         
               = \sum_{k = 1}^\infty k \cdot e^{-\lambda} \cdot
                 \frac{\lambda^k}{k!}                                         
               = e^{-\lambda} \cdot \sum_{k = 1}^\infty
                 \frac{\lambda^k}{(k-1)!}                                     
               = e^{-\lambda} \cdot \lambda \cdot \sum_{k = 1}^\infty
                 \frac{\lambda^{k-1}}{(k-1)!}                                 
               = e^{-\lambda} \cdot \lambda \cdot e^\lambda 
               = \lambda                                                     \\
        & E[X(X-1)] = \sum_{k = 2}^\infty k(k-1) \cdot P(X = k)
                    = \sum_{k = 2}^\infty k(k-1) e^{-\lambda} \cdot
                      \frac{\lambda^k}{k!}
                    = e^{-\lambda} \cdot \sum_{i = 2}^\infty
                      \frac{\lambda^k}{(k-2)!}
                    = \lambda^2 \cdot e^{-\lambda} \cdot e^\lambda
                    = \lambda^2                                              \\
        & E[X^2] = E[X(X - 1)] + E[X] = \lambda^2 + \lambda                  \\
        & V[X] = E[X^2] - (E[X])^2  
               = \lambda^2 + \lambda - \lambda^2
               = \lambda
    \end{align*}
\end{proof}

\begin{example}
    A book has $500$ pages. The probability that a typographical error occurs
in a page is $0.005$. Use Poisson approximation to find the probability that
not more than one page contains a typo.
\end{example}
\begin{solution}
    Let $X$ denote the number of pages that contain a typo. Then $X
\sim \binomialdist{500}{0.005}$. Hence,
    \[ P(X \leq 1) = P(X = 0) + P(X = 1) \approx P(Y = 0) + P(Y = 1) \]
    where, $Y \sim \poissondist{2.5}$ ($500 \cdot 0.005 = 2.5$). Hence,
    \[ 
       P(X \leq 1) = e^{-2.5} \cdot (1 + \frac{2.5}{1!})
                   = 3.5 \cdot e^{-2.5}
    \]
\end{solution}

\begin{example}
    An item is a store has a marked price of $\$100$. For each customer
purchasing the item during a particular day, the store owner reduces the price
of the item by a factor of one-half (i.e. the first customer pays $\$50$, the
second pays $\$25$ etc.). If the number of customers who purchase the item
during the day follows a Poisson distribution with mean $2$, find the expected
price of the item at the end of the day.
\end{example}
\begin{solution}
    If $Y$ denotes the number of customers who purchase that item during the
day, then,
    \[ Y \sim \poissondist{2} (\lambda = 2 \text{ as mean equals two.}) \]
    The cost at the end of the day is $C = 100 \cdot (\frac{1}{2})^Y$. Hence, 
    \[
        E[(\frac{1}{2})^Y] = \sum_{k = 0}^\infty (\frac{1}{2})^k \cdot
                             e^{-2} \cdot \frac{2^k}{k!}
                           = \sum_{k = 0}^\infty e^{-2} \cdot \frac{1}{k!}
                           = e^{-2} \cdot e 
                           = e^{-1}
    \]
    \[
        E[C] = E[100 \cdot (\frac{1}{2})^Y]
             = 100 \cdot E[(\frac{1}{2})^Y]
             = 100 \cdot e^{-1}
    \]
\end{solution}

\begin{example}
    A radioactive piece of rock emits particles at random time intervals. Let
$X_1$ denote the time at which the first particle is emitted. If the number of
particles emitted in a time interval of length $t$ follows a Poisson
distribution with mean $\lambda t$, find $P(X_1 > x)$ for $x > 0$.
    \begin{figure*}[!htp]
        \centering
        \def\svgwidth{\textwidth}
        \includesvg[./section3/figure/]{sec3-sub5-fig1}
    \end{figure*}
\end{example}
\begin{solution}
    Let $N_x \coloneqq \#\text{particles emitted in time interval }[0, x]$.
Then, 
    \[ N_x \sim \poissondist{\lambda x} \] 
    Now, 
    \[P(X_1 > x) = P(N_x = 0) = e^{-\lambda x} \]
\end{solution}
\subsection{Geometric Distribution}
\begin{definition}
    An random variable $X$ follows a geometric distribution with probability of
success $p$, or simply $X \sim \geometricdist{p}$, if the pmf of $X$ is given
by, 
    \[ p_X(k) = P(X = k) = p \cdot (1-p)^{k - 1} \quad k = 1, 2, 3, \dots \]
\end{definition}
This is indeed a pmf, since $p_X(k) \geq 0, \forall k \geq 1$, and 
    \[ \sum_{k = 1}^\infty p_X(k) = p \cdot \sum_{k = 1}^\infty (1 - p)^{k - 1}
                                  = \frac{p}{1 - (1 - p)}
                                  = 1                                        \]

\noindent
\textbf{Interpretation}: suppose that, 
\begin{itemize}[noitemsep, topsep=0em]
    \item you have an infinite sequence of independent and identical trials;
    \item each trial can result in either success or failure;
    \item probability of success in each trial is $p$ (hence probability of
          failure in each trial is $q = 1-p$).
\end{itemize}
Let $Y$ denote the first trial that results in a success. Then $Y \sim
\geometricdist{p}$. Then, 
\begin{align*}
    P(Y = k) &= P( \bigcap_{i=1}^{k-1} \lbrace \text{failure in $i$-th trial}
                \rbrace \cap \lbrace \text{success in $k$-th trial} \rbrace) \\
             &= \prod_{i=1}^{k - 1} P(\text{failure in $i$-th trial}) \cdot 
                P(\text{success in $k$-th trial})                            \\
             &= (1-p)^{k-1} \cdot p
\end{align*}

\begin{example}
    Exactly $10^6$ lottery tickets are sold every week. Jim buys $2$ tickets
every week. If $X$ denotes the number of weeks Jim has to wait before he wins
for the first time, what is the distribution of $X$?
\end{example}
\begin{solution}
    \[ X \sim \geometricdist{\frac{2}{10^6}} \]
\end{solution}

\begin{theorem}
    If $X \sim \geometricdist{p}$, then
    \[ E[X] = \frac{1}{p} \qquad V[X] = \frac{1-p}{p^2} \]
\end{theorem}
\begin{proof}
    \[ E[X] = \sum_{k = 1}^\infty k \cdot p \cdot (1-p)^{k - 1}
            = p \cdot \frac{1}{(1 - ( 1 - p))^2}
            = \frac{1}{p}                                                    \]
    \begin{align*}
        E[X^2] &= \sum_{k = 1}^\infty k^2 \cdot p \cdot (1 - p)^{k - 1}      \\
               &= \sum_{k = 1}^\infty k(k - 1) \cdot p \cdot (1 - p)^{k - 1} +
                  \sum_{k = 1}^\infty k \cdot p \cdot (1 - p)^{k - 1}        \\
               &= p(1 - p) \sum_{k = 1}^\infty k(k - 1)(1 - p)^{k - 2} +
                  \frac{1}{p}                                                \\
               &= p(1 - p) \cdot 2 \cdot (1 - (1 - p))^{-3} + \frac{1}{p}    \\
               &= \frac{2(1-p)}{p^2} + \frac{1}{p}                           
                = \frac{2}{p^2} - \frac{1}{p}
    \end{align*}
    \[ V[X] = E[X^2] - (E[X])^2
            = \frac{2}{p^2} - \frac{1}{p} - (\frac{1}{p})^2
            = \frac{1}{p^2} - \frac{1}{p}
            = \frac{1 - p}{p^2}                                              \]
\end{proof}

\begin{example}
    If $Y \sim \geometricdist{p}$, show that, 
    \begin{enumerate}[noitemsep, topsep=0em]
        \item $P(Y > k) = (1 - p)^k$, for $k = 1, 2, 3, \dots$, 
        \item $P(Y > k_1 + k+2 \vert Y > k_1) = (1 - p)^{k_2} = P(Y > k_2)$,
              for $k_1, k_2 = 1, 2, 3, \dots$,
        \item is called the memoryless property of the geometric distribution
    \end{enumerate}
\end{example}
\begin{solution} \quad                                                       \\
    \begin{enumerate}[noitemsep, topsep=0em]
        \item 
        \begin{align*}
           P(Y > k) &= \sum_{j = k + 1}^\infty P(Y = j)                      \\
                    &= \sum_{j = k + 1}^\infty p \cdot (1 - p)^{j - 1}       \\
                    &= p \cdot (1 - p)^k \cdot \sum_{j = k + 1}^\infty        
                       (1 - p)^{j - k - 1}                                   \\
                    &= p (1 - p)^k \cdot \frac{1}{1 - (1 - p)}               \\
                    &= (1 - p)^k                                             
        \end{align*}
        Alternatively,
        \[ P(Y > k) = P(\text{zero successes in first $k$ trials})
                    = (1 - p)^k                                              \]
        \item
        \[   P(Y > k_1 + k_2 \vert Y > k_1)
           = \frac{P(\lbrace Y > k_1 + k_2 \rbrace \cap \lbrace Y > k_1
             \rbrace)}{P(Y > k_1)}  
           = \frac{Y > k_1 + k_2}{P(Y > k_1)}
           = \frac{(1 - p)^{k_1 + k_2}}{(1 - p)^{k_1}}
           = (1 - p)^{k_2}                                                   \]
    \end{enumerate}
\end{solution}

\begin{example}
    Snakes and ladders is a two-players game in which Player $1$ and Player $2$
take turns rolling a fair die, and the player to roll $1$ first gets to move
his chip first. Assume that player $1$ rolls the die first.
\begin{enumerate}[noitemsep, topsep=0em]
    \item What is the probability that Player $1$ moves his chip first?
    \item Given that player $1$ moves first, what is the probability that he
moves on his second roll (overall third roll of the die)?
\end{enumerate}
\end{example}
\begin{solution}
    \begin{enumerate}[noitemsep, topsep=0em]
        \item 
        Let $Y$ be the first time $1$ turns up. Then,
        \begin{align*}
            P(\text{Player $1$ moves first})
                &= P(Y = 1) + P(Y = 3) + P(Y = 5) + \dots                    \\
                &= \sum_{k = 0}^\infty \frac{1}{6}(1 - \frac{1}{6})^{2k}     \\
                &= \frac{1}{6} \sum_{k = 0}^\infty (\frac{5}{6})^{2k}        \\
                &= \frac{\frac{1}{6}}{1 - \frac{25}{36}}
                 = \frac{1}{6} \cdot \frac{36}{11}
                 = \frac{6}{11}
        \end{align*}
        \item
        \[
            P(Y = 3 \vert \text{Player $1$ moves first})
                = \frac{P(Y = 3)}{\frac{6}{11}}                             
                = (\frac{5}{6})^2 \cdot \frac{1}{6} \cdot \frac{11}{6}      
                = \frac{275}{1296}
        \]
    \end{enumerate}
\end{solution}
\subsection{Negative Binomial Distribution}
This is a generalization of the geometric distribution. 
\begin{definition}
    An random variable $Y$ follows a negative binomial distribution with
parameters $r$ and $p$, where $r \in \lbrace 1, 2, \dots \rbrace$ and $p
\in [0, 1]$, if its pmf is given by, 
\[ p_Y(k) = P(Y = k) = {{k - 1} \choose {r - 1}} p^r (1 - p)^{k - r}
   \quad k = r, r+1, r+2, \dots                                              \]
\end{definition}
This is a pmf, since, 
\[   \sum_{k = r}^\infty p_Y(k) 
   = \sum_{k = r}^\infty {{k - 1} \choose {r - 1}} p^r (1 - p)^{k - r}  
   = p^r ((1 - (1 - p))^{-r}
   = p^r \cdot p^{-r}
   = 1                                                                       \]

\noindent
\textbf{Interpretation}: Same setup as a geometric distribution:
\begin{enumerate}[noitemsep, topsep=0em]
    \item infinite sequence of independent and identical trials;
    \item each trial has two possible outcomes - success and failure;
    \item probability of success in each trial is $p$, and probability of
          failure in each trial is $(1 - p)$. 
\end{enumerate}
If the $r$-th success occurs in the $y$-th trial, then,
\[ Y \sim \negbinomialdist{r}{p} \]
(When $r = 1$, we get a $\binomialdist{p}$ distribution). As, for $k \geq n$
\[
    P(Y = k) = P(\lbrace \text{success in $k$-th trial} \rbrace \cap
                 \lbrace \text{$r-1$ success in the first $k-1$ trials}\rbrace)
\]
Note that the probability of any sequence of successes and failures of length
$k - 1$ that has $r - 1$ success and $k - r$ failures is $p^{r - 1}(1 - p)^{k -
r}$. Thus,
\begin{align*}
    P(Y = k) &= p \cdot p^{r - 1}(1 - p)^{k - r} \cdot
                (\# \text{sequences of length $k - 1$ that have $r - 1$
                successes and $k - r$ failures})                             \\
             &= p^r \cdot (1 - p)^{k - r} \cdot {{k - 1} \choose {r - 1}}  
\end{align*}
Suppose the first success occurs at the $X_1$-th trial, and then the second
success occurs $X_2$ many trials after that. Define the random  variables,
$X_2, X_3, \dots, X_r$ similarly. 
\begin{figure*}[!htp]
    \centering
    \def\svgwidth{\textwidth}
    \includesvg[./section3/figure/]{sec3-sub7-fig1}
\end{figure*}

\noindent
Note that,
\[ X_1 \sim \geometricdist{p} \quad X_2 \sim \geometricdist{p} \quad
   \dots \quad X_r \sim \geometricdist{p}                                   \]
and
\[ Y = X_1 + \dots + X_r                                                    \]
Thus a $\negbinomialdist{r}{p}$ random variable can be expressed as a sum of
$r$ $\geometricdist{p}$ random variables. 

\begin{theorem}
    If $Y \sim \negbinomialdist{r}{p}$, then,
    \[ E[Y] = \frac{r}{p} \qquad V[Y] = \frac{r(1 - p)}{p^2}                \]
\end{theorem}
\begin{proof}
    If $Y \sim \negbinomialdist{r}{p}$ then,
    \[ E[Y] = E[X_1 + X_2 + \dots + x_r]                                    \]
    where $X_i \sim \geometricdist{p} \quad i = 1, 2, \dots, r$. Hence,
    \[ E[Y] = E[X_1] + E[X_2] + \dots + E[X_r]
            = \frac{r}{p}                                                   \]
    Alternatively,
    \begin{align*}
        E[Y] &= \sum_{k = r}^\infty k \cdot {{k - 1} \choose {r - 1}} p^r
                (1 - p)^{k - r}                                             \\
             &= \sum_{k = r + 1}^\infty (k - r) \cdot {{k - 1} \choose {r - 1}}
                p^r \cdot (1 - p)^{k - r} + r \sum_{k = r}^\infty {{k - 1} 
                \choose {r - 1}} p^r (1 - p)^{k - r}                        \\
             &= p^r (1 - p) (\sum_{k = r + 1}^\infty (k - r){{k - 1} \choose
                {r - 1}} (1 - p)^{k - r - 1}) + r                           \\
             &= p^r (1 - p) (- \frac{\delta}{\delta p} \sum_{k = r}^\infty
                {{k - 1} \choose {r - 1}} \cdot (1 - p)^{k - r}) + r        \\
             &= p^r (1 - p)(- \frac{\delta}{\delta p} (1 - (1 - p))^{-r})
                + r                                                         \\
             &= p^r (1 - p) \cdot \frac{r}{p^{r + 1}} + r                   \\
             &= \frac{r(1 - p)}{p} + r
              = \frac{r}{p}
    \end{align*}
    \[ V[Y] = E[Y^2] - (E[Y])^2 = E[Y^2] - \frac{r^2}{p^2}                  \]
    To compute $E[Y^2]$, one can use a similar technique, $i.e.$ differentiate
the power series twice. The calculation is slightly larger.
\end{proof}

\begin{example}
    A radio station asks a question with four possible choices for its answers
and asks people to call and answer the question. The second caller to answer
correctly will win a special price. Assuming that people just make a random
guess about the answer, find the probability that the fifth caller will win the
prize. 
\end{example}
\begin{solution}
    If the $X$-th caller wins the prize, then $X \sim
\negbinomialdist{2}{\frac{1}{4}}$. Hence, 
    \[ P(X = 5) = {4 \choose 1} (\frac{1}{4})^2 (\frac{3}{4})^3
                = \frac{27}{256}                                            \]
\end{solution}
\subsection{Hypergeometric Distribution}
\begin{definition}
    Suppose $\mathbf{N}, n$ and $r$  are positive integers such that $n \leq
\mathbf{N}$ and $r \leq \mathbf{N}$. Then an random variable $Y$ follows a
hypergeometric distribution with parameters $\mathbf{N}$, $n$ and $r$, or
simply $Y \sim \hypergeometricdist{\mathbf{N}}{n}{r}$, if
\[ p_Y(k) = 
   P(Y = k) = \frac{{r \choose k} {{\mathbf{N} - r} \choose {n - k}}}       
              {{\mathbf{N} \choose n}}                                       \]
where $k$ is an integer such that, $0 \leq k \leq r$ and $0 \leq n - k \leq
\mathbf{N} - r$.
\end{definition}

\noindent
This is indeed a pmf. Clearly, $P(Y = k) \geq 0$. To see why $\sum_{k = 0}^n
P(Y = k) = 1$, note that, 
\[ (1 + x)^\mathbf{N} = (1 + x)^r \cdot (1 + x)^{\mathbf{N} - r}             \]
and the coefficient of $x^n$ on the left side is ${\mathbf{N} \choose n}$,
where as the right side can be expanded as, $(\sum_{j = 0}^r {r \choose j}
\cdot x^j) \cdot (\sum_{s = 0}^{\mathbf{N} - r} {{\mathbf{N} - r} \choose s}
\cdot x^s)$, and hence the coefficient of $x^n$ on the right side is 
$\sum_{k = 0 \lor (n + r - \mathbf{N})}^{r \land n} {r \choose k} \cdot
{{\mathbf{N} - r} \choose {n - k}}$. Hence, 
\[ 
    \sum_{k = 0 \lor (n + r - \mathbf{N})}^{r \land n} 
        {r \choose k} \cdot
        {{\mathbf{N} - r} \choose {n - k}}
    =
    {{\mathbf{N}} \choose n}
\]
which shows that $\sum P(Y = k) = 1$.

\note There is a simple combinatorial argument for proving the above identity
which we will do next.

\noindent
\textbf{Interpretation}: suppose an jar contains $\mathbf{N}$ distinguishable
balls of which $r$ are red and $(\mathbf{N} - r)$ are blue, and you select an
unordered sample of $n$ balls without replacement at random. If $Y$ denotes the
number of red balls in the sample, then $Y \sim
\hypergeometricdist{\mathbf{N}}{n}{r}$.

Total number of possible samples is $\mathbf{N} \choose n$. The number of
sample in which there are exactly $k$ read and $(n - k)$ blue balls is ${r
\choose k} {{\mathbf{N} - r} \choose {n - k}}$. Hence the claim follows.

\begin{theorem}[Binomial approximation to Hypergeometric Distribution]
    If $\mathbf{N} \rightarrow \infty$, $\frac{r_\mathbf{N}}{\mathbf{N}}
\rightarrow p \in (0, \infty)$, $n$ is kept fixed, and $Y_{\mathbf{N}, n,
r_\mathbf{N}} \sim \hypergeometricdist{\mathbf{N}}{n}{r_\mathbf{N}}$, then 
\[
    \lim_{\mathbf{N} \rightarrow \infty} P(Y_{\mathbf{N}, n, r_\mathbf{N}} = k)
  = {n \choose k} \cdot p^k \cdot (1 - p)^{n - k}
\]
for $k = 0, 1, \dots, n$.
\end{theorem}
\note An unordered sample (without replacement) of $n$ balls out of
$\mathbf{N}$ many balls can be drawn by sequentially drawing $n$ balls without
replacement and then ignoring the order in which they were drawn. The theorem
above essentially says that if both $\mathbf{N}$ and $r_\mathbf{N}$ are large,
such that $\frac{r_\mathbf{N}}{\mathbf{N}} \approx p$, then sampling without
replacement is almost equivalent to sampling with replacement which corresponds
to binomial distribution with parameters $n$ and $p$.
\begin{proof}
    First note that the bounds $0 \lor (n + r_\mathbf{N} - \mathbf{N}) \leq k
\leq r_\mathbf{N} \land n$ correspond to $0 \leq k \leq n$ in the case
$\mathbf{N} \rightarrow \infty$ limit. Now for every $0 \leq k \leq n$,
\begin{align*}
    \frac{{r_\mathbf{N} \choose k}{{N - r_\mathbf{N}} \choose {n -k}}}
         {{\mathbf{N} \choose n}}
 &= \frac{r_\mathbf{N} (r_\mathbf{N} - 1) \dots (r_\mathbf{N} - k + 1)}{k!}
    \cdot
    \frac{(\mathbf{N} - r_\mathbf{N}) \dots (\mathbf{N} - r_\mathbf{N} - n
         + k + 1)}{(n-k)!}
    \cdot
    \frac{n!}{\mathbf{N}(\mathbf{N} - 1) \dots (\mathbf{N} - n + 1)}         \\
 &\approx \frac{r_\mathbf{N}^k}{k!} \cdot
          \frac{(\mathbf{N} - r_\mathbf{N})^{n - k}}{(n - k)!} \cdot
          \frac{n!}{\mathbf{N}^n}                                            \\
 &= \frac{r_\mathbf{N}^k \mathbf{N}^{n - k}}{\mathbf{N}^n} \cdot
    \frac{n!}{k! (n - k)!} \cdot
    (1 - \frac{r_\mathbf{N}}{\mathbf{N}})^{n - k}                            \\
 &\approx p^k \cdot {n \choose k} \cdot (1 - p)^{n - k}
\end{align*}
\end{proof}

\begin{theorem}
    If $Y \sim \hypergeometricdist{\mathbf{N}}{n}{r}$, then,
    \[ E[Y] = \frac{n \cdot r}{\mathbf{N}} \qquad
       V[Y] = n \cdot 
              \frac{r}{\mathbf{N}} \cdot 
              \frac{\mathbf{N} - r}{\mathbf{N}} \cdot
              \frac{\mathbf{N} - n}{\mathbf{N} - 1}                          \]
\end{theorem}

%% pending section
%% section3/pending/section3_sub7_problem_on_linearity_of_expectation.tex