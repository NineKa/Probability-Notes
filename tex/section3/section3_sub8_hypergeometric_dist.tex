\subsection{Hypergeometric Distribution}
\begin{definition}
    Suppose $\mathbf{N}, n$ and $r$  are positive integers such that $n \leq
\mathbf{N}$ and $r \leq \mathbf{N}$. Then an random variable $Y$ follows a
hypergeometric distribution with parameters $\mathbf{N}$, $n$ and $r$, or
simply $Y \sim \hypergeometricdist{\mathbf{N}}{n}{r}$, if
\[ p_Y(k) = 
   P(Y = k) = \frac{{r \choose k} {{\mathbf{N} - r} \choose {n - k}}}       
              {{\mathbf{N} \choose n}}                                       \]
where $k$ is an integer such that, $0 \leq k \leq r$ and $0 \leq n - k \leq
\mathbf{N} - r$.
\end{definition}

\noindent
This is indeed a pmf. Clearly, $P(Y = k) \geq 0$. To see why $\sum_{k = 0}^n
P(Y = k) = 1$, note that, 
\[ (1 + x)^\mathbf{N} = (1 + x)^r \cdot (1 + x)^{\mathbf{N} - r}             \]
and the coefficient of $x^n$ on the left side is ${\mathbf{N} \choose n}$,
where as the right side can be expanded as, $(\sum_{j = 0}^r {r \choose j}
\cdot x^j) \cdot (\sum_{s = 0}^{\mathbf{N} - r} {{\mathbf{N} - r} \choose s}
\cdot x^s)$, and hence the coefficient of $x^n$ on the right side is 
$\sum_{k = 0 \lor (n + r - \mathbf{N})}^{r \land n} {r \choose k} \cdot
{{\mathbf{N} - r} \choose {n - k}}$. Hence, 
\[ 
    \sum_{k = 0 \lor (n + r - \mathbf{N})}^{r \land n} 
        {r \choose k} \cdot
        {{\mathbf{N} - r} \choose {n - k}}
    =
    {{\mathbf{N}} \choose n}
\]
which shows that $\sum P(Y = k) = 1$.

\note There is a simple combinatorial argument for proving the above identity
which we will do next.

\noindent
\textbf{Interpretation}: suppose an jar contains $\mathbf{N}$ distinguishable
balls of which $r$ are red and $(\mathbf{N} - r)$ are blue, and you select an
unordered sample of $n$ balls without replacement at random. If $Y$ denotes the
number of red balls in the sample, then $Y \sim
\hypergeometricdist{\mathbf{N}}{n}{r}$.

Total number of possible samples is $\mathbf{N} \choose n$. The number of
sample in which there are exactly $k$ read and $(n - k)$ blue balls is ${r
\choose k} {{\mathbf{N} - r} \choose {n - k}}$. Hence the claim follows.

\begin{theorem}[Binomial approximation to Hypergeometric Distribution]
    If $\mathbf{N} \rightarrow \infty$, $\frac{r_\mathbf{N}}{\mathbf{N}}
\rightarrow p \in (0, \infty)$, $n$ is kept fixed, and $Y_{\mathbf{N}, n,
r_\mathbf{N}} \sim \hypergeometricdist{\mathbf{N}}{n}{r_\mathbf{N}}$, then 
\[
    \lim_{\mathbf{N} \rightarrow \infty} P(Y_{\mathbf{N}, n, r_\mathbf{N}} = k)
  = {n \choose k} \cdot p^k \cdot (1 - p)^{n - k}
\]
for $k = 0, 1, \dots, n$.
\end{theorem}
\note An unordered sample (without replacement) of $n$ balls out of
$\mathbf{N}$ many balls can be drawn by sequentially drawing $n$ balls without
replacement and then ignoring the order in which they were drawn. The theorem
above essentially says that if both $\mathbf{N}$ and $r_\mathbf{N}$ are large,
such that $\frac{r_\mathbf{N}}{\mathbf{N}} \approx p$, then sampling without
replacement is almost equivalent to sampling with replacement which corresponds
to binomial distribution with parameters $n$ and $p$.
\begin{proof}
    First note that the bounds $0 \lor (n + r_\mathbf{N} - \mathbf{N}) \leq k
\leq r_\mathbf{N} \land n$ correspond to $0 \leq k \leq n$ in the case
$\mathbf{N} \rightarrow \infty$ limit. Now for every $0 \leq k \leq n$,
\begin{align*}
    \frac{{r_\mathbf{N} \choose k}{{N - r_\mathbf{N}} \choose {n -k}}}
         {{\mathbf{N} \choose n}}
 &= \frac{r_\mathbf{N} (r_\mathbf{N} - 1) \dots (r_\mathbf{N} - k + 1)}{k!}
    \cdot
    \frac{(\mathbf{N} - r_\mathbf{N}) \dots (\mathbf{N} - r_\mathbf{N} - n
         + k + 1)}{(n-k)!}
    \cdot
    \frac{n!}{\mathbf{N}(\mathbf{N} - 1) \dots (\mathbf{N} - n + 1)}         \\
 &\approx \frac{r_\mathbf{N}^k}{k!} \cdot
          \frac{(\mathbf{N} - r_\mathbf{N})^{n - k}}{(n - k)!} \cdot
          \frac{n!}{\mathbf{N}^n}                                            \\
 &= \frac{r_\mathbf{N}^k \mathbf{N}^{n - k}}{\mathbf{N}^n} \cdot
    \frac{n!}{k! (n - k)!} \cdot
    (1 - \frac{r_\mathbf{N}}{\mathbf{N}})^{n - k}                            \\
 &\approx p^k \cdot {n \choose k} \cdot (1 - p)^{n - k}
\end{align*}
\end{proof}

\begin{theorem}
    If $Y \sim \hypergeometricdist{\mathbf{N}}{n}{r}$, then,
    \[ E[Y] = \frac{n \cdot r}{\mathbf{N}} \qquad
       V[Y] = n \cdot 
              \frac{r}{\mathbf{N}} \cdot 
              \frac{\mathbf{N} - r}{\mathbf{N}} \cdot
              \frac{\mathbf{N} - n}{\mathbf{N} - 1}                          \]
\end{theorem}
\begin{proof}
    If $1 \leq k \leq n$ and $0 \leq n - k \leq \mathbf{N} - r$, then,
    \[   k {r \choose k} {{\mathbf{N} - r} \choose {n - k}}
       = \frac{n!}{(k - 1)!(r - k)!} {{\mathbf{N} - r} \choose {n - k}}      
       = r {{r - 1} \choose {k - 1}} {{\mathbf{N} - r} \choose {n - k}}      
       = r {{r - 1} \choose {k - 1}}
         {{\mathbf{N} - 1 - (r - r)} \choose {n - 1 - (k - 1)}}              \]
    Hence,
    \[   \sum k {r \choose k} {{\mathbf{N} - r} \choose {n - k}}
       = r {{\mathbf{N} - 1} \choose {n - 1}}                                
       \qquad \Rightarrow
         E[Y]
       = \frac{r {{\mathbf{N} - 1} \choose {n - 1}}}{{\mathbf{N} \choose n}} 
       = \frac{r n}{\mathbf{N}}                                              \]
    Similar argument can be applied on $V[Y]$ but with a larger computation.
\end{proof}

\begin{example}[exercise 3.108 of textbook]
    A shipment of $20$ cameras includes $3$ that are defective. What is the
minimum number of cameras that must be selected if we require $P(\text{at least
} 1 \text{ defective}) \geq 0.8$?
\end{example}
\begin{solution}
    If $n$ cameras are selected, then
    \[ f(n) = P(0 \text{ defective}) = \frac{{17 \choose n}}{{20 \choose n}}  
       \qquad \Rightarrow
       f(8) \leq 0.2 \leq f(7)                                               \]
    So minimum sample size is $8$.
\end{solution}
