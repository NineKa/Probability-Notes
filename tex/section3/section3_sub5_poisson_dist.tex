\subsection{Poisson Distribution}
\begin{definition}
    An random variable $X$ is said to follow Poisson distribution with
parameter $\lambda$ ($\lambda > 0$), in short $X \sim \poissondist{\lambda}$,
if the pmf of $X$ is given by, 
\begin{equation*}
    p_X(k) = P(X = k) = e^{-\lambda} \cdot \frac{\lambda^k}{k!}
    \quad k = 0, 1, 2, \dots
\end{equation*}
\end{definition}
A Poisson random variable approximates the total number of successes in a large
number of independent trials each of which has a very small probability of
success. 

\begin{example}
    If probability of hooking a fish at each attempt is $0.04$, then the number
of fish caught after $50$ attempts follows $\binomialdist{50}{0.04} \approx
\poissondist{2}$.
\end{example}

\begin{theorem}
    Let $X_n \sim \binomialdist{n}{p_n}$ where $n \cdot p_n \xrightarrow{n
\rightarrow \infty} \lambda$ for some $\lambda \in (0, \infty)$. Then for each
$k = 0, 1, 2, \dots$,
    \[ P(X_n = k) \rightarrow   e^{-\lambda} \cdot \frac{\lambda^k}{k!}
                              = P(Y = k)                                     \]
    where $Y \sim \poissondist{\lambda}$.
\end{theorem}
\begin{proof}
\begin{align*}
    P(X_n = k) &= {n \choose k} p_n^k (1 - p_n)^{n - k}                      \\
               &= \frac{n(n-1) \cdot (n - k + 1)}{k!} \cdot p_n^k \cdot 
                  (1 - p_n)^n \cdot (1 - p_n)^{-k}                           \\
               &\xrightarrow{n \rightarrow \infty} 
                  \frac{1}{k!} \cdot 1 \cdot \lambda^k \cdot e^{-\lambda} 
                  \cdot 1                                                    \\
               &= e^{-\lambda} \frac{\lambda^k}{k!}  
\end{align*}
Note that $n \cdot p_n \xrightarrow{n \rightarrow \infty} \lambda \in (0,
\infty)$, and $p_n \xrightarrow{n \rightarrow \infty} 0$.
\end{proof}

\begin{theorem}
    If $X \sim \poissondist{\lambda}$, then,
    \begin{equation*}
        E[X] = \lambda \qquad V[X] = \lambda
    \end{equation*}
\end{theorem}
\begin{proof}
    \begin{align*}
        & E[X] = \sum_{k = 0}^\infty k \cdot P(X = k)                         
               = \sum_{k = 1}^\infty k \cdot e^{-\lambda} \cdot
                 \frac{\lambda^k}{k!}                                         
               = e^{-\lambda} \cdot \sum_{k = 1}^\infty
                 \frac{\lambda^k}{(k-1)!}                                     
               = e^{-\lambda} \cdot \lambda \cdot \sum_{k = 1}^\infty
                 \frac{\lambda^{k-1}}{(k-1)!}                                 
               = e^{-\lambda} \cdot \lambda \cdot e^\lambda 
               = \lambda                                                     \\
        & E[X(X-1)] = \sum_{k = 2}^\infty k(k-1) \cdot P(X = k)
                    = \sum_{k = 2}^\infty k(k-1) e^{-\lambda} \cdot
                      \frac{\lambda^k}{k!}
                    = e^{-\lambda} \cdot \sum_{i = 2}^\infty
                      \frac{\lambda^k}{(k-2)!}
                    = \lambda^2 \cdot e^{-\lambda} \cdot e^\lambda
                    = \lambda^2                                              \\
        & E[X^2] = E[X(X - 1)] + E[X] = \lambda^2 + \lambda                  \\
        & V[X] = E[X^2] - (E[X])^2  
               = \lambda^2 + \lambda - \lambda^2
               = \lambda
    \end{align*}
\end{proof}

\begin{example}
    A book has $500$ pages. The probability that a typographical error occurs
in a page is $0.005$. Use Poisson approximation to find the probability that
not more than one page contains a typo.
\end{example}
\begin{solution}
    Let $X$ denote the number of pages that contain a typo. Then $X
\sim \binomialdist{500}{0.005}$. Hence,
    \[ P(X \leq 1) = P(X = 0) + P(X = 1) \approx P(Y = 0) + P(Y = 1) \]
    where, $Y \sim \poissondist{2.5}$ ($500 \cdot 0.005 = 2.5$). Hence,
    \[ 
       P(X \leq 1) = e^{-2.5} \cdot (1 + \frac{2.5}{1!})
                   = 3.5 \cdot e^{-2.5}
    \]
\end{solution}

\begin{example}
    An item is a store has a marked price of $\$100$. For each customer
purchasing the item during a particular day, the store owner reduces the price
of the item by a factor of one-half (i.e. the first customer pays $\$50$, the
second pays $\$25$ etc.). If the number of customers who purchase the item
during the day follows a Poisson distribution with mean $2$, find the expected
price of the item at the end of the day.
\end{example}
\begin{solution}
    If $Y$ denotes the number of customers who purchase that item during the
day, then,
    \[ Y \sim \poissondist{2} (\lambda = 2 \text{ as mean equals two.}) \]
    The cost at the end of the day is $C = 100 \cdot (\frac{1}{2})^Y$. Hence, 
    \[
        E[(\frac{1}{2})^Y] = \sum_{k = 0}^\infty (\frac{1}{2})^k \cdot
                             e^{-2} \cdot \frac{2^k}{k!}
                           = \sum_{k = 0}^\infty e^{-2} \cdot \frac{1}{k!}
                           = e^{-2} \cdot e 
                           = e^{-1}
    \]
    \[
        E[C] = E[100 \cdot (\frac{1}{2})^Y]
             = 100 \cdot E[(\frac{1}{2})^Y]
             = 100 \cdot e^{-1}
    \]
\end{solution}

\begin{example}
    A radioactive piece of rock emits particles at random time intervals. Let
$X_1$ denote the time at which the first particle is emitted. If the number of
particles emitted in a time interval of length $t$ follows a Poisson
distribution with mean $\lambda t$, find $P(X_1 > x)$ for $x > 0$.
    \begin{figure*}[!htp]
        \centering
        \def\svgwidth{\textwidth}
        \includesvg[./section3/figure/]{sec3-sub5-fig1}
    \end{figure*}
\end{example}
\begin{solution}
    Let $N_x \coloneqq \#\text{particles emitted in time interval }[0, x]$.
Then, 
    \[ N_x \sim \poissondist{\lambda x} \] 
    Now, 
    \[P(X_1 > x) = P(N_x = 0) = e^{-\lambda x} \]
\end{solution}