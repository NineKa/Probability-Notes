\documentclass{article}

\usepackage[letterpaper, margin=1in]{geometry}
\usepackage{amsmath}
\usepackage{amssymb}
\usepackage{amsthm}
\usepackage{enumitem}
\usepackage{mathtools}
\usepackage{graphicx}
\usepackage{color}
\usepackage{import}
\usepackage{float}
\usepackage[open]{bookmark}

\theoremstyle{definition}
\newtheorem{theorem}{Theorem}[section]
\newtheorem{definition}[theorem]{Definition}
\newtheorem{example}[theorem]{Example}
\newcommand\note{\noindent \textbf{Note}:}


\newcommand\perm[2][^n]{\prescript{#1\mkern-2.5mu}{}P_{#2}}
\newcommand\comb[2][^n]{\prescript{#1\mkern-0.5mu}{}C_{#2}}

\newcommand\binomialdist[2]{\text{\emph{Bin}}(#1, #2)}
\newcommand\bernoullidist[1]{\text{\emph{Bern}}(#1)}
\newcommand\poissondist[1]{\text{\emph{Poi}}(#1)}
\newcommand\geometricdist[1]{\text{\emph{Geo}}(#1)}
\newcommand\negbinomialdist[2]{\text{$\mathbf{N}$\emph{Bin}}(#1, #2)}
\newcommand\hypergeometricdist[3]{\text{\emph{HG}}(#1, #2, #3)}

\newenvironment{solution}[1][\proofname]{
  \proof[\ttfamily \scshape \large Solution.]
}{\endproof}

\newcommand{\includesvg}[2][]{\subimport{#1}{#2.pdf_tex}}

\begin{document}
\section{Experiments, Sample Space, Events and Probability}
\paragraph{Recall}
\begin{itemize}[noitemsep,topsep=0pt]
    \item
    A set $A$ is countably infinite if $A$ is inifinite and there exists
a one-to-one map from $A$ to $\mathbb{N}$, where $\mathbb{N} = \lbrace 0, 1,
2, 3, \dots \rbrace$, the set of natural numbers. An infinite set that is not
countable is called uncountable. $\mathbb{N}, \mathbb{Z}, \mathbb{Q}$ (set of
rationals) are countably infinite. $\mathbb{R}$ is uncountable. $\lbrace x
\in \mathbb{R} \vert x > 2 \rbrace$ is uncountable. The set of irrational
numbers is uncountable.
    \item
    If $\lbrace a_n \rbrace_{n \geq 1}$ is a sequence of real numbers, then we
say the limit of $\lbrace a_n \rbrace_{n \geq 1}$ exists and equals $a$ (i.e. $
\lim_{n \rightarrow \infty} a_n = a$), if for all $\epsilon > 0, \exists
N_\epsilon$ such that $\forall n \geq N_\epsilon$, $\vert a_n - a \vert
\leq \epsilon$.
\end{itemize}

\begin{definition}
    An experiment is the process lay which an observation is made. The set of
all possible outcomes of an experiment form the sample space for the
experiment. A subset of the sample space is called an event.
\end{definition}

\begin{example}
experiments, sample space, events
\begin{itemize}[noitemsep,topsep=0pt]
    \item
    \textbf{Experiment}:
        measuring the lifetime of a cell in a continuous time unit.          \\
    \textbf{Sample Space}: $S = [0, \infty)$                                 \\
    \textbf{Event}:
        $E = [1000, 2000]$ corresponds to the lifetime being in between $1000$
        and $2000$ units.                                                    \\
    \item
    \textbf{Experiment}:
        rolling a die                                                        \\
    \textbf{Sample Space}: $S = \lbrace 1, 2, 3, 4, 5, 6 \rbrace$            \\
    \textbf{Events}: $E = \lbrace 3 \rbrace$, $E = \lbrace 2, 4, 6 \rbrace$
\end{itemize}
\end{example}

\begin{definition}
    an event that contains only one sample point (i.e. corresponds to a single
outcome) is called a \emph{simple event}. An event that is not simple is called
a \emph{compound} event. A sample space that is finite or countably infinite is
called a \emph{discrete} sample space.
\end{definition}

A probabilistic model for an experiment with a discrete sample space $S =
\lbrace w_1, w_2, w_3, \dots \rbrace$ can be constructed by assigning a
non-negative numbers $p(w_i)$ to each $w_i$, such that $\sum_{w_i
\in S} p(w_i) = 1$. The number $p(w_i)$ is called the probability of the simple
event $\lbrace w_i \rbrace$. Intuitively, the interpretation is that if we
repeat the experiment $n$ times and $n_i$ is the number of times the outcome
turns out to be $w_i$, then $\lim_{n \rightarrow \infty} \frac{n_i}{n} =
p(w_i)$. Probability of an event $E$ is defined as $p(E) = \sum_{w_i
\in E} p(w_i)$.

\begin{example}
probability
\begin{itemize}[noitemsep,topsep=0pt]
    \item
    \textbf{Experiment}: rolling a fair die.                                 \\
    The corresponding probabilities are given by $p(i) = \frac{1}{6}, \forall i
    \in \mathbb{N}, 1 \leq i \leq 6$.                                        \\
    $p(\text{even numbers} = p(\lbrace 2, 4, 6 \rbrace) = \frac{1}{6} +
    \frac{1}{6} + \frac{1}{6} = \frac{1}{2}$.
    \item
    \textbf{Experiment}: tossing a fair coin twice.                          \\
    $S = \lbrace HH, HT, TH, TT \rbrace$, $p(HH) = \frac{1}{4} = p(HT) = p(TH)
    = P(TT)$                                                                 \\
    $p(\text{at least one head}) = p(\lbrace HH, HT, TH \rbrace) = \frac{1}{4}
    + \frac{1}{4} + \frac{1}{4} = \frac{3}{4}$.
\end{itemize}
\end{example}

\begin{definition}
    let $S$ be the sample space associated with an experiment. A probability
$P$ on $S$ assigns to each $A \subseteq S$, a number $P(A)$, called the
probability of the event $A$, such that
\begin{itemize}[noitemsep,topsep=0pt]
    \item $P(A) \geq 0$,
    \item $P(S) = 1$,
    \item if $A_1, A_2, A_3, \dots$ are events such that $A_i \cap A_j =
          \emptyset \quad \forall i \neq j$, then
          \begin{equation*}
              P(\bigcup_{i=1}^{\infty}A_i) = \sum_{i=1}^{\infty} P(A_i)
          \end{equation*}
          In particular,
          \begin{equation*}
              P(A_1 \cup A_2 \cup \dots \cup A_n) = \sum_{i=1}^n P(A_i)
              \text{ whenever }
              A_i \cap A_j = \emptyset \quad \forall i \leq i < j \leq n
          \end{equation*}
\end{itemize}
\end{definition}

\textbf{Note} This definition also makes sense for sample space that are not
discrete. We will work with such examples later in this course.

\begin{theorem}
some basic properties:
\begin{enumerate}[noitemsep,topsep=0pt]
    \item If $A \subseteq B$, then $P(A) \leq P(B)$,
    \item $P(A^\complement) = 1 - P(A)$
    \item $P(A \cup B) = P(A) + P(B) - P(A \cap B)$
    \item $P(\bigcup_{i = 1}^{\infty} A_i) \leq \sum_{i=1}^{\infty} P(A_i)$.
          In particular, $P(\bigcup_{i=1}^{n} A_i) \leq \sum_{i=1}^{n} P(A_i)$.
    \item $P(A) = P(A \cap B) + P(A \cap B^\complement)
                = P(A \cap B) + P(A \setminus B)$
\end{enumerate}
\end{theorem}
\begin{proof}$\quad$                                                         \\
\begin{enumerate}[noitemsep,topsep=0pt]
\item
    $B = A \cup (B \setminus A)$, since $A \subseteq B$. Further, $A$ and
    $B \setminus A$ are disjoint events. Hence, by finite additivity, $P(B) =
    P(A) + P(B \setminus A)$. Since $P(B \setminus A) \geq 0$, $P(A) \leq
    P(B)$.
\item
    $S = A \cup A^\complement$ and $A$ and $A^\complement$ are disjoint. Hence,
    $P(S) = P(A \cup A^\complement) = P(A) + P(A^\complement)$, by finite
    additivity. Since $P(S) = 1$, $P(A^\complement) = 1 - P(A)$.
\item
    $A \cup B = (A \setminus B) \cup (B \setminus A) \cup (A \cap B)$, and
    obviously, $A \setminus B$, $B \setminus A$, $A \cap B$ are mutually
    disjoint. Hence, by finite additivity,
    \begin{equation}
        \label{thm-probability-basic-3-1}
        P(A \cup B) = P(A \setminus B) + P(B \setminus A) + P(A \cap B)
    \end{equation}
    Further, $A = (A \setminus B) \cup (A \cap B)$. Hence,
    \begin{equation}
        \label{thm-probability-basic-3-2}
        P(A) = P(A \setminus B) + P(A \cap B)
        \Rightarrow
        P(A \setminus B) = P(A) - P(A \cap B)
    \end{equation}
    Similarly,
    \begin{equation}
        \label{thm-probability-basic-3-3}
        P(B \setminus A) = P(B) - P(A \cap B)
    \end{equation}
    Using, \ref{thm-probability-basic-3-2} and \ref{thm-probability-basic-3-3}
    in \ref{thm-probability-basic-3-1}, we get,
    \begin{align*}
        P(A \cup B) &= P(A) - P(A \cap B) + P(B) - P(A \cap B) + P(A \cap B) \\
                    &= P(A) + P(B) - P(A \cap B)
    \end{align*}
\item
    Define,
    \begin{align*}
        B_1 &= A_1                                                           \\
        B_2 &= A_2 \setminus A_1                                             \\
        B_3 &= A_3 \setminus (A_1 \cup A_2)                                  \\
            &  \dots                                                         \\
        B_{k+1} &= A_{k+1} \setminus (A_1 \cup A_2 \cup \dots \cup A_k)      \\
            &  \dots
    \end{align*}
    Note that,
    \begin{equation}
        \label{thm-probability-basic-4-1}
        B_i \subseteq A_i \quad \forall i \geq 1
    \end{equation}
    Further, $B_j = A_j \setminus (A1 \cup \dots \cup A_{j-1})$ and $(A_1 \cup
    \dots A_{j-1})$ are disjoint events. Hence, if $i < j$, then $B_j$ and
    $A_1 \cup \dots \cup A_i$ are mutually disjoint as well. Since $B_i
    \subseteq A_i \subseteq (A_1 \cup \dots A_i)$, $B_j$ and $B_i$ are disjoint
    whenever $i < j$. Hence,
    \begin{equation}
        \label{thm-probability-basic-4-2}
        B_1, B_2, B_3, \dots \text{ is a sequence of mutually disjoint events}
    \end{equation}
    Now,
    \begin{equation}
        \label{thm-probability-basic-4-3}
        B_i \subseteq A_i \quad \forall i \geq 1
        \Rightarrow
        \bigcup_{i=1}^\infty B_i \subseteq \bigcup_{i=1}^\infty A_i
    \end{equation}
    Concisely, if $x \in \bigcup_{i=1}^\infty A_i$, then $\exists i_0 \geq 1$
    such that $i_0$ is the smallest index for with $x \in A_{i_0}$, i.e. $x
    \in A_{i_0}$ but $x \notin A_1 \cup \dots \cup A_{i_0 - 1} \Rightarrow x
    \in A_{i_0} \setminus (A_1 \cup \dots \cup A_{i_0 - 1}) = B_{i_0}$. Hence,
    \begin{equation}
        \label{thm-probability-basic-4-4}
        \forall x \in \bigcup_{i = 1}^\infty A_i
        \text{ is an element of } B_{i_0}
        \text{ for some } i_0
        \Rightarrow
        \bigcup_{i=1}^\infty A_i \subseteq \bigcup_{i=1}^\infty B_i
    \end{equation}
    Combing \ref{thm-probability-basic-4-3} and \ref{thm-probability-basic-4-4}
    we get,
    \begin{equation*}
        \bigcup_{i=1}^\infty A_i = \bigcup_{i=1}^\infty B_i
    \end{equation*}
    Since $B_i \quad \forall i \geq 1$ are mutually disjoint,
    \begin{equation*}
        P(\bigcup_{i=1}^\infty A_i) = P(\bigcup_{i=1}^\infty B_i)
                                    = \sum_{i=1}^\infty P(B_i)
                                    \leq \sum_{i=1}^\infty P(A_i)
    \end{equation*}
    where the last step uses the fact that $B_i \subseteq A_i$ and hence
    $P(B_i) \leq P(A_i)$.
\item
    \begin{itemize}[noitemsep,topsep=0pt]
    \item $P(A) = P(A \cap B) + P(A \cap B^\complement)$                     \\
    We want to show that $A \cap B$ and $A \cap B^\complement$ are disjoint.
    Assume the opposite, there exist a $x \in A \cap B$ and $x \in A \cap
    B^\complement$. But we have,
    \begin{equation*}
        \begin{cases}
        x \in A \cap B &\Rightarrow x \in A \land x \in B                    \\
        x \in A \cap B^\complement &\Rightarrow x \in A \land x \in
            B^\complement
        \end{cases} \Rightarrow
        x \in B \land x \in B^\complement
    \end{equation*}
    By contradiction, we show that $A \cap B$ and $A \cap B^\complement$ are
    actually disjoint set. Obviously we have $A = (A \cap B) \cup (A \cap
    B^\complement)$. Thus, by finite additivity, we have $P(A) = P(A \cap B) +
    P(A \cap B^\complement)$.
    \item $P(A) = P(A \cap B) + P(A \setminus B)$                            \\
    Note that $A \setminus B = A \cap (B^\complement)$. Thus, it is equivalent
    to the equality that we showed above.
    \end{itemize}
\end{enumerate}
\end{proof}

\section{Computing Probabilities}
\noindent \textbf{Recall}
For a discrete sample space $S = \lbrace w_1, w_2, w_3, \dots \rbrace$,
probability of any event $E$, $P(E)$ is given by 
\begin{equation*}
    P(E) = \sum_{w_i \in E} P(\lbrace w_i \rbrace)
\end{equation*}
So given an experiment, in order to compute probabilities of events, one can
first compute the probability of each sample point (i.e. $P(\lbrace w_i
\rbrace)$, and then for any event $E$, $P(E)$ will be given by sum of the
probabilities of sample points that are elements of $E$ (i.e. using the
equation above).

\noindent \textbf{Special Case}: If the sample space $S$ is finite and each
outcome is equally likely, then $\forall w \in S$, 
\begin{equation*}
    P(\lbrace w \rbrace) = \frac{1}{\vert S \vert}
\end{equation*}
Hence, 
\begin{equation*}
    P(E) = \frac{\vert E \vert}{\vert S \vert} \quad \forall E \subseteq S
\end{equation*}

\begin{example}
    Products manufactured in a facility are tested for defects. Historical
records indicate that $8\%$ of products have a type A defect, $6\%$ have a type
B defect, and $2\%$ have both type of defects. If one item is chosen at random,
what is the probability that, 
\begin{enumerate}[noitemsep,topsep=0pt]
    \item it has exactly one type of defect?
    \item it is not defective?
\end{enumerate}
\end{example}
\begin{solution} Let $A$ denote the event that the item has a type A
defect, and let $B$ denote the item has a type B defect. Then, $P(A) = 0.08$,
$P(B) = 0.06$ and $P(A \cap B) = 0.02$.
\begin{enumerate}[noitemsep,topsep=0pt]
\item
    The event that the item has exactly one type of defect is represented by, 
    \begin{align*}
        A \triangle B &= (A \setminus B) \cup (B \setminus A)                \\
        P(A \setminus B) &= P(A) - P(A \cap B)                               \\
        P(B \setminus A) &= P(B) - P(A \cap B)
    \end{align*}
    Since $A \setminus B$ and $B \setminus A$ are disjoint, 
    \begin{align*}
        P(A \triangle B) &= P(A \setminus B) + P(B \setminus A)              \\
                         &= P(A) + P(B) - 2 P(A \cap B)                      \\
                         &= 0.08 + 0.06 - 2 \cdot 0.02 = 0.1
    \end{align*}
\item
    The event that the item is not defective is given by $(A \cup
    B)^\complement$. Now,
    \begin{align*}
        P(A \cup B) &= P(A) + P(B) - P(A \cap B)                             \\
                    &= 0.08 + 0.06 - 0.02                                    \\
                    &= 0.12
    \end{align*}
    Hence, 
    \begin{align*}
        P((A \cup B)^\complement) &= 1 - P(A \cup B)                         \\
                                  &= 1 - 0.12                                \\
                                  &= 0.88
    \end{align*}
\end{enumerate}
\end{solution}

\subsection{Counting Techniques}
Suppose you have to complete two tasks. If Task 1 can be completed in $n_1$
ways, and for each of these $n_1$ possibilities, Task 2 can be completed in
$n_2$ ways, then the total number of ways the two tasks can be completed is
$n_1 \cdot n_2$.
\begin{example}
Toss a coin and roll a die simultaneously. The former can have two outcomes,
and for each of these, the latter can have six outcomes. Thus, the total number
of possible outcomes is $2 \times 6 = 12$. 
\end{example}

\noindent \textbf{Recall} for any $n \in \mathbb{N}$, $n! = 1 \times 2 \times
\dots \times n$, and $0! = 1$ by definition. 

\begin{theorem}
    If we are given $n$ distinct objects and $m$ distinct boxes each of which
can hold one object and $n \leq m$, then the number of ways of filling up the
boxes is, 
\begin{equation*}
    \perm[n]{m} = \frac{n!}{(n - m)!}
\end{equation*}
\end{theorem}
\begin{proof}
    The first box can be filled up in $n$ ways. For each of these $n$ ways, the
second box can be filled up in $n - 1$ ways (by using any of the remaining $n
- 1$ objects). Thus the first two boxes can be filled up in $n \cdot (n - 1)$
ways. For each of these $n \cdot (n - 1)$ ways, the third box can be filled up
in $(n - 2)$ ways. Thus, the first $3$ boxes can be filled up in $n \cdot (n -
1) \cdot (n - 2)$ ways. Continuing like this, the $m$ boxes can be filled up
in, 
\begin{equation*}
    n \cdot (n-1) \cdot \dots \cdot (n - m + 1) = \frac{n!}{(n - m)!}
\end{equation*}
\end{proof}

\begin{example}
The papers from $3$ different exams need to be graded. $5$ graders are
available for this job. Each exam requires only one grader. Then the number of
ways of assigning graders to different exams is, 
\begin{equation*}
    \perm[5]{3} = \frac{5!}{(5 - 3)!} = 60
\end{equation*}
\end{example}

\begin{theorem}
    If $r \leq n$, the number of ways of selecting an unordered sample of $r$
objects from a total of $n$ distinct objects is,
\begin{equation*}
    \comb[n]{r} = \frac{n!}{(n - r)!r!}
\end{equation*}
This is also denoted by $n \choose r$.
\end{theorem}
\begin{proof}
    Suppose this can be done in $m$ ways. Now if we had $r$ distinct boxes,
then we could fill them up in two steps:
\begin{enumerate}[noitemsep,topsep=0pt]
\item 
    select an unordered sample of size $r$ from the $n$ distinct objects.
    This can be done in $m$ ways.
\item
    use these $r$ objects to fill up the $r$ boxes by putting one object in
    each box. This can be done in $\perm[r]{r}$ ways.
\end{enumerate}
Thus the total number of ways of filling up those $r$ boxes is $m \cdot
r!$. Hence,
\begin{equation*}
    m \cdot r! = \perm[n]{r} = \frac{n!}{(n - r)!}
    \Rightarrow
    m = \frac{n!}{(n - r)!r!}
\end{equation*}
\end{proof}

\begin{example}
    A committee of $3$ is to be formed from a group of $10$ people. How many
different committees are possible?
\end{example}
\begin{solution}
    Number of different committees is,
    \begin{equation*}
        {10 \choose 3} = \frac{10 \cdot 9 \cdot 8}{3 \cdot 2 \cdot 1} = 120
    \end{equation*}
\end{solution}

\noindent \textbf{Recall} ${n \choose r}$ is the coefficient of $x^r$ in the
expansion of $(1+x)^n$,
\begin{equation*}
    (1+x)^n = \sum_{r = 0}^n {n \choose r} x^r
\end{equation*}
These are called binomial coefficients. 

\begin{theorem}
    The number of ways of partitioning $n$ distinct objects into $k$ distinct
groups containing $n_1, n_2, \dots, n_k$ object respectively where $n_1 +
\dots + n_k = n$ is,
\begin{equation*}
    {n \choose {n_1, n_2, \dots, n_k}} = 
    \frac{n!}{n_1! \cdot n_2! \cdot \cdots \cdot n_k!}
\end{equation*}
\end{theorem}
\begin{proof}
    The first group can be formed in $n \choose n_1$ ways. For each of these
$n \choose n_1$ possibilities, the second group can be formed in ${n - n_1}
\choose n_2$ many ways. Continuing like this, the total number of ways of
partitioning the $n$ objects into $k$ groups is, 
\begin{align*}
    &{n \choose n_1} \cdot {{n - n_1} \choose n_2} \cdot
    {{n - n_1 - n_2} \choose n_3} \cdot \dots \cdot 
    {{n - n_1 - n_2 - \dots - n_{k-1}} \choose n_k}                          \\
    =& \frac{n!}{n_1!(n-n_1)!} \cdot \frac{(n - n_1)!}{n_2! (n - n_1 - n_2)!}
       \cdot \frac{(n - n_1 - n_2)!}{n_3!(n - n_1 - n_2 - n_3)!} \cdot
       \dots \cdot \frac{(n - n_1 - \dots - n_{k-1})!}{n_k! (n - \sum_{i=1}^k
       n_i)!}                                                                \\
    =& \frac{n!}{n_1! \cdot n_2! \cdot \dots \cdot n_k!}
\end{align*}
\end{proof}

\begin{example}
    $4$ people get into an elevator, and each of them can get one of $10$
floors. Assuming each person gets off at a random floor, what is the
probability that no two get off at the same floor?
\end{example}

\begin{solution}
    Let $S$ be the sample space and $E$ denote the event that no two get off at
the same floor. The first person can get off at any of the $10$ floors. For
each of these $10$ possibilities, the second person can get off at any of the
$10$ floors and so on. Hence,
    \begin{equation*}
        \vert S \vert = 10 \times 10 \times 10 \times 10 = 10^4
    \end{equation*}
    Each of these outcomes is equally likely. And, 
    \begin{equation*}
        \vert E \vert = \perm[10]{4} = 10 \times 9 \times 8 \times 7
    \end{equation*}
    Hence required probability is, 
    \begin{equation*}
        P(E) = \frac{\vert E \vert}{\vert S \vert} 
             = \frac{10 \times 9 \times 8 \times 7}{10^4}
             = \frac{63}{125}
    \end{equation*}
\end{solution}

\textbf{Note} This is similar to the birthday problem (Example 2.7 of
textbook).

\begin{example}
    A committee of $5$ is to be selected from a group of $6$ men and $9$ women.
If the selection is made randomly, what is the probability that the committee
consists of $3$ men and $2$ women?
\end{example}
\begin{solution}
    Let $S$ be the sample space and $E$ denote the event that the committee
consists of $3$ men and $2$ women. Then $\vert S \vert = {15 \choose 5}$. Each
of these outcomes is equally likely. And, $\vert E \vert = {6 \choose 3} \cdot
{9 \choose 2}$. Hence, required possibility is, 
\begin{equation*}
    P(E) = \frac{\vert E \vert}{\vert S \vert}
         = \frac{{6 \choose 3} \cdot {9 \choose 2}}{{15 \choose 5}}
         = \frac{240}{1001}
\end{equation*}
\end{solution}

\begin{example}
    A $5$-card poker hand is said to be a full house if it consists of $3$ card
s of the same denomination and $2$ other cards of the same denomination (the
two denominations are necessarily different). If a $5$-card hand is dealt out
randomly, what is the probability that it is a full house?
\end{example}
\begin{solution}
    Obviously, $\vert S \vert = {52 \choose 5}$. Each of these outcomes is
equally likely. In order to get a full house, the two denominations can be
chosen in $13 \cdot 12$ ways, and then the suits can be chosen in ${4 \choose
3} \cdot {4 \choose 2}$ ways. Hence,
\begin{equation*}
    \vert E \vert = 13 \cdot 12 \cdot {4 \choose 3} \cdot {4 \choose 2}
\end{equation*}
And the required probability, 
\begin{equation*}
    P(E) = \frac{\vert E \vert}{\vert S \vert}
         = \frac{13 \cdot 12 \cdot {4 \choose 3} \cdot {4 \choose 2}}{{52
           \choose 5}}
         = \frac{6}{4165}
\end{equation*}
\end{solution}

\begin{example}[Exercise 2.64 of textbook]
    A balanced die is tossed six times, and the numbers on the uppermost face
is recorded each time. What is the probability that the numbers recorded are
$1, 2, 3, 4, 5$ and $6$ in any order?
\end{example}
\begin{solution}
    The first toss can result in any of the $6$ faces. For each of them, the
second toss can result in any of the $6$ faces and so on. Hence, $\vert S
\vert = 6^6$, and each of the outcomes is equally likely. And, $\vert E \vert
= 6!$. Hence, the required probability is, 
\begin{equation*}
    P(E) = \frac{\vert E \vert}{\vert S \vert} 
         = \frac{6!}{6^6} 
         = \frac{5}{324}
\end{equation*}
\end{solution}

\begin{theorem}
    Suppose $k \geq 1$ and $r \geq 0$ are integers, then the number of
nonnegative integer solutions $(x_1, x_2, \dots, x_k)$ to
\begin{equation}
    \label{thm-counting-tech-1-1}
    x_1 + x_2 + \dots + x_k = r
\end{equation}
is ${{r + k - 1} \choose {k - 1}} = {{r + k - 1} \choose {n}}$.
\end{theorem}
\begin{proof}
    If we expand $(1 + x + x^2 + \dots)^k$ into a power series, then each term
will be a monomial of the form $x^{n_1} \cdot x^{n_2} \cdot \dots \cdot
x^{n_k} = x^{n_1 + \dots + n_k}$. Hence the numbers of non-negative integer
solutions to \ref{thm-counting-tech-1-1} above is same as the coefficient of
$x^r$ in the expansion of $(1 + x + x^2 + \dots)^k = (1 - x)^{-k}$, which
equals ${{r + k - 1} \choose {k - 1}}$ (recall the negative binomial series).
\end{proof}

\begin{example}
    If $P(A) = 0.8$, and $P(B) = 0.7$, then
    \begin{enumerate}[noitemsep,topsep=0pt]
        \item What is the largest possible value of $P(A \cap B)$?
        \item What is the smallest possible value of $P(A \cap B)$?
    \end{enumerate}
\end{example}
\begin{solution} \quad                                                       \\
    \begin{enumerate}[noitemsep,topsep=0pt]
        \item 
        $P(A \cap B) \leq P(B)$, (since $A \cap B \subseteq B$) and similarly, 
        $P(A \cap B) \leq P(A)$. Hence $P(A \cap B) \leq \min{\lbrace P(A),
        P(B) \rbrace} = 0.7$.
        \item
        \begin{align*}
                        & P(A \cup B) = P(A) + P(B) - P(A \cap B)            \\
            \Rightarrow & P(A) + P(B) - P(A \cap B) \leq 1                   \\
            \Rightarrow & P(A \cap B) \geq P(A) + P(B) - 1 = 0.5
        \end{align*}
    \end{enumerate}
\end{solution}

\begin{example}
    A deck of $52$ cards is shuffled, and the cards are turned up one at a
time. What is the probability that a card with face value $4$ appears before
any card with face value $5$?
\end{example}
\begin{solution}
    Let $C$ denote the set of cards. The sample space $S$ consists of all $52!$
arrangements of the cards. Further let $E_1$ be the event that a card with face
value $4$ appears before a card with face value $5$, and $E_2$ be the event
that a card with face value $5$ appears before a card with face value $4$.
Then, 
\begin{equation}
    \label{thm-counting-tech-2-1}
    E_1 \cup E_2 = S \text{ and } E_1 \cap E_2 = \emptyset
\end{equation}
Let $\phi : C \rightarrow C$ be the map that takes $4$ of any suit to $5$ of
that suit, and $5$ of any suit to $4$ of that suit, and leaves the reset of
cards fixed. Then $\phi$ is a bijection between $E_1$ and $E_2$. Hence $\vert
E_1 \vert = \vert E_2 \vert \Rightarrow P(E_1) = \frac{\vert E_1 \vert}{\vert
S \vert} = \frac{\vert E_2 \vert}{\vert S \vert} = P(E_2)$. Since,
\begin{align*}
    1 &= P(S)                                                                \\
      &= P(E_1 \cup E_2)                                                     \\
      &= P(E_1) + P(E_1)     &\text{(because of \ref{thm-counting-tech-2-1})}
\end{align*}
We have, $2 \cdot P(E_1) = 1 \Rightarrow P(E_1) = \frac{1}{2}$.
\end{solution}

\begin{example}
    Mary will go to a parity of either Tim, Jim, or Pim goes.
    \begin{align*}
        P(\text{Tim going}) &= 0.15                                          \\
        P(\text{Jim going}) &= 0.10                                          \\
        P(\text{Pim going}) &= 0.05
    \end{align*}
    Would you say Mary will go to the party with probability at least
    $\frac{1}{2}$?
\end{example}
\begin{solution}
    Let $T$ denote the event that Tim goes to the party, $J$ denote the event
that $Jim$ goes to the party, and $P$ denote the event that Pim goes to the
party. Then, 
\begin{align*}
    P(\text{Mary goes to the party}) &= P(T \cup J \cup P)                   \\
                                     &\leq P(T) + P(J) + P(P)                \\
                                     &= 0.30
\end{align*}
The answer is no.
\end{solution}

\begin{example}
    A football team consists of $20$ offensive and $20$ defensive players. THe
players are to be paired in groups of $2$ for the purpose of determining
roommates. If the pairing is done at random, what is the probability that there
are probability that there are no defensive-offensive roommate pairs?
\end{example}
\begin{solution}
    Let $S$ be the set of ways of pairing $40$ players into $20$ unordered
groups of size $2$ each, then 
\begin{equation*}
    \vert S \vert 
    = \frac{{40 \choose {2, 2, \dots, 2}}}{20!}
    = \frac{40!}{2^{20} \cdot 20!}
\end{equation*}
In order for no offensive-defensive pairing to happen, the defensive players
need to be paired among themselves, and the offensive players need to be paired
among themselves. This can be done in,
\begin{equation*}
    \frac{{20 \choose {2, 2, \dots, 2}}}{10!} \cdot \frac{{20 \choose {2, 2,
    \dots, 2}}}{10!}
    = \frac{(20!)^2}{(2^{10} \cdot 10!)^2}
    = \frac{(20!)^2}{2^{20} \cdot (10!)^2}
\end{equation*}
\end{solution}
\begin{solution}
    Rank the players $1$ through $40$. To pair up $40$ players into $20$
groups, the first players can be paired up in $39$ ways. Once this has
happened, the players with the smallest rank among the remaining $38$ players
can be paired up in $37$ ways. In the next step, the player with the smallest
rank among the remaining $36$ players, can be paired up in $35$ ways and so on.
Hence, 
\begin{equation*}
    \vert S \vert = 39 \cdot 37 \cdot 35 \cdot \dots \cdot 3 \cdot 1
\end{equation*}
    By the same argument, players can be paired up so that there are no
offensive-defensive pairs in $(19 \cdot 17 \cdot \dots \cdot 3 \cdot 1)^2$ many
ways. Thus, the required probability is, 
\begin{equation*}
    P(\text{no defensive-offensive pairs}) = 
    \frac{(19 \cdot 17 \cdot \dots \cdot 3 \cdot 1)^2}
         {39 \cdot 37 \cdot 35 \cdot \dots \cdot 3 \cdot 1}
\end{equation*}
\end{solution}

Now suppose the problem had been posed like this:
\begin{example}
    A football team consists of $20$ offensive and $20$ defensive players. A
travel agency has booked $20$ rooms in a hotel for the football team. The
players are to be paired in groups of $2$ and each pair will share one hotel
room. If the players are paired and each pair is assigned a hotel room all at
random, then what is the probability that there are no defensive-offensive
roommate pairs?
\end{example}
\noindent \textbf{Note}: for this problem, because the hotel room, a roommate
pair checks into, is taken into account, the sample space will consist of
elements each of which is an ordered collection of $20$ roommate pairs. So the
sample space would be different. However, the probability of the required event
is still the same. This is because of the following reason:

An ordered group of $20$ roommate pairs can be formed in two steps:
\begin{enumerate}[noitemsep,topsep=0pt]
    \item form an unordered group of $20$ roommate pairs
    \item order them in $20!$ possible ways. 
\end{enumerate}

Thus, when computing the required probability, the factor of $20!$ will cancel
out in the numerators and the denominators, leading to the same answer. But one
can also use a slightly different counting technique to get to the same answer.
\begin{solution}
    Let $S$ denotes the set of ways of pairing $40$ players into $20$ ordered
groups of size $2$ each. Then,
\begin{equation*}
    \vert S \vert = {{40} \choose {2, 2, \dots, 2}} = \frac{40!}{2^{20}}
\end{equation*}

    The offensive players can be paired among themselves into $10$ ordered
groups in $\frac{20!}{2^{10}}$ ways, and same for defensive players. Now we
would like to merge the ordered collection of $10$ offensive-offensive pairs
with the ordered collection of $10$ defensive-defensive pairs, to get one
ordered collection of $20$ roommate pairs. 

    Pictorially this is same as having $10$ numbered sticks and $10$ numbered
balls, and we want to insert the sticks in to $11$ possible places marked in
the picture. Keeping the relative position of the sticks unchanged (i.e., stick
$2$ will be inserted in a place to the right of stick $1$ etc.)
\begin{figure}[H]
    \centering
    \includesvg[./section2/figure/]{sec2-sub1-fig1}
\end{figure}
letting $x_k$ be the number of sticks inserted at mark $k$, we see that the
number of ways done is same as the number of nonnegative integer solutions to
$x_1 + x_2 + \dots + x_{11} = 10$, which equals ${{10 + 11 - 1} \choose {11 -
1}} = {20 \choose 10}$. Hence the required probability is,
\begin{equation*}
    P(\text{no defensive-offensive pairs}) 
    = \frac{\frac{20!}{2^{10}} \cdot
            \frac{20!}{2^{10}} \cdot
            {20 \choose 10}}
           {\frac{40!}{2^{20}}}
    = \frac{(20!)^3}{(10!)^2 \cdot 40!}
\end{equation*}

    Same as before.
\end{solution}

%% appending notes from 01-19-2017 %%
\begin{example}
    A hand of $5$ cards is dealt at random from a deck of $52$ cards. What is
the probability that the hand is a two-pair? (Example of two-pair: 5 of hearts,
5 of spades, 7 of hearts, 7 of clubs, and king of diamonds)
\end{example}
\begin{solution}
    \textbf{Experiment}: selecting a hand of $5$ cards at random from a deck of
$52$ cards. Let $C$ be set of $52$ cards. Then $S = \lbrace \lbrace c_1, c_2,
c_3, c_4, c_5 \rbrace \vert c_i \in C \quad \forall 1 \leq i \leq 5 \land c_i
\neq c_j \quad \forall 1 \leq i < j \leq 5 \rbrace$, and $\vert S \vert = {52
\choose 5}$.

    For any $c \in C$, let $f(c)$ denote its face value. Then the event we are
interested in is, 
\begin{align*}
    E = \lbrace \lbrace c_1, \dots, c_5 \rbrace \in S \vert
        & \exists \text{ a permutation } i_1, i_2, \dots, i_5 \text{ of } 
        1, 2, \dots, 5 \text{ s.t. }                                         \\
        & f(c_{i_1}) = f(c_{i_2}) \land
        f(c_{i_3}) = f(c_{i_4}) \land
        f(c_{i_1}), f(c_{i_3}), f(c_{i_5}) \text{ are distinct }
        \rbrace
\end{align*}
Then, 
\begin{align*}
    \vert E \vert &= {13 \choose 3} \cdot {3 \choose 2} \cdot {4 \choose 2}
                     \cdot {4 \choose 2} \cdot {4 \choose 1}                 \\
                  &= \frac{13 \cdot 12 \cdot 11}{2} \cdot 36 \cdot 4
\end{align*}
Thus, the required probability is, 
\begin{equation*}
    P(\text{the hand is a two-pair}) = \frac{13 \cdot 12 \cdot 11 \cdot
                                       72}{{52 \choose 5}}
\end{equation*}
\end{solution}
\subsection{Conditional Probability}
The knowledge that an event has occurred affects the odds of occurrence of a
correlated event. This is formalized in the notion of conditional probability.

\begin{example}
    Roll a fair die. Given the information that the outcome is even, what
probability should be assigned to the event $\lbrace 6 \rbrace$?
\end{example}
\begin{solution}
    As $S^\prime = \lbrace 2, 4, 6 \rbrace$. The conditional probability of the
event $\lbrace 6 \rbrace$ should be $\frac{1}{3}$. (We are using the fact that
the die is fair).
\end{solution}

\begin{definition}
    If $A$ and $B$ are events and $P(B) > 0$, then the "conditional probability
of $A$ given $B$" is defined as, 
\begin{equation*}
    P(A \vert B) = \frac{P(A \cap B)}{P(B)}
\end{equation*}
\end{definition}

\begin{definition}
    Two events $A$ and $B$ are independent if $P(A \cap B) = P(A) \cdot P(B)$.
In general $A_1, A_2, \dots, A_k$ are independent events if for any $1 \leq
i_1 < i_2 < \dots < i_r \leq k$,
\begin{equation*}
    P(A_{i_1} \cap A_{i_2} \cap \dots \cap A_{i_r}) = 
    P(A_{i_1} \cdot A_{i_2} \cdot \dots \cdot P(A_{i_r})
\end{equation*}
\end{definition}

\noindent \textbf{Note}: If $A$ and $B$ are independent and $P(B) > 0$, then 
\begin{equation*}
    P(A \vert B) = \frac{P(A \cap B)}{P(B)}
                 = \frac{P(A) \cdot P(B)}{P(B)}
                 = P(A)
\end{equation*}
In other words, if $A$ and $B$ are independent, then knowledge about occurrence
of $B$ does not affect the chances of occurrence of $A$.

\begin{example}
    In a biased die, the probability of face $i$ is proportional to $i$, $1
\leq i \leq 6$. Given the information that a number less than or equal to $4$
came up when the die was rolled, what is the probability that the number was
less than or equal to $2$?
\end{example}
\begin{solution}
    As, $P(\lbrace i \rbrace) = c \cdot i, \forall 1 \leq i \leq 6$. Let $A =
\lbrace 1, 2, 3, 4 \rbrace$ and $B = \lbrace 1, 2 \rbrace$. Then, 
\begin{equation*}
    P(B \vert A) = \frac{P(A \cap B)}{P(A)}
                 = \frac{P(B)}{P(A)}
                 = \frac{c \cdot (1 + 2)}{c \cdot (1 + 2 + 3 + 4)}
                 = \frac{3}{10}
\end{equation*}
\end{solution}
\subsection{Multiplicative Law of Probability}
If $P(B) > 0$, then $P(A \cap B) = P(B) \cdot P(A \vert B)$. Similarly, if
$P(A) > 0$, then $P(A \cap B) = P(A) \cdot P(B \vert A)$. Using this
repeatedly,
\begin{align*}
    P(A_1 \cap A_2 \cap \dots \cap A_k) 
    &= P(A_k \vert A_1 \cap A_2 \cap \dots \cap A_{k-1}) \cdot
       P(A_1 \cap \dots \cap A_{k-1})                                        \\
    &= P(A_k \vert A_1 \cap A_2 \cap \dots \cap A_{k-1}) \cdot
       P(A_{k-1} \vert A_1 \cap A_2 \cap \dots \cap A_{k-2}) \cdot
       P(A_1 \cap \dots \cap A_{k-2})                                        \\
    &\dots                                                                   \\
    &= P(A_k \vert A_1 \cap A_2 \cap \dots \cap A_{k-1}) \cdot
       P(A_{k-1} \vert A_1 \cap A_2 \cap \dots \cap A_{k-2}) \cdot
       \dots \cdot
       P(A_2 \vert A_1) \cdot
       P(A_1)
\end{align*}
provided with $P(A_1 \cap \dots \cap A_{k-1}) > 0$.

\begin{example}
    A student estimates his probability of receiving an A grade to be
$\frac{1}{2}$ in a French elective course and $\frac{2}{3}$ in a chemistry
course. He has to take exactly one of these courses. If he decides to make his 
decision on the flip of a fair coin, what is the probability that he takes
chemistry and gets an A?
\end{example}
\begin{solution}
    Let $C$ be the events that the student takes chemistry course, and $A$ be
the events that the student is getting an A. Then, 
\begin{equation*}
    P(A \cap C) = P(C) \cdot P(A \vert C)
                = \frac{1}{2} \cdot \frac{2}{3}
                = \frac{1}{3}
\end{equation*}
\end{solution}

\begin{example}
    If $P(B) > 0$, show that $P(A^\complement \vert B) = 1 - P(A \vert B)$.
\end{example}
\begin{solution}
    \begin{equation*}
        P(A^\complement \vert B) 
        = \frac{P(A^\complement \cap B)}{P(B)}                              
        = \frac{P(B) - P(A \cap B)}{P(B)}                                   
        = 1 - \frac{P(A \cap B)}{P(B)}                                      
        = 1 - P(A \vert B)
    \end{equation*}
\end{solution}
\subsection{Law of Total Probability}
\begin{theorem}
    If $\lbrace B_1, B_2, \dots \rbrace$ (finite or countably infinite) is a
partition of a sample space $S$ with $P(B_i) > 0$, then for any event $A$,
\begin{equation*}
    P(A) = \sum_{i \geq 1} P(A \vert B_i) \cdot P(B_i)
\end{equation*}
\end{theorem}
\begin{proof}
    Since $\bigcup_{i \geq 1} B_i = S$, then 
    \begin{equation*}
        A = A \cap (\bigcup_{i \geq 1} B_i)
          = \bigcup_{i \geq 1} (A \cap B_1)
    \end{equation*}
    Since, $B_i \cap B_j = \emptyset \quad \forall i \neq j$, $(A \cap B_i)
\cap (A \cap B_j) = \emptyset$. Hence,
    \begin{equation*}
        P(A) = P(\bigcup_{i \geq 1} (A \cap B_i))
             = \sum_{i \geq 1} P(A \cap B_i)
             = \sum_{i \geq 1} P(A \vert B_i) \cdot P(B_i)
    \end{equation*}
\end{proof}

\begin{example}
    Mary is given two jars. The first jar contains $5$ red balls and $3$ blue
balls. The second jar contains $4$ red balls and $3$ blue balls. Mary chooses
jar $1$ with probability $\frac{1}{3}$ and jar $2$ with probability
$\frac{2}{3}$, and then selects one ball at random from the chosen jar. What is
the probability that the selected ball is red?
\end{example}
\begin{solution}
    Let $U_1$ denotes the event that Mary choose jar $1$, $U_2$ denotes the
event that Mary choose jar $2$, and $R$ denotes the event that the chosen ball
is actually red.
\begin{equation*}
    P(R) = P(R \vert U_1) \cdot P(U_1) + P(R \vert U_2) \cdot P(U_2)        
         = \frac{5}{8} \cdot \frac{1}{3} + \frac{6}{7} \cdot \frac{2}{3}    
         = \frac{5}{24} + \frac{8}{21}
         = \frac{33}{56}
\end{equation*}
\end{solution}

\begin{theorem}[Bayes' rule]
    If $\lbrace B_1, B_2, \dots \rbrace$ is a partition of a sample space $S$
and $P(B_i) > 0$, then for any event $A$ with $P(A) > 0$,
\begin{equation*}
    P(B_i \vert A) = \frac{P(A \vert B_i) \cdot P(B_i)}
                          {\sum_{k \geq 1} P(A \vert B_k) \cdot P(B_k)}
\end{equation*}
\end{theorem}
\begin{proof}
    \begin{equation*}
          P(B_i \vert A) = \frac{P(A \cap B_i)}{P(A)}
        = \frac{P(A \cap B_i) P(B_i)}{\sum_{k \geq 1} P(A \vert B_k) P(B_k)}
    \end{equation*}
\end{proof}

\begin{example}
    In the setup of the previous problem, if the selected ball turns out be
red, then what is the probability that jar $1$ was selected?
\end{example}
\begin{solution}
    \begin{equation*}
        P(U_1 \vert R) = \frac{P(R \vert U_1) \cdot P(U_1)}{P(R)}
                       = \frac{\frac{5}{8} \cdot \frac{1}{3}}
                              {\frac{33}{56}}
                       = \frac{35}{99}
    \end{equation*}
\end{solution}
\subsection{Independent Experiments}
\begin{definition}
    Two experiments $E_1$ and $E_2$ conducted jointly are said to have been
conducted independently if the following holds. If $A_1$ is any event whose
occurrence depends only on the outcome of $E_1$ and $A_2$ is any event whose
occurrence depends only of $E_2$, then $A_1$ and $A_2$ are independent.
\end{definition}
Can be generalized to countably many experiments.

\begin{example}
    Toss a fair coin three times independently. $P(\text{three tails}) =
(\frac{1}{2})^3 = \frac{1}{8}$.
\end{example}

\begin{example}
    Toss a fair coin and roll a fair die independently. What is
$P(\lbrace \text{tails} \rbrace \cap \lbrace \text{even numbers from
die} \rbrace)$?
\end{example}
\begin{solution}
    \begin{equation*}
        P(\lbrace \text{tails} \rbrace \cap \lbrace \text{even numbers from
          die} \rbrace)
        = P(\lbrace \text{tails} \rbrace) \cdot
          P(\lbrace \text{even numbers from die} \rbrace)
        = \frac{1}{2} \cdot \frac{1}{2}
        = \frac{1}{4}
    \end{equation*}
\end{solution}

\begin{example}[exercise 2.119 of textbook]
    Two fair dice are tossed repeatedly and the sum of the two uppermost faces
is determined on each toss. What is the probability that we obtain a sum of $3$
before obtaining a sum of $7$?
\end{example}
\begin{solution}
    The experiment is rolling two dice independently countably infinitely many
times, and notice the sum of the two uppermost faces. Let $E_{3, i}$ be the
event that we get a sum of $3$ in the $i$-th toss. Let $E_{7, i}$ be the event
that we get a sum of $7$ in the $i$-th toss. Let $F_i$ be the event that we get
a sum that is not $3$ nor $7$ in the $i$-th toss. Then, 
\begin{align*}
    P(E_{3, i}) &= \frac{2}{36} = \frac{1}{18} \quad \forall i \geq 1        \\
    P(E_{7, i}) &= \frac{6}{36} = \frac{1}{6}  \quad \forall i \geq 1        \\
    P(F_i)     &= 1 - P(F_{3, i}) - P(F_{7, i}) = \frac{7}{9}
\end{align*}
\begin{align*}
       P(\text{sum of $3$ before sum of $7$})
    &= P(E_{3, 1}) + P(F_1 \cap E_{3, 2}) + P(F_1 \cap F_2 \cap E_{3, 3}) +
       \dots                                                                 \\
    &= P(E_{3, 1}) + P(F_1) \cdot P(E_{3, 2}) + P(F_1) \cdot P(F_2) \cdot
       P(E_{3, 3}) + \dots                                                   \\
    &= \sum_{i = 0}^\infty (\frac{7}{9})^i \cdot \frac{1}{18}                \\
    &= \frac{1}{18} \cdot \frac{1}{1 - \frac{7}{9}}
     = \frac{1}{18} \cdot \frac{9}{2}
     = \frac{1}{4}
\end{align*}
\end{solution}

\begin{example}[exercise 2.154(b) of textbook]
    A drawer contains $n$ different pairs of socks. The two socks making up a
matching pair are distinguishable. A person selects $2r$ many socks at random,
where $2r < n$. What is the probability that there is no matching pair in the
sample?
\end{example}
\begin{solution}
    As, $\vert S \vert = {{2n} \choose {2r}}$, and $\vert E \vert = {{n}
\choose {2r}} \cdot 2^{2r}$ (choose $2r$ pairs and then select one of the two
socks in each pair). Thus, the required probability is,
\begin{equation*}
       P(\text{no matching pair})
    = \frac{{{n} \choose {2r}} \cdot 2^{2r}}{{{2n} \choose {2r}}}
\end{equation*}
\end{solution}

\begin{example}[exercise 2.181 of textbook]
    Suppose that $n$ indistinguishable balls are to be arranged in $N$
distinguishable boxes. If $n \geq N$ and all arrangements are equally likely,
then show that the probability of no box being empty is given by
$\frac{{{n-1} \choose {N - 1}}}{{{N + n -1} \choose {N - 1}}}$
\end{example}
\begin{solution}
    Since the ball are indistinguishable. The sample space $S$ is given by,
    \begin{equation*}
        S = \lbrace (x_1, x_2, \dots, x_N) \vert x_i \text{ are nonnegative
                    integers such that } \sum_{i = 1}^N x_i = n \rbrace
    \end{equation*}
    Here $x_i$ represents the number of balls in box $i$. Hence, $\vert S
\vert = {{N + n - 1} \choose {N - 1}}$. If $E$ is the event that no box is
empty, then $\vert E \vert$ equals the number of positive integer valued
solutions to,
    \begin{equation}
        \label{thm-counting-tech-5-1}
        x_1 + x_2 + \dots + x_N = n
    \end{equation}
    Let $y_i = x_i - 1$. Then $y_i \geq 0$. The number of positive
integer-valued solutions to \ref{thm-counting-tech-5-1} above is same as the
number of nonnegative integer-valued solution to $y_1 + y_2 + \dots + y_N = n
- N$. Hence, $\vert E \vert = {{n -N + N - 1} \choose {N - 1}} = {{n - 1}
\choose {N - 1}}$. Thus the required probability is, 
    \begin{equation*}
          P(\text{no ox being empty}) 
        = \frac{\vert E \vert}{\vert S \vert}
        = \frac{{{n - 1}\choose {N - 1}}}{{{N + n - 1} \choose {N - 1}}}
    \end{equation*}
\end{solution}
\section{Discrete Random Variables (RVs)}
\begin{definition}
    A random variable is a function defined on some sample space that takes
values in $\mathbb{R}^d$, i.e., if $S$ is a sample space and $X : S
\rightarrow \mathbb{R}^d$, then $X$ is an $\mathbb{R}^d$-valued random
variable. If $d \geq 2$, these are also called random vectors.
\end{definition}

\begin{example} \quad                                                        \\
\begin{enumerate}[noitemsep, topsep=0em]
    \item \label{sec3-example-1}
          \textbf{Experiment}: roll a fair die                               \\
          \textbf{Sample Space}: $S = \lbrace 1, 2, \dots, 6 \rbrace$        \\
          Then $X_1 : S \rightarrow \mathbb{R}^2$ defined as
          \begin{equation*}
              X_1(i) = (i^2, 3 \cdot i), i \in \lbrace 1, 2, \dots, 6 \rbrace
          \end{equation*}
          is a random variable.
    \item \label{sec3-example-2}
          \textbf{Experiment}: select a person from a city and measure his/her
          height                                                             \\
          \textbf{Sample Space}: $S = (0, \infty)$. 
          Then $X_1 : S \rightarrow \mathbb{R}$ defined as
          \begin{equation*}
              X_2(u) = u, u \in S
          \end{equation*}
          is a random variable defined on $S$.
\end{enumerate}
\end{example}

\begin{definition}
    An random variable $X$ is discrete if it can assume only a finite or
countably infinite number of distinct values.
\end{definition}

\begin{example}
    In (\ref{sec3-example-1}) above, $X_1$ is a discrete, where in
(\ref{sec3-example-2}) above, $X_2$ is not.
\end{example}

If $X$ is an $\mathbb{R}^d$-valued random variable defined on a sample space
$S$, then for any $A \subseteq \mathbb{R}^d$, $X^{-1}(A) = \lbrace w \in S
\vert X(w) \in A \rbrace$ is an event. The event $X^{-1}(A)$ is written simply
as $\lbrace X \in A \rbrace$. If $O$ is a probability on $S$, then we write
$P(X \in A)$ to mean $P(\lbrace w \in S \vert X(w) \in A \rbrace)$.

\begin{definition}
    If $X$ is a discrete random variable defined on a sample space $S$ with
associated probability $P$ and the set of all possible values of $X$ are $x_1,
x_2, x_3, \dots$, then the probability distribution of $X$ is represented by
the collection $p(x_i) = P(X = x_i), i = 1, 2, 3, \dots$
\end{definition}

To compute the probability distribution of a discrete random variable $X$, one
can follow these steps:
\begin{itemize}[noitemsep, topsep=0em]
    \item List all possible values of $X$ as $x_1, x_2, x_3, \dots$
    \item For each $i$, compute $P(X = x_i)$. Then the distribution of $X$ will
          be given by $p(x_i) = P(X = x_i), i = 1, 2, 3, \dots$ 
\end{itemize}

\begin{example}
    Toss a fair coin twice, let $X = \text{ number of tails }$. Find the
distribution of $X$.
\end{example}
\begin{solution}
    The possible values of $X$ are $0$, $1$ and $2$. Thus the distribution of
$X$ is given by, 
\begin{align*}
    p(0) &= P(X = 0) = P(HH) = \frac{1}{4}                                   \\
    p(1) &= P(X = 1) = P(HT) + P(TH) = \frac{1}{4}+\frac{1}{4} = \frac{1}{2} \\
    p(2) &= P(X = 2) = P(TT) = \frac{1}{4}
\end{align*}
\end{solution}

\begin{example}
    Roll a fair die until $6$ appears. Let $X$ equal the number of required
tosses. Find the distribution $X$.
\end{example}
\begin{solution}
    Possible values of $X$ are $1, 2, 3, \dots$, the distribution of $X$ is
given by, 
\begin{align*}
    p(i) &= P(X = i)                                                         \\
         &= P(\bigcap_{k=1}^{i-1} \lbrace \text{not $6$ in $k$-th toss} \rbrace
            \cap \lbrace \text{$6$ in $i$-th toss} \rbrace)                  \\
         &= \prod_{k=1}^{i-1} P(\text{not $6$ in $k$-th toss}) \cdot
            P(\text{$6$ in $i$-th toss})                                     \\
         &= (\frac{5}{6})^{i-1} \cdot \frac{1}{6}          & i = 1, 2, 3, \dots
\end{align*}
\end{solution}

\begin{example}
    Let $S$ be a sample space and $E$ be an event. Define the random variable
$1_E$ as, 
\begin{equation*}
    1_E(w) = \begin{cases}
        0 & w \notin E                                                      \\
        1 & w \in E 
    \end{cases}
\end{equation*}
find the distribution of $1_E$
\end{example}
\begin{solution}
    Possible values of $1_E$ are $0$ and $1$. Thus the distribution of $1_E$ is
given by, 
\begin{align*}
    p(0) &= P(1_E = 0)
          = P(\lbrace w \vert 1_E(w) = 0\rbrace)
          = P(E^\complement) 
          = 1 - P(E)                                                         \\
    p(1) &= P(1_E = 1) = P(E)
\end{align*}
\end{solution}

\begin{example}
    Toss a fair coin and roll a fair die independently. Let,
    \begin{equation*}
        X_1 = \begin{cases}
            0 & \text{toss results in tails}                                 \\
            1 & \text{toss results in heads}
        \end{cases}
    \end{equation*}
    and, 
    \begin{equation*}
        X_2 = \text{the number on the uppermost face of the die}
    \end{equation*}.
    Let $X = (X_1, X_2)$ (so $X$ is a $\mathbb{R}^2$-valued random variable).
What is the distribution of $X$?
\end{example}
\begin{solution}
    possible values of $X$ are $\langle i, j\rangle$, where $i \in \lbrace 0, 1
\rbrace$ and $j \in \lbrace 1, 2, \dots, 6 \rbrace$, and for any such $i, j$,
\begin{equation*}
    p(\langle i,j \rangle) = P(X = \langle i,j \rangle)
                           = \frac{1}{2} \cdot \frac{1}{6}
                           = \frac{1}{12}
\end{equation*}
This specifies the distribution of $X$.
\end{solution}

\begin{theorem}
    If $X$ is a discrete random variable with possible values $x_1, x_2,
\dots$ and $p(x_i), i = 1, 2, \dots$ is the probability distribution of $X$,
then,
\begin{enumerate}[noitemsep, topsep=0em]
    \item \label{sec3-thm-1}
          $0 \leq p(x_i) \leq 1$, and,
    \item \label{sec3-thm-2}
          $\sum_{i = 1, 2, \dots} p(x_i) = 1$
\end{enumerate}
\end{theorem}
\begin{proof}
    As $p(x_i) = P(X = x_i)$ by definition. This immediately proves
(\ref{sec3-thm-1}). Next, since the events $\lbrace X = x_i \rbrace, i = 1, 2,
\dots$ are mutually disjoint,
\begin{equation*}
      \sum_{i = 1, 2, \dots} p(x_i) = \sum_{i = 1, 2, \dots} P(X = x_i)
    = P(X \in \lbrace x_1, x_2, \dots \rbrace)
\end{equation*}
    Now, $\lbrace x_1, x_2, \dots \rbrace$ is the set of all possible values of
$X$. Hence the event $\lbrace w \in S \vert X(w) \in \lbrace x_1, x_2, \dots
\rbrace \rbrace$ is the entire sample space $S$. Therefore $\sum_{i = 1, 2,
\dots} p(x_i) = P(S) = 1$, as claimed in part (\ref{sec3-thm-2}).
\end{proof}

\begin{example}
    A single cell can either die, with probability $0.1$, or split into two
cells, with probability $0.9$ producing a new generation of cells. Each cel in
the new generation dies or splits independently with the same probabilities. If
you start with one cell (generation zero), and let it split/die to produce
cells in generation one, find the distribution of the number of cells in
generation two. 
\end{example}
\begin{solution}
    Let $X_1 = \text{number of cells in generation one}$, and $X_2 =
\text{number of cells in generation two}$. Then, 
\begin{align*}
    P(X_1 = 0) &= 0.1                                                        \\
    P(X_2 = 2) &= 0.9
\end{align*}
The possible values $X_2$ can take one from $0$, $2$ and $4$.
\begin{align*}
    P(X_2 = 0 \vert X_1 = 0) &= 1                                            \\
    P(X_2 = 2 \vert X_1 = 0) &= 0                                            \\
    P(X_2 = 4 \vert X_1 = 0) &= 0                                            \\
    P(X_2 = 0 \vert X_1 = 2) &= (0.1)^2                                      \\
    P(X_2 = 2 \vert X_1 = 2) &= 0.1 \cdot 0.9 + 0.1 \cdot 0.9
                              = 2 \cdot 0.1 \cdot 0.9                        \\
    P(X_2 = 4 \vert X_1 = 2) &= (0.9)^2
\end{align*}
Hence the distribution of $X_2$ is given by,
\begin{align*}
    P(X_2 = 0) &= P(X_2 = 0 \vert X_1 = 0) \cdot P(X_1 = 0) +
                  P(X_2 = 0 \vert x_1 = 2) \cdot P(X_1 = 2)
                = 1 \cdot 0.1 + (0.1)^2 \cdot 0.9
                = 0.109                                                     \\
    P(X_2 = 2) &= P(X_2 = 2 \vert X_1 = 0) \cdot P(X_1 = 0) +
                  P(X_2 = 2 \vert X_1 = 2) \cdot P(X_1 = 2)
                = 0 + 2 \cdot 0.1 \cdot 0.9^2
                = 0.162                                                     \\
    P(X_2 = 4) &= P(X_2 = 4 \vert X_1 = 0) \cdot P(X_1 = 0) +
                  P(X_2 = 4 \vert X_1 = 2) \cdot P(X_1 \ 2)
                = 0 + 0.9*3
                = 0.729
\end{align*}
\end{solution}

\subsection{Functions of Random Variable}
If $X : S \rightarrow \mathbb{R}^m$ is an random variable and $g : \mathbb{R}^m
\rightarrow \mathbb{R}^n$ is a function, then $g(x)$ is an
$\mathbb{R}^n$-valued random variable. If the possible values of $X$ are $x_1,
x_2, \dots$ and $p_X(x_i), i = 1, 2, \dots$ is the distribution of $X$, then
the distribution of $Y = g(x)$ can be computed as follows:
\begin{enumerate}[noitemsep, topsep=0em]
    \item list all possible values $y_1, y_2, \dots$ of $Y$, 
    \item for each $j$, find the set 
          \[ E_j = \lbrace x \vert x \in \lbrace x_1, x_2, \dots \rbrace, 
                                   g(x) = y_i \rbrace,                       \]
    \item the distribution of $Y$ is given by,
          \begin{equation*}
              p_Y(y_j) = \sum_{x \in E_j} p_X(x), \quad j = 1, 2, 3, \dots
          \end{equation*}
\end{enumerate}

\begin{example}
    The distribution of $X$ is given by,
    \begin{align*}
        &p_X(-3) = 1 / 16                    &p_X(-2) = 2 / 16              \\
        &p_X(-1) = 1 / 16                    &p_X(0)  = 3 / 16              \\
        &p_X(1)  = 4 / 16                    &p_X(2)  = 3 / 16              \\
        &p_X(4)  = 2 / 16                    &\quad
    \end{align*}
\end{example}
\begin{solution}
    The possible values of $Y$ are $0, 1, 4, 9, 16$,
    \begin{align*}
        p_Y(0) &= p_X(X = 0) = 3 / 16                                      \\
        p_Y(1) &= p_X(X = -1) + p_X(X = 1) = 1 / 16 + 4 / 16 = 5 / 16      \\
        p_Y(4) &= p_X(X = -2) + p_X(X = 2) = 2 / 16 + 3 / 16 = 5 / 16      \\
        p_Y(9) &= p_X(X = -3) = 1 / 16                                     \\
        p_Y(16) &= p_X(X = 4) = 2 / 16
    \end{align*}
\end{solution}
\subsection{Expectation of a Discrete Random Variable}
\begin{definition}
        If $X$ is a discrete random variable with probability distribution
    $p(x_i), i = 1, 2, \dots$, then the expectation of $X$ is given by,
    \begin{equation*}
        E[X] = \sum_{i = 1, 2, \dots} x_i \cdot p(x_i)
    \end{equation*}
        If $X$ is $\mathbb{R}^d$-valued, then $E[X]$ is a number in
    $\mathbb{R}^d$.
\end{definition}

\noindent
\textbf{Note}: If $X$ is $\mathbb{R}$-valued and $x_1^+, x_2^+, \dots$ are the
positive values assumed by, and $x_1^-, x_x^-, \dots$ are the negative values
assumed by $X$, then $E[X]$ is only defined when either $\sum_{i = 1, 2,
\dots} x_i^+ \cdot p(x_i^+) < \infty$ or $\sum_{i=1, 2, \dots} x_i^- \cdot
p(x_i^-) < \infty$. If both of these quantities are infinite, then we say
expectation of $X$ does not exist. Similar consideration apply to
$\mathbb{R}^d$-valued random variables.

\begin{example}
    Roll a fair die until $6$ appears and let $X$ be the number of required
tosses. We have already computed the distribution of $X$ as $p(i) =
(\frac{5}{6})^{i-1} \cdot \frac{1}{6}$, $i = 1, 2, \dots$. Hence, 
\begin{equation*}
    E[X] = \sum_{i \geq 1} i \cdot (\frac{5}{6})^{i-1} \cdot \frac{1}{6}    
         = \frac{1}{6} \cdot (1 - \frac{5}{6})^{-2}                         
         = \frac{1}{6} \cdot 36
         = 6
\end{equation*}
\end{example}

\begin{example}
    If $E$ is an event and $1_E$ is the random variable defined by,
    \begin{equation*}
        1_E(w) = \begin{cases}
            1 & w \in E                                                      \\
            0 & w \notin E
        \end{cases}
    \end{equation*}
    then, 
    \begin{equation*}
        E[1_E] = 1 \cdot P(E) + 0 \cdot P(E^\complement) 
               = P(E)
    \end{equation*}
\end{example}

\begin{example}
    $k$ buses carrying a total of $n_1 + n_2 + \dots + n_k$ students set off on
a trip. The $j$-th bus is carrying $n_j$ many students.
\begin{enumerate}[noitemsep, topsep=0em]
    \item 
    Upon arrival, one student is chosen at random, and every student on the
same bus as the selected student is given a candy. If each candy costs $\$1$
and $X$ denotes the total cost of the candy, find the distribution of $X$ and
compute $E[X]$.
    \item
    Upon arrival, one buses out of $k$-bus is chosen at random and each student
on that bus is given a candy. If each candy costs $\$1$ and $Y$ denotes the
cost of the candy, find the distribution of $Y$ and compute $E[Y]$.
    \item
    Show that $E[X] \geq E[Y]$.
\end{enumerate}
    Assume $n_i \neq n_j, \quad \forall i \neq j$.
\end{example}
\begin{solution} \quad                                                       \\
    \begin{enumerate}[noitemsep, topsep=0em]
        \item 
        possible values of $X$ are $n_1, n_2, \dots, x_k$ (By assumption these
        are all distinct).
        \begin{equation*}
            p_X(n_i) = P(X = n_i) = \frac{n_i}{n_1 + \dots +n_k} 
            \quad i = 1, 2, \dots, k
        \end{equation*}
        Hence, 
        \begin{equation*}
            E[X] = \sum_{i = 1}^k n_i \cdot p_X(n_i)
                 = \sum_{i = 1}^k n_i \cdot \frac{n_i}{n_1 + \dots + n_k}
                 = \sum_{i = 1}^k \frac{(n_i)^2}{n_1 + \dots + n_k}
                 = \frac{\sum_{i = 1}^k (n_i)^2}{\sum_{i = 1}^k n_i}
        \end{equation*}
        
        \item
        possible values of $Y$ are $n_1, n_2, \dots, n_k$ and,
        \begin{equation*}
            p_Y(n_i) = P(Y = n_i) = \frac{1}{k} \quad i = 1, 2, \dots, k
        \end{equation*}
        Hence,
        \begin{equation*}
            E[Y] = \sum_{i = 1}^k n_i \cdot p_Y(n_i)
                 = \frac{\sum_{i = 1}^k n_i}{k}
        \end{equation*}
        
        \item
        Since $\sum_{i=1}^k (n_i)^2 \geq \frac{(\sum_{i=1}^k n_i)^2}{k}$ by
        Cauchy-Schwarz inequality. It follows that, 
        \begin{equation*}
            E[X] \geq E[Y]
        \end{equation*}
    \end{enumerate}
\end{solution}

\begin{theorem}
    Let $X$ be on $\mathbb{R}^m$-valued discrete random variable and $g :
\mathbb{R}^m \rightarrow \mathbb{R}^n$ be a mapping function. If the set of
possible values of $X$ are $x_1, x_2, \dots$ and the distribution of $X$ is
given by $p_X(x_i)$, $i = 1, 2, \dots$, then
\begin{equation*}
    E[g(X)] = \sum_{i \geq 1} g(x_i) \cdot p_X(x_i)
\end{equation*}
\end{theorem}
\begin{proof}
    If the possible values of $Y = g(X)$ are $y_1, y_2, \dots$, then the
probability distribution of $Y$ is given by,
\begin{equation*}
    p_Y(y_j) = \sum_{i : g(x_i) = y_j} p_X(x_i) \quad j = 1, 2, \dots
\end{equation*}
Hence,
\begin{align*}
    E[g(X)] &= E[Y]                                                          \\
            &= \sum_{j \geq 1} y_j \cdot p_Y(y_j)                            \\
            &= \sum_{j \geq 1} \sum_{i : g(x_i) = y_j} g(x_i) \cdot p_X(x_i) \\
            &= \sum_{i \geq 1} g(x_i) \cdot p_X(x_i)
\end{align*}
as claimed.
\end{proof}

\begin{example}
    The distribution of $X$ is given by,
    \begin{align*}
        &p_X(-2) = 1 / 10                     &p_X(-1) = 2 / 10              \\
        &p_X(0) = 2 / 10                      &p_X(1) = 3 / 10               \\
        &p_X(2) = 2 / 10                      &
    \end{align*}
    Then,
    \begin{align*}
        E[X^2] &= (-2)^2 \cdot \frac{1}{10} +
                  (-1)^2 \cdot \frac{2}{10} +
                  0^2 \cdot \frac{2}{10} +
                  1^2 \cdot \frac{3}{10} +
                  2^2 \cdot \frac{2}{10}                                     \\
               &= \frac{4 + 2 + 0 + 3 + 8}{10}                               \\
               &= \frac{17}{10}
    \end{align*}
    This is often easier than finding the distribution of $X^2$ explicitly and
then computing $E[X^2]$.
\end{example}

\begin{theorem}
    If $X$ is an $\mathbb{R}^m$-valued discrete random variable and $g_i :
\mathbb{R}^m \rightarrow \mathbb{R}^n$, for $i = 1, 2, \dots, k$, then,
\begin{equation*}
    E[g_1(X) + g_2(X) + \dots + g_k(X)]
    = E[g_1(X)] + E[g_2(X)] + \dots + E[g_k(X)]
\end{equation*}
\end{theorem}
\begin{proof}
    \begin{align*}
           E[g_1(X) + g_2(X) + \dots + g_k(X)]
        &= \sum_{x} (g_1(x) + g_2(x) + \dots + g_k(x)) \cdot p_X(x)          \\
        &= \sum_{i = 1}^k \sum_{x} g_i(x) \cdot p_X(x)                       \\
        &= \sum_{i = 1}^k E[g_i(X)]
    \end{align*}
\end{proof}

\begin{theorem}[linearity of expectation]
    If $X$ is an $\mathbb{R}^d$ valued random variable, $b \in \mathbb{R}^d$,
and $a \in \mathbb{R}$, then 
\begin{equation*}
    E[aX + b] = aE[X] + b
\end{equation*}
\end{theorem}
\begin{solution} \quad                                                       \\
    \begin{equation*}
        E[aX+b] = E[aX] + E[b]
    \end{equation*}
    Then the expectation of the constant random variable $Y = b$ is 
    \begin{equation*}
        \begin{cases}
            E(Y) = b \cdot P(Y = b) = b                                      \\
            E(aX) = \sum_{x} a \cdot x \cdot p_X(x) 
                  = a \sum_{x} x \cdot p_X(x)
                  = a E[X]
        \end{cases}
        \Rightarrow
        E[aX + b] = E[aX] + E[b] = aE[X] + b
    \end{equation*}
    Thus the claim follows.
\end{solution}

\begin{example}
    If $X$ is a real-valued discrete random variable, then
    \begin{equation*}
          E[X^2 + 2 \sin{X} + 3]
        = E[X^2] + 2 E[\sin{X}] + 3
    \end{equation*}
\end{example}

\begin{theorem}
    If $X_1, X_2, \dots, X_k$ are $\mathbb{R}^d$-valued discrete randome
variables defined on the same sample space $S$ then,
\begin{equation*}
    E[X_1 + X_2 + \dots + X_k] = E[X_1] + E[X_2] + \dots + E[X_k]   
\end{equation*}
\end{theorem}
\begin{proof}
    Let $X = \langle x_1, x_2, \dots, x_k \rangle : S \rightarrow
\mathbb{R}^{kd}$ be a discrete random variable. Further, letting $g_i(X) =
X_i$, we see that,
\begin{align*}
    E[X_1 + X_2 + \dots + X_k] &= E[g_1(X) + g_2(X) + \dots + g_k(X)]        \\
                               &= \sum_{i = 1}^k E[g_i(X)]                   \\
                               &= \sum_{i = 1}^k E[X_i]
\end{align*}
\end{proof}
This is an \textbf{extremely useful fact}. Note that this theorem \textbf{does
not} require any assumption of independence.

\begin{example} \quad                                                        \\
\begin{itemize}[noitemsep, topsep=0em]
    \item
    If $X$ is a discrete real-valued random variable that is nonnegative (i.e.
the values assumed by $X$ are nonnegative), then $E[X] \geq 0$.
    \item
    If $X$ and $Y$ are two discrete real-valued random variables defined on the
same sample space and $X \geq Y$, then $E[X] \geq E[Y]$.
\end{itemize}
\end{example}
\begin{solution} \quad                                                       \\
\begin{itemize}[noitemsep, topsep=0em]
    \item 
    If the possible value of $X$ are $x_1, x_2, \dots$, then $x_i \geq 0 \quad
\forall i \geq 1$. Hence $E[X] = \sum_{i \geq 1} x_i \cdot p_X(x_i) \geq 0$.
    \item
    Consider $E[X- Y] = E[X] + E[-Y] = E[X] - E[Y]$. Since, $X \geq Y$, then
for any possible out come $x_1, x_2, \dots$ and $y_1, y_2, \dots$, we have
$\forall i \geq 1 \quad x_i \geq y_i$. Thus $X - Y$ is a nonnegative discrete
random variable, and by previous example, we have $E[X - Y] \geq 0 \Rightarrow
E[X] - E[Y] \geq 0 \Rightarrow E[X] \geq E[Y]$.
\end{itemize}
\end{solution}

\begin{example}
    Let $X$ be the number of times heads appears when a coin is tossed $n$
times. Assume that probability of landing heads in a single toss is $p$. Find
$E[X]$.
\end{example}
\begin{solution}
    Define random variable $X_i$ as below,
    \begin{equation*}
        X_i = \begin{cases}
            1   & i\text{-th toss results in heads}                          \\
            0   & \text{otherwise}
        \end{cases}
    \end{equation*}
    Then $E[X_i] = P(\text{heads in }i\text{-th toss}) = p$. Since $X = X_1 +
X_2 + \dots + X_n$,
    \begin{equation*}
        E[X] = E[X_1 + X_2 + \dots + X_n]
             = E[X_1] + E[X_2] + \dots + E[X_n]
             = n \cdot p
    \end{equation*}
\end{solution}
\note Assume the tosses to be independent one could write down the
distribution of $X$ and then compute the expectation. We will do it later.

\begin{theorem}
    If $X_1, X_2, \dots$ are $\mathbb{R}$-valued random variables defined on
the same sample space and each $X_i$ is nonnegative, then
    \begin{equation*}
        E[X_1 + X_2 + \dots] = \sum_{i = 1}^\infty E[X_i]
    \end{equation*}
\end{theorem}
\note Since $X_i \geq 0 \quad \forall i$, the series $\sum_{i = 1}^\infty$
always makes sense. It can converge to a finite real value or it can be
infinite. Same is true for $\sum_{i = 1}^\infty E[X_i]$.
\subsection{Variance of an Random Variable}
\begin{definition}
    If $X$ is an $\mathbb{R}$-valued random variable with $E[X^2] < \infty$,
then the variance of $X$ is defined as,
    \begin{equation*}
        V[X] = E[(X - E(X))^2]
    \end{equation*}
    It is also often denoted by $\sigma_X^2$. The standard deviation of $X$ is
defined as $\sqrt{V[X]} = \sigma_X$.
\end{definition}
Variance of an random variable is a measure of how dispensed it is around its
mean. 

\begin{theorem}
    \begin{equation*}
        V[X] = E[X^2] - (E[X])^2
    \end{equation*}
\end{theorem}
\begin{proof}
    \begin{align*}
        V[X] &= E[(X - E[X])^2]                                              \\
             &= E[X^2 - 2X \cdot E[X] + (E[X])^2]                            \\
             &= E[X^2] - 2 \cdot E[X] \cdot E[X] + (E[X])^2                  \\
             &= E[X^2] - (E[X])^2
    \end{align*}
\end{proof}

\begin{theorem}
    If $a, b \in \mathbb{R}$, then,
    \begin{equation*}
        V[aX + b] = a^2 V[X]
    \end{equation*}
\end{theorem}
\begin{proof}
    Let $Y = aX + b$, then, 
    \begin{align*}
                     E[Y] = aE[X] + b
        &\Rightarrow Y - E[Y] = a(X - E[X])                                  \\
        &\Rightarrow V[Y] = E[(Y - E[Y])^2]
                          = E[a^2 \cdot (X - E[X])^2]
                          = a^2 \cdot E[(X - E[X])^2]
                          = a^2 \cdot V[X]
    \end{align*}
\end{proof}

\begin{example}
    If $X$ has discrete uniform distribution on $\lbrace 1, 2, \dots, n
\rbrace$, i.e., 
    \begin{equation*}
        P(X = i) = \frac{1}{n} \quad i = 1, 2, \dots, n
    \end{equation*}
    Find $E[X]$ and $\sigma_X$.
\end{example}
\begin{solution}
    \begin{align*}
        & E[X] = \sum_{i = 1}^n \frac{i}{n} 
               = \frac{n \cdot (n+1)}{2n} 
               = \frac{n+1}{2}                                               \\
        & E[X^2] = \sum_{i = 1}^n \frac{i^2}{n}
                 = \frac{n \cdot (n_1) \cdot (2n+1)}{6n}
                 = \frac{(n+1) \cdot (2n+1)}{6}                              \\
        & V[X] = E[X^2] - (E[X])^2
               = \frac{(n+1) \cdot (2n+1)}{6} - (\frac{n+1}{2})^2
               = \frac{n+1}{12} \cdot (4n + 2 - 3n -3)
               = \frac{n^2 - 1}{12}                                          \\
        & \sigma_X = \sqrt{v[X]} = \sqrt{\frac{n^2 - 1}{12}}
    \end{align*}
\end{solution}

\begin{example}
    If $X$ denotes the number of the uppermost face while rolling a fair die,
then $X$ follows a discrete uniform distribution on $\lbrace 1, 2,
\dots, 6 \rbrace$. Find $E[X], \sigma_X$.
\end{example}
\begin{solution}
    \begin{equation*}
        E[X] = \frac{6 + 1}{2} = \frac{7}{2} \qquad
        V[X] = \frac{6^2 - 1}{12} = \frac{35}{12} \qquad
        \sigma_X = \frac{35}{12}
    \end{equation*}
\end{solution}
\subsection{Binomial Distribution}
\begin{definition}
    Let $n \geq 1$ and $p \in [0, 1]$. An random variable $X$ follows a
binomial distribution with parameters $n$ and $p$, in short $X \sim
\binomialdist{n}{p}$, if the probability mass function (pmf) of $X$ is given
by,
    \begin{equation*}
        p_X(k) = P(X = k) = {n \choose k}p^k \cdot (1-p)^{n-k}
        \quad
        k = 0, 1, \dots, n
    \end{equation*}
\end{definition}
\noindent
\textbf{Interpretation}: Suppose an experiment consists of $n$ identical trials
such that, 
\begin{itemize}[noitemsep, topsep=0em]
\item
    each trial results in one of two possible outcomes: success($S$) or failure
    ($F$);
\item
    possibility of success in each trial is $p$ ( and hence probability of
    failure is $q = 1 - p$);
\item
    the trials are independent.
\end{itemize}
Then the number of successes in the experiment follows $\binomialdist{n}{p}$
distribution.

\subsubsection*{Deriving the Binomial Probability Mass Function}
Suppose probability of landing head while tossing a coin $\frac{1}{4}$. Toss
the coin $3$ times independently. The sample space will be $S = \lbrace HHH,
HHT, HTH, THH, HTT, THT, TTH, TTT \rbrace$. Then, 
\begin{align*}
    P(HHH) &= \frac{1}{4}^3                                                  \\
    P(HHT) &= P(HTH) = P(THH) = (\frac{1}{4})^2 \cdot \frac{3}{4}            \\
    P(HTT) &= P(THT) = P(TTH) = \frac{1}{4} \cdot (\frac{3}{4})^2            \\
    P(TTT) &= (\frac{3}{4})^3
\end{align*}
Thus, probability of any sample point that has $k$ heads and $3-k$ tails (in
any order) is $(\frac{1}{4})^k \cdot (\frac{3}{4})^{3-k}$. Let $X =
\#\text{heads in three tosses}$. For $k = 0, 1, 2, 3$, 
\begin{align*}
    P(X = k) &= (\frac{1}{4})^k \cdot (\frac{3}{4})^{3 - k} \cdot
                (\text{the number of sample points that have $k$ heads and
                $3-k$ tails})                                                \\
             &= (\frac{1}{4})^k \cdot (\frac{3}{4})^{3 - k} \cdot
                {3 \choose k}
\end{align*}
Same reasoning applies to general $n$ and $p$.

\begin{example}
    $100$ couples live in a community. Each couple can have either $0$, $1$ or
$2$ children with probability $\frac{1}{3}$ each. Assume that the couples have
children independently. Find the distribution of $X = \text{number of couples
that have at least one child}$.
\end{example}
\begin{solution}
    \[ X \sim \binomialdist{100}{\frac{2}{3}} \]
\end{solution}

\subsubsection*{Bernoulli Distribution}
$\binomialdist{1}{p}$ distribution is called the Bernoulli distribution with
parameter $p$. We will write it as $\bernoullidist{p}$.Thus, if $Y \sim
\bernoullidist{p}$, then $P(Y = 1) = p$, and $P(Y = 0) = 1 - p$.

The total number of success $X$ in $n$ independent identical trials each
having probability of of success $p$, follows $\binomialdist{n}{p}$. Let $Y_i =
1_{\lbrace i\text{-th trial results in success} \rbrace}$. Then $Y_i \sim
\bernoullidist{p} \quad i = 1, 2, \dots, n$, and $X = \sum_{i = 1}^n Y_i$.

\begin{theorem}
    If $X \sim \binomialdist{n}{p}$, then,
    \begin{enumerate}[noitemsep, topsep=0em]
        \item $E[X] = n \cdot p$,
        \item $V[X] = n \cdot p \cdot (1 - p)$.
    \end{enumerate}
\end{theorem}
\begin{proof}
    As $X = \sum_{i = 1}^n Y_i$, where $Y_i \sim \bernoullidist{p}$,
    \[ E[Y_i] = 1 \cdot p + 0 \cdot (1 - p) = p \]
    \[ \Rightarrow 
       E[X] = E[\sum_{i = 1}^n Y_i] = \sum_{i = 1}^n E[Y_i] = n \cdot p \]
\end{proof}
\begin{proof}
    \begin{align*}
        E[X] &= \sum_{k = 0}^n k \cdot P(X = k)                              \\
             &= \sum_{k = 0}^n k \cdot {n \choose k} \cdot p^k \cdot 
                (1-p)^{n-k}                                                  \\
             &= \sum_{k = 1}^n k \cdot \frac{n!}{k!(n-k)!} \cdot p^k \cdot
                (1-p)^{n-k}                                                  \\
             &= \sum_{k = 1}^n \frac{n!}{(k-1)! (n-k)!} \cdot p^k \cdot 
                (1-p)^{n-k}                                                  \\
             &= n \cdot p \cdot \sum_{k = 1}^n \frac{(n - 1)!}{(k-1)!(n-k)!}
                \cdot p^{k - 1} \cdot (1-p)^{n-k}                            \\
             &= n \cdot p \cdot (p + (1 - p))^{n-1}                          \\
             &= n \cdot p
    \end{align*}
    \begin{align*}
        E[X(X - 1)] &= \sum_{k = 0}^n k \cdot (k - 1) \cdot P(X = k)        \\
                    &= \sum_{k = 2}^n k \cdot (k - 1) \cdot
                       {n \choose k} \cdot p^k \cdot (1-p)^{n-k}            \\
                    &= \sum_{k = 2}^n \frac{n!}{(k - 2)!(n - k)!} \cdot
                       p^k \cdot (1 - p)^{n - k}                            \\
                    &= n(n-1)p^2 \sum_{k = 2}^n \frac{(n-2)!}{(k-2)!(n-k)!}
                       \cdot p^{k - 2} \cdot (1-p)^{n - k}                  \\
                    &= n \cdot (n-1) \cdot p^2 \cdot (p + (1 - p))^{n - 2}  \\
                    &= n \cdot (n-1) \cdot p^2
    \end{align*}
    \begin{equation*}
        E[X^2] = E[X(X - 1)] + E[X] 
               = E[X(X - 1)] + n \cdot p
               = n \cdot (n - 1) \cdot p^2 + n \cdot p
    \end{equation*}
    \begin{align*}
        V[X] &= E[X^2] - (E[X])^2                                           \\
             &= n \cdot (n - 1) \cdot p^2 + n \cdot p - n^2 \cdot p^2       \\
             &= n^2p^2 - np^2 + np - n^2p^2                                 \\
             &= np(1-p)
    \end{align*}
\end{proof}

\begin{example}[exercise 3.58 of textbook]
    Let $X$ denote the number of defectives among four items selected randomly
from a large set that is known to contain $10\%$ defectives. If the repair cost
for the defectives is given by $C = 3X^2 + X + 2$, find $E[C]$.
\end{example}
\begin{solution}
    $X \sim \binomialdist{n}{p}$ where $n = 1$, and $p = 0.1$.Then, 
    \begin{align*}
        E[C] &= E[3X^2 + X + 2]                                             \\
             &= E[3X^2] + E[X] + 2                                          \\
             &= 3 (V[X] + (E[X])^2) + E[X] + 2                              \\
             &= 3 (np(1-p) + n^2p^2) + np + 2                               \\
             &= 3n^2p^2 - 3np^2 + 4np + 2                                   \\
             &= 3.96
    \end{align*}
\end{solution}

\begin{example}
    A communication system consists of $n$ components, each of which will
independently function with probability $p$. The system operates effectively if
at least half of its components function. For what values of $p$ is a
$5$-component system more likely to operate effectively than a $3$-component
system?
\end{example}
\begin{solution}
    The number of functioning components $\sim \binomialdist{n}{p}$.
Probability that a $5$-component system will be effective is,
\[ {5 \choose 3} p^3 (1-p)^2 + {5 \choose 4} p^4 (1 - p) + p^5 \]
Probability that a $3$-component system operates effectively is,
\[ {3 \choose 2} p^2 (1-p) + p^3 \]
A $5$-component system is more likely to operate effectively if and only if
\begin{align*}
    {5 \choose 3} p^3 (1-p)^2 + {5 \choose 4} p^4 (1 - p) + p^5 &>
    {3 \choose 2} p^2 (1-p) + p^3                                           \\
    10p^3(1-p)^2 + 5p^4(1-p) + p^5 &> 3p^2(1-p) + p^3 + p^3                 \\
    (1-p)(10p - 10p^2 + 5p^2 - 3 - p - p^2) &> 0                            \\
    (1-p)(9p - 6p^2 - 3) = (1-p)(3p - 2p^2 - 1) = (1-p)(1-p)(2p-1) &> 0     \\
\end{align*}
Hence, $p > \frac{1}{2}$
\end{solution}
\subsection{Poisson Distribution}
\begin{definition}
    An random variable $X$ is said to follow Poisson distribution with
parameter $\lambda$ ($\lambda > 0$), in short $X \sim \poissondist{\lambda}$,
if the pmf of $X$ is given by, 
\begin{equation*}
    p_X(k) = P(X = k) = e^{-\lambda} \cdot \frac{\lambda^k}{k!}
    \quad k = 0, 1, 2, \dots
\end{equation*}
\end{definition}
A Poisson random variable approximates the total number of successes in a large
number of independent trials each of which has a very small probability of
success. 

\begin{example}
    If probability of hooking a fish at each attempt is $0.04$, then the number
of fish caught after $50$ attempts follows $\binomialdist{50}{0.04} \approx
\poissondist{2}$.
\end{example}

\begin{theorem}
    Let $X_n \sim \binomialdist{n}{p_n}$ where $n \cdot p_n \xrightarrow{n
\rightarrow \infty} \lambda$ for some $\lambda \in (0, \infty)$. Then for each
$k = 0, 1, 2, \dots$,
    \[ P(X_n = k) \rightarrow   e^{-\lambda} \cdot \frac{\lambda^k}{k!}
                              = P(Y = k)                                     \]
    where $Y \sim \poissondist{\lambda}$.
\end{theorem}
\begin{proof}
\begin{align*}
    P(X_n = k) &= {n \choose k} p_n^k (1 - p_n)^{n - k}                      \\
               &= \frac{n(n-1) \cdot (n - k + 1)}{k!} \cdot p_n^k \cdot 
                  (1 - p_n)^n \cdot (1 - p_n)^{-k}                           \\
               &\xrightarrow{n \rightarrow \infty} 
                  \frac{1}{k!} \cdot 1 \cdot \lambda^k \cdot e^{-\lambda} 
                  \cdot 1                                                    \\
               &= e^{-\lambda} \frac{\lambda^k}{k!}  
\end{align*}
Note that $n \cdot p_n \xrightarrow{n \rightarrow \infty} \lambda \in (0,
\infty)$, and $p_n \xrightarrow{n \rightarrow \infty} 0$.
\end{proof}

\begin{theorem}
    If $X \sim \poissondist{\lambda}$, then,
    \begin{equation*}
        E[X] = \lambda \qquad V[X] = \lambda
    \end{equation*}
\end{theorem}
\begin{proof}
    \begin{align*}
        & E[X] = \sum_{k = 0}^\infty k \cdot P(X = k)                         
               = \sum_{k = 1}^\infty k \cdot e^{-\lambda} \cdot
                 \frac{\lambda^k}{k!}                                         
               = e^{-\lambda} \cdot \sum_{k = 1}^\infty
                 \frac{\lambda^k}{(k-1)!}                                     
               = e^{-\lambda} \cdot \lambda \cdot \sum_{k = 1}^\infty
                 \frac{\lambda^{k-1}}{(k-1)!}                                 
               = e^{-\lambda} \cdot \lambda \cdot e^\lambda 
               = \lambda                                                     \\
        & E[X(X-1)] = \sum_{k = 2}^\infty k(k-1) \cdot P(X = k)
                    = \sum_{k = 2}^\infty k(k-1) e^{-\lambda} \cdot
                      \frac{\lambda^k}{k!}
                    = e^{-\lambda} \cdot \sum_{i = 2}^\infty
                      \frac{\lambda^k}{(k-2)!}
                    = \lambda^2 \cdot e^{-\lambda} \cdot e^\lambda
                    = \lambda^2                                              \\
        & E[X^2] = E[X(X - 1)] + E[X] = \lambda^2 + \lambda                  \\
        & V[X] = E[X^2] - (E[X])^2  
               = \lambda^2 + \lambda - \lambda^2
               = \lambda
    \end{align*}
\end{proof}

\begin{example}
    A book has $500$ pages. The probability that a typographical error occurs
in a page is $0.005$. Use Poisson approximation to find the probability that
not more than one page contains a typo.
\end{example}
\begin{solution}
    Let $X$ denote the number of pages that contain a typo. Then $X
\sim \binomialdist{500}{0.005}$. Hence,
    \[ P(X \leq 1) = P(X = 0) + P(X = 1) \approx P(Y = 0) + P(Y = 1) \]
    where, $Y \sim \poissondist{2.5}$ ($500 \cdot 0.005 = 2.5$). Hence,
    \[ 
       P(X \leq 1) = e^{-2.5} \cdot (1 + \frac{2.5}{1!})
                   = 3.5 \cdot e^{-2.5}
    \]
\end{solution}

\begin{example}
    An item is a store has a marked price of $\$100$. For each customer
purchasing the item during a particular day, the store owner reduces the price
of the item by a factor of one-half (i.e. the first customer pays $\$50$, the
second pays $\$25$ etc.). If the number of customers who purchase the item
during the day follows a Poisson distribution with mean $2$, find the expected
price of the item at the end of the day.
\end{example}
\begin{solution}
    If $Y$ denotes the number of customers who purchase that item during the
day, then,
    \[ Y \sim \poissondist{2} (\lambda = 2 \text{ as mean equals two.}) \]
    The cost at the end of the day is $C = 100 \cdot (\frac{1}{2})^Y$. Hence, 
    \[
        E[(\frac{1}{2})^Y] = \sum_{k = 0}^\infty (\frac{1}{2})^k \cdot
                             e^{-2} \cdot \frac{2^k}{k!}
                           = \sum_{k = 0}^\infty e^{-2} \cdot \frac{1}{k!}
                           = e^{-2} \cdot e 
                           = e^{-1}
    \]
    \[
        E[C] = E[100 \cdot (\frac{1}{2})^Y]
             = 100 \cdot E[(\frac{1}{2})^Y]
             = 100 \cdot e^{-1}
    \]
\end{solution}

\begin{example}
    A radioactive piece of rock emits particles at random time intervals. Let
$X_1$ denote the time at which the first particle is emitted. If the number of
particles emitted in a time interval of length $t$ follows a Poisson
distribution with mean $\lambda t$, find $P(X_1 > x)$ for $x > 0$.
    \begin{figure*}[!htp]
        \centering
        \def\svgwidth{\textwidth}
        \includesvg[./section3/figure/]{sec3-sub5-fig1}
    \end{figure*}
\end{example}
\begin{solution}
    Let $N_x \coloneqq \#\text{particles emitted in time interval }[0, x]$.
Then, 
    \[ N_x \sim \poissondist{\lambda x} \] 
    Now, 
    \[P(X_1 > x) = P(N_x = 0) = e^{-\lambda x} \]
\end{solution}
\subsection{Geometric Distribution}
\begin{definition}
    An random variable $X$ follows a geometric distribution with probability of
success $p$, or simply $X \sim \geometricdist{p}$, if the pmf of $X$ is given
by, 
    \[ p_X(k) = P(X = k) = p \cdot (1-p)^{k - 1} \quad k = 1, 2, 3, \dots \]
\end{definition}
This is indeed a pmf, since $p_X(k) \geq 0, \forall k \geq 1$, and 
    \[ \sum_{k = 1}^\infty p_X(k) = p \cdot \sum_{k = 1}^\infty (1 - p)^{k - 1}
                                  = \frac{p}{1 - (1 - p)}
                                  = 1                                        \]

\noindent
\textbf{Interpretation}: suppose that, 
\begin{itemize}[noitemsep, topsep=0em]
    \item you have an infinite sequence of independent and identical trials;
    \item each trial can result in either success or failure;
    \item probability of success in each trial is $p$ (hence probability of
          failure in each trial is $q = 1-p$).
\end{itemize}
Let $Y$ denote the first trial that results in a success. Then $Y \sim
\geometricdist{p}$. Then, 
\begin{align*}
    P(Y = k) &= P( \bigcap_{i=1}^{k-1} \lbrace \text{failure in $i$-th trial}
                \rbrace \cap \lbrace \text{success in $k$-th trial} \rbrace) \\
             &= \prod_{i=1}^{k - 1} P(\text{failure in $i$-th trial}) \cdot 
                P(\text{success in $k$-th trial})                            \\
             &= (1-p)^{k-1} \cdot p
\end{align*}

\begin{example}
    Exactly $10^6$ lottery tickets are sold every week. Jim buys $2$ tickets
every week. If $X$ denotes the number of weeks Jim has to wait before he wins
for the first time, what is the distribution of $X$?
\end{example}
\begin{solution}
    \[ X \sim \geometricdist{\frac{2}{10^6}} \]
\end{solution}

\begin{theorem}
    If $X \sim \geometricdist{p}$, then
    \[ E[X] = \frac{1}{p} \qquad V[X] = \frac{1-p}{p^2} \]
\end{theorem}
\begin{proof}
    \[ E[X] = \sum_{k = 1}^\infty k \cdot p \cdot (1-p)^{k - 1}
            = p \cdot \frac{1}{(1 - ( 1 - p))^2}
            = \frac{1}{p}                                                    \]
    \begin{align*}
        E[X^2] &= \sum_{k = 1}^\infty k^2 \cdot p \cdot (1 - p)^{k - 1}      \\
               &= \sum_{k = 1}^\infty k(k - 1) \cdot p \cdot (1 - p)^{k - 1} +
                  \sum_{k = 1}^\infty k \cdot p \cdot (1 - p)^{k - 1}        \\
               &= p(1 - p) \sum_{k = 1}^\infty k(k - 1)(1 - p)^{k - 2} +
                  \frac{1}{p}                                                \\
               &= p(1 - p) \cdot 2 \cdot (1 - (1 - p))^{-3} + \frac{1}{p}    \\
               &= \frac{2(1-p)}{p^2} + \frac{1}{p}                           
                = \frac{2}{p^2} - \frac{1}{p}
    \end{align*}
    \[ V[X] = E[X^2] - (E[X])^2
            = \frac{2}{p^2} - \frac{1}{p} - (\frac{1}{p})^2
            = \frac{1}{p^2} - \frac{1}{p}
            = \frac{1 - p}{p^2}                                              \]
\end{proof}

\begin{example}
    If $Y \sim \geometricdist{p}$, show that, 
    \begin{enumerate}[noitemsep, topsep=0em]
        \item $P(Y > k) = (1 - p)^k$, for $k = 1, 2, 3, \dots$, 
        \item $P(Y > k_1 + k+2 \vert Y > k_1) = (1 - p)^{k_2} = P(Y > k_2)$,
              for $k_1, k_2 = 1, 2, 3, \dots$,
        \item is called the memoryless property of the geometric distribution
    \end{enumerate}
\end{example}
\begin{solution} \quad                                                       \\
    \begin{enumerate}[noitemsep, topsep=0em]
        \item 
        \begin{align*}
           P(Y > k) &= \sum_{j = k + 1}^\infty P(Y = j)                      \\
                    &= \sum_{j = k + 1}^\infty p \cdot (1 - p)^{j - 1}       \\
                    &= p \cdot (1 - p)^k \cdot \sum_{j = k + 1}^\infty        
                       (1 - p)^{j - k - 1}                                   \\
                    &= p (1 - p)^k \cdot \frac{1}{1 - (1 - p)}               \\
                    &= (1 - p)^k                                             
        \end{align*}
        Alternatively,
        \[ P(Y > k) = P(\text{zero successes in first $k$ trials})
                    = (1 - p)^k                                              \]
        \item
        \[   P(Y > k_1 + k_2 \vert Y > k_1)
           = \frac{P(\lbrace Y > k_1 + k_2 \rbrace \cap \lbrace Y > k_1
             \rbrace)}{P(Y > k_1)}  
           = \frac{Y > k_1 + k_2}{P(Y > k_1)}
           = \frac{(1 - p)^{k_1 + k_2}}{(1 - p)^{k_1}}
           = (1 - p)^{k_2}                                                   \]
    \end{enumerate}
\end{solution}

\begin{example}
    Snakes and ladders is a two-players game in which Player $1$ and Player $2$
take turns rolling a fair die, and the player to roll $1$ first gets to move
his chip first. Assume that player $1$ rolls the die first.
\begin{enumerate}[noitemsep, topsep=0em]
    \item What is the probability that Player $1$ moves his chip first?
    \item Given that player $1$ moves first, what is the probability that he
moves on his second roll (overall third roll of the die)?
\end{enumerate}
\end{example}
\begin{solution}
    \begin{enumerate}[noitemsep, topsep=0em]
        \item 
        Let $Y$ be the first time $1$ turns up. Then,
        \begin{align*}
            P(\text{Player $1$ moves first})
                &= P(Y = 1) + P(Y = 3) + P(Y = 5) + \dots                    \\
                &= \sum_{k = 0}^\infty \frac{1}{6}(1 - \frac{1}{6})^{2k}     \\
                &= \frac{1}{6} \sum_{k = 0}^\infty (\frac{5}{6})^{2k}        \\
                &= \frac{\frac{1}{6}}{1 - \frac{25}{36}}
                 = \frac{1}{6} \cdot \frac{36}{11}
                 = \frac{6}{11}
        \end{align*}
        \item
        \[
            P(Y = 3 \vert \text{Player $1$ moves first})
                = \frac{P(Y = 3)}{\frac{6}{11}}                             
                = (\frac{5}{6})^2 \cdot \frac{1}{6} \cdot \frac{11}{6}      
                = \frac{275}{1296}
        \]
    \end{enumerate}
\end{solution}
\subsection{Negative Binomial Distribution}
This is a generalization of the geometric distribution. 
\begin{definition}
    An random variable $Y$ follows a negative binomial distribution with
parameters $r$ and $p$, where $r \in \lbrace 1, 2, \dots \rbrace$ and $p
\in [0, 1]$, if its pmf is given by, 
\[ p_Y(k) = P(Y = k) = {{k - 1} \choose {r - 1}} p^r (1 - p)^{k - r}
   \quad k = r, r+1, r+2, \dots                                              \]
\end{definition}
This is a pmf, since, 
\[   \sum_{k = r}^\infty p_Y(k) 
   = \sum_{k = r}^\infty {{k - 1} \choose {r - 1}} p^r (1 - p)^{k - r}  
   = p^r ((1 - (1 - p))^{-r}
   = p^r \cdot p^{-r}
   = 1                                                                       \]

\noindent
\textbf{Interpretation}: Same setup as a geometric distribution:
\begin{enumerate}[noitemsep, topsep=0em]
    \item infinite sequence of independent and identical trials;
    \item each trial has two possible outcomes - success and failure;
    \item probability of success in each trial is $p$, and probability of
          failure in each trial is $(1 - p)$. 
\end{enumerate}
If the $r$-th success occurs in the $y$-th trial, then,
\[ Y \sim \negbinomialdist{r}{p} \]
(When $r = 1$, we get a $\binomialdist{p}$ distribution). As, for $k \geq n$
\[
    P(Y = k) = P(\lbrace \text{success in $k$-th trial} \rbrace \cap
                 \lbrace \text{$r-1$ success in the first $k-1$ trials}\rbrace)
\]
Note that the probability of any sequence of successes and failures of length
$k - 1$ that has $r - 1$ success and $k - r$ failures is $p^{r - 1}(1 - p)^{k -
r}$. Thus,
\begin{align*}
    P(Y = k) &= p \cdot p^{r - 1}(1 - p)^{k - r} \cdot
                (\# \text{sequences of length $k - 1$ that have $r - 1$
                successes and $k - r$ failures})                             \\
             &= p^r \cdot (1 - p)^{k - r} \cdot {{k - 1} \choose {r - 1}}  
\end{align*}
Suppose the first success occurs at the $X_1$-th trial, and then the second
success occurs $X_2$ many trials after that. Define the random  variables,
$X_2, X_3, \dots, X_r$ similarly. 
\begin{figure*}[!htp]
    \centering
    \def\svgwidth{\textwidth}
    \includesvg[./section3/figure/]{sec3-sub7-fig1}
\end{figure*}

\noindent
Note that,
\[ X_1 \sim \geometricdist{p} \quad X_2 \sim \geometricdist{p} \quad
   \dots \quad X_r \sim \geometricdist{p}                                   \]
and
\[ Y = X_1 + \dots + X_r                                                    \]
Thus a $\negbinomialdist{r}{p}$ random variable can be expressed as a sum of
$r$ $\geometricdist{p}$ random variables. 

\begin{theorem}
    If $Y \sim \negbinomialdist{r}{p}$, then,
    \[ E[Y] = \frac{r}{p} \qquad V[Y] = \frac{r(1 - p)}{p^2}                \]
\end{theorem}
\begin{proof}
    If $Y \sim \negbinomialdist{r}{p}$ then,
    \[ E[Y] = E[X_1 + X_2 + \dots + x_r]                                    \]
    where $X_i \sim \geometricdist{p} \quad i = 1, 2, \dots, r$. Hence,
    \[ E[Y] = E[X_1] + E[X_2] + \dots + E[X_r]
            = \frac{r}{p}                                                   \]
    Alternatively,
    \begin{align*}
        E[Y] &= \sum_{k = r}^\infty k \cdot {{k - 1} \choose {r - 1}} p^r
                (1 - p)^{k - r}                                             \\
             &= \sum_{k = r + 1}^\infty (k - r) \cdot {{k - 1} \choose {r - 1}}
                p^r \cdot (1 - p)^{k - r} + r \sum_{k = r}^\infty {{k - 1} 
                \choose {r - 1}} p^r (1 - p)^{k - r}                        \\
             &= p^r (1 - p) (\sum_{k = r + 1}^\infty (k - r){{k - 1} \choose
                {r - 1}} (1 - p)^{k - r - 1}) + r                           \\
             &= p^r (1 - p) (- \frac{\delta}{\delta p} \sum_{k = r}^\infty
                {{k - 1} \choose {r - 1}} \cdot (1 - p)^{k - r}) + r        \\
             &= p^r (1 - p)(- \frac{\delta}{\delta p} (1 - (1 - p))^{-r})
                + r                                                         \\
             &= p^r (1 - p) \cdot \frac{r}{p^{r + 1}} + r                   \\
             &= \frac{r(1 - p)}{p} + r
              = \frac{r}{p}
    \end{align*}
    \[ V[Y] = E[Y^2] - (E[Y])^2 = E[Y^2] - \frac{r^2}{p^2}                  \]
    To compute $E[Y^2]$, one can use a similar technique, $i.e.$ differentiate
the power series twice. The calculation is slightly larger.
\end{proof}

\begin{example}
    A radio station asks a question with four possible choices for its answers
and asks people to call and answer the question. The second caller to answer
correctly will win a special price. Assuming that people just make a random
guess about the answer, find the probability that the fifth caller will win the
prize. 
\end{example}
\begin{solution}
    If the $X$-th caller wins the prize, then $X \sim
\negbinomialdist{2}{\frac{1}{4}}$. Hence, 
    \[ P(X = 5) = {4 \choose 1} (\frac{1}{4})^2 (\frac{3}{4})^3
                = \frac{27}{256}                                            \]
\end{solution}
\subsection{Hypergeometric Distribution}
\begin{definition}
    Suppose $\mathbf{N}, n$ and $r$  are positive integers such that $n \leq
\mathbf{N}$ and $r \leq \mathbf{N}$. Then an random variable $Y$ follows a
hypergeometric distribution with parameters $\mathbf{N}$, $n$ and $r$, or
simply $Y \sim \hypergeometricdist{\mathbf{N}}{n}{r}$, if
\[ p_Y(k) = 
   P(Y = k) = \frac{{r \choose k} {{\mathbf{N} - r} \choose {n - k}}}       
              {{\mathbf{N} \choose n}}                                       \]
where $k$ is an integer such that, $0 \leq k \leq r$ and $0 \leq n - k \leq
\mathbf{N} - r$.
\end{definition}

\noindent
This is indeed a pmf. Clearly, $P(Y = k) \geq 0$. To see why $\sum_{k = 0}^n
P(Y = k) = 1$, note that, 
\[ (1 + x)^\mathbf{N} = (1 + x)^r \cdot (1 + x)^{\mathbf{N} - r}             \]
and the coefficient of $x^n$ on the left side is ${\mathbf{N} \choose n}$,
where as the right side can be expanded as, $(\sum_{j = 0}^r {r \choose j}
\cdot x^j) \cdot (\sum_{s = 0}^{\mathbf{N} - r} {{\mathbf{N} - r} \choose s}
\cdot x^s)$, and hence the coefficient of $x^n$ on the right side is 
$\sum_{k = 0 \lor (n + r - \mathbf{N})}^{r \land n} {r \choose k} \cdot
{{\mathbf{N} - r} \choose {n - k}}$. Hence, 
\[ 
    \sum_{k = 0 \lor (n + r - \mathbf{N})}^{r \land n} 
        {r \choose k} \cdot
        {{\mathbf{N} - r} \choose {n - k}}
    =
    {{\mathbf{N}} \choose n}
\]
which shows that $\sum P(Y = k) = 1$.

\note There is a simple combinatorial argument for proving the above identity
which we will do next.

\noindent
\textbf{Interpretation}: suppose an jar contains $\mathbf{N}$ distinguishable
balls of which $r$ are red and $(\mathbf{N} - r)$ are blue, and you select an
unordered sample of $n$ balls without replacement at random. If $Y$ denotes the
number of red balls in the sample, then $Y \sim
\hypergeometricdist{\mathbf{N}}{n}{r}$.

Total number of possible samples is $\mathbf{N} \choose n$. The number of
sample in which there are exactly $k$ read and $(n - k)$ blue balls is ${r
\choose k} {{\mathbf{N} - r} \choose {n - k}}$. Hence the claim follows.

\begin{theorem}[Binomial approximation to Hypergeometric Distribution]
    If $\mathbf{N} \rightarrow \infty$, $\frac{r_\mathbf{N}}{\mathbf{N}}
\rightarrow p \in (0, \infty)$, $n$ is kept fixed, and $Y_{\mathbf{N}, n,
r_\mathbf{N}} \sim \hypergeometricdist{\mathbf{N}}{n}{r_\mathbf{N}}$, then 
\[
    \lim_{\mathbf{N} \rightarrow \infty} P(Y_{\mathbf{N}, n, r_\mathbf{N}} = k)
  = {n \choose k} \cdot p^k \cdot (1 - p)^{n - k}
\]
for $k = 0, 1, \dots, n$.
\end{theorem}
\note An unordered sample (without replacement) of $n$ balls out of
$\mathbf{N}$ many balls can be drawn by sequentially drawing $n$ balls without
replacement and then ignoring the order in which they were drawn. The theorem
above essentially says that if both $\mathbf{N}$ and $r_\mathbf{N}$ are large,
such that $\frac{r_\mathbf{N}}{\mathbf{N}} \approx p$, then sampling without
replacement is almost equivalent to sampling with replacement which corresponds
to binomial distribution with parameters $n$ and $p$.
\begin{proof}
    First note that the bounds $0 \lor (n + r_\mathbf{N} - \mathbf{N}) \leq k
\leq r_\mathbf{N} \land n$ correspond to $0 \leq k \leq n$ in the case
$\mathbf{N} \rightarrow \infty$ limit. Now for every $0 \leq k \leq n$,
\begin{align*}
    \frac{{r_\mathbf{N} \choose k}{{N - r_\mathbf{N}} \choose {n -k}}}
         {{\mathbf{N} \choose n}}
 &= \frac{r_\mathbf{N} (r_\mathbf{N} - 1) \dots (r_\mathbf{N} - k + 1)}{k!}
    \cdot
    \frac{(\mathbf{N} - r_\mathbf{N}) \dots (\mathbf{N} - r_\mathbf{N} - n
         + k + 1)}{(n-k)!}
    \cdot
    \frac{n!}{\mathbf{N}(\mathbf{N} - 1) \dots (\mathbf{N} - n + 1)}         \\
 &\approx \frac{r_\mathbf{N}^k}{k!} \cdot
          \frac{(\mathbf{N} - r_\mathbf{N})^{n - k}}{(n - k)!} \cdot
          \frac{n!}{\mathbf{N}^n}                                            \\
 &= \frac{r_\mathbf{N}^k \mathbf{N}^{n - k}}{\mathbf{N}^n} \cdot
    \frac{n!}{k! (n - k)!} \cdot
    (1 - \frac{r_\mathbf{N}}{\mathbf{N}})^{n - k}                            \\
 &\approx p^k \cdot {n \choose k} \cdot (1 - p)^{n - k}
\end{align*}
\end{proof}

\begin{theorem}
    If $Y \sim \hypergeometricdist{\mathbf{N}}{n}{r}$, then,
    \[ E[Y] = \frac{n \cdot r}{\mathbf{N}} \qquad
       V[Y] = n \cdot 
              \frac{r}{\mathbf{N}} \cdot 
              \frac{\mathbf{N} - r}{\mathbf{N}} \cdot
              \frac{\mathbf{N} - n}{\mathbf{N} - 1}                          \]
\end{theorem}
\subsection{Moments and Moment-Generating Function (mgf)}
\begin{definition}
    The $k$-th moment of a real-valued random variable $X$ is defined to be
$E[X^k]$, often denoted by $M_k^\prime$. The $k$-th central moment of $X$ is
defined to be,
    \[ M_k \coloneqq E[X - M_1^\prime]^k = E[X- E[X]]^k                      \]
\end{definition}
\note $M_1^\prime = E[X]$ and $M_2 = V[X]$.

\begin{definition}
    The moment-generating function (mgf) of a real-valued random variable $X$
is defined to be,
    \[ m_X(t) \coloneqq E[e^{tX}] \quad t \in \mathbb{R}                     \]
\end{definition}
\note Sine $e^{tX}$ is a nonnegative random variable, $E[e^{tX}]$ always makes
sense, but it can be infinite for some values of $t$.

\note $m_X(0) = 1$ and $m_X(t) \geq 0 \quad \forall t$

\begin{example}[examples of real-valued random variable for which not all
                moments are finite]
\[ p_1(n) = P(X_1 = n) = \frac{6}{\pi^2 n^2} \quad n = 1, 2, 3, \dots       \]
Clearly $P_1(n) \geq 0$. That $\sum_{n = 1}^\infty p_1(n) = 1$ follows from the
fact that $\sum_{n = 1}^\infty \frac{1}{n^2} = \frac{\pi^2}{6}$. Now,
\[ E[X_1] = \frac{6}{\pi^2} \sum_{n = 1}^\infty \frac{n}{n^2}
          = \frac{6}{\pi^2} \sum_{n = 1}^\infty \frac{1}{n}
          = \infty                                                          \]
Now consider,
\[ p_2(n) = P(X_2 = n) = \frac{1}{\zeta(3) n^3} \quad n = 1, 2, 3, \dots    \]
Then
\[  E[X_2] = \frac{1}{\zeta(3)} \sum_{n = 1}^\infty \frac{1}{n^2} < \infty
    \qquad
    E[X_2^2] = \frac{1}{\zeta(3)} \sum_{n = 1}^\infty \frac{1}{n} = \infty  \]
\end{example}

\begin{theorem}
    If the mgt of $X$, $m_X(t)$ is finite on $(-a, a)$, for some $a > 0$, then
    \[ E[\vert X \vert^\beta] < \infty \quad \forall \beta > 0              \]
\end{theorem}
\begin{proof}
    For every $\alpha, \beta > 0$, $\exists c_{\alpha, \beta} > 0$, such that,
    \[ y^\beta \leq c_{\alpha, \beta} e^{\alpha y} \quad \forall y \geq 0   \]
    Hence,
    \[ X^\beta \cdot 1_{\lbrace X > 0 \rbrace} \leq
           c_{\alpha, \beta} e^{\alpha X} \cdot 1_{\lbrace X > 0 \rbrace}
       \qquad \Rightarrow
       E[X^\beta \cdot 1_{\lbrace X > 0 \rbrace}]
       \leq c_{\alpha, \beta} E[e^{\alpha X} \cdot 1_{\lbrace X > 0 \rbrace}]
       \leq c_{\alpha, \beta} E[e^{\alpha X}]
       \leq \infty
    \]
    Whenever $0 < \alpha < a$, By considering $-X$ instead of $X$ we get,
    \[ \begin{cases}
           E[\vert X \vert^\beta \cdot 1_{\lbrace X > 0 \rbrace}] < \infty   \\
           E[\vert X \vert^\beta \cdot 1_{\lbrace X < 0 \rbrace}] < \infty
       \end{cases}
       \Rightarrow
       E[\vert X \vert^\beta] =
           E[\vert X \vert^\beta \cdot 1_{\lbrace X > 0 \rbrace}] +
           E[\vert X \vert^\beta \cdot 1_{\lbrace X < 0 \rbrace}]
       < \infty                                                              \]
\end{proof}

\begin{theorem}
    If the mgf of $X$, $m_X(\cdot)$, is finite on the interval $(-a, a)$, then
$m_X(\cdot)$ admits a power-series expansion on $(-a, a)$:
\begin{equation}
    \label{section4_thm1_ref1}
    m_X(t) = \sum_{k = 0}^\infty \frac{t^k}{k!} E[X^k]
    \quad \text{for } t \in (-a, a)
\end{equation}
   Consequently, $m_X(\cdot)$ is differentiable infinitely many times on $(-a,
a)$, and,
\[   \frac{\delta^r}{\delta t^r} m_X(t)
   = \sum_{k = 0}^\infty \frac{t^k}{k!} E[X^{k+r}]
   \quad \text{for } t \in (-a, a), r = 1, 2, 3, \dots                      \]
    In particular,
\[        \left. \frac{\delta^r}{\delta t^r} m_X(t) \right\vert_{t = 0} =E[X^r]
   \qquad \left. \frac{\delta}{\delta t} m_X(t) \right\vert_{t = 0} = E[X],
          \left. \frac{\delta^2}{\delta t^2} m_X(t) \right
              \vert_{t = 0} = E[X^2],
          \dots                                                             \]
\end{theorem}
\note \ref{section4_thm1_ref1} is the reason $m_X(\cdot)$ is called the moment
generating function.

\begin{example}[Binomial Distribution]
    $X \sim \binomialdist{n}{p}$,
    \[
         E[e^{tX}] = \sum_{k = 0}^n e^{tk} {n \choose k} p^k (1 - p)^{n - k}
                   = \sum_{k = 0}^n {n \choose k} (pe^t)^k (1 - p)^{n - k}
                   = (1 - p + p e^t)^n
    \]
    Note that,
    \[
         \left. \frac{\delta}{\delta t} (1 - p + p e^t)^n \right \vert_{t = 0}
       = \left. n (1 - p + p e^t)^{n - 1} \cdot p \right \vert_{t = 0}
       = n p
       = E[X]
    \]
\end{example}

\begin{example}[Poisson Distribution]
    $X \sim \poissondist{\lambda}$,
    \[
         E[e^{tX}] = \sum_{k = 0}^\infty e^{tk} e^{-\lambda} \cdot
                     \frac{\lambda^k}{k!}
                   = \sum_{k = 0}^\infty e^{-\lambda} \cdot
                     \frac{(\lambda e^t)^k}{k!}
                   = e^{-\lambda} \cdot e^{\lambda e^t}
                   = e^{\lambda e^t - \lambda}
    \]
    Note that,
    \[
         \left. \frac{\delta}{\delta t} e^{\lambda e^t - \lambda} \right
             \vert_{t = 0}
       = \left. \lambda \cdot e^{-\lambda + \lambda e^t + t} \right
             \vert_{t = 0}
       = \lambda
       = E[X]
    \]
\end{example}

\begin{example}[Geometric Distribution]
    $X \sim \geometricdist{p}$,
    \[
        E[e^{tX}] = \sum_{k = 1}^\infty e^{kt} \cdot (1 - p)^{k - 1} \cdot p
                  = p \cdot e^t \cdot \sum_{r = 0}^\infty e^{rt} (1 - p)^r
    \]
    This sum is finite only when,
    \[ (1 - p) \cdot e^t < 1 \Rightarrow t < -\log{(1 - p)}                  \]
    For $t \in (- \infty, \vert \log{(1 - p)} \vert)$,
    \[ E[e^{tX}] = \frac{p \cdot e^t}{1 - (1 - p)e^t}                        \]
    Note that,
    \[
\left. \frac{\delta}{\delta t} \frac{p \cdot e^t}{1 - (1 - p)e^t} \right
            \vert_{t = 0}
      = \left. \frac{e^t p}{1 - e^t(1 - p)} +
               \frac{e^{2t} (1 - p) p}{(1 - e^t(1 - p))^2}
                                                           \right\vert_{t = 0}
      = 1 + \frac{1 - p}{p}
      = \frac{p + 1 - p}{p}
      = \frac{1}{p}
                                                                             \]
\end{example}

\begin{theorem}
    Suppose $X$ and $Y$ are two real-valued random variables with respective
mgfs, $m_X(\cdot)$ and $m_Y(\cdot)$. If $\exists a > 0$ such that $m_X(t) <
\infty \quad \forall t \in (-a, a)$, and $m_Y(t) < \infty \quad \forall t
\in (-a, a)$, and $m_X(t) = m_Y(t) \forall t \in (-a, a)$, then $X$ and $Y$
have the same probability distribution.
\end{theorem}

\begin{example}
    If $m_X(t) = \frac{1}{6} e^t + \frac{2}{6} e^{2t} + \frac{3}{6} e^{3t}$,
find,
    \begin{enumerate}[noitemsep, topsep=0em]
        \item $E[X]$,
        \item $V[X]$, and,
        \item the distribution of $X$.
    \end{enumerate}
\end{example}
\begin{solution} \quad                                                       \\
    \begin{enumerate}[noitemsep, topsep=0em]
        \item
        \[
             m_X^\prime(t) 
           = \frac{1}{6}e^t + \frac{4}{6}e^{2t} + \frac{9}{6}e^{3t}
           \Rightarrow
           E[X] = m_X^\prime(0) = \frac{14}{6} = \frac{7}{3}
        \]
        \item
        \[
             m_X^{\prime\prime}(t)
           = \frac{1}{6}e^t + \frac{8}{6}e^{2t} + \frac{27}{6}e^{3t}
           \Rightarrow
           E[X^2] = m_X^{\prime\prime}(0) = \frac{36}{6} = 6
           \Rightarrow
           V[X] = E[X^2] - (E[X])^2
                = 6 - \frac{49}{9} = \frac{5}{9}
        \]
        \item
        Let $Y$ have the distribution,
        \[
            P(Y = 1) = \frac{1}{6} \quad
            P(Y = 2) = \frac{2}{6} \quad
            P(Y = 3) = \frac{3}{6}
        \]
        Then $m_X(t) = m_Y(t) \quad \forall t \in \mathbb{R}$, which implies
that $X$ has the same distribution as $Y$.
    \end{enumerate}
\end{solution}
\subsection{Markov's Inequality}
\begin{theorem}
    If $X$ is a real-valued nonnegative random variable, then
    \[ P(X \geq a) \leq \frac{E[X]}{a} \qquad a > 0                          \]
\end{theorem}
\begin{proof}
\begin{align*}
   &X = X \cdot 1_{\lbrace x \geq a \rbrace} + X \cdot 1_{\lbrace x < a\rbrace}
      \geq a \cdot 1_{\lbrace x \geq a \rbrace} + 0
      = a \cdot 1_{\lbrace x \geq a \rbrace}                                 \\
   &\Rightarrow \exists X \geq E[a \cdot 1_{\lbrace x \geq a \rbrace}]
                          = a P(X \geq a)                                    \\
   &\Rightarrow P(X \geq a) \leq \frac{E[X]}{a}
\end{align*}
\end{proof}

\paragraph{A simple consequence of Markov's inequality}
\begin{lemma}
If $X$ is a real-valued random variable and $E[\vert X \vert^\alpha]$ for some
$\alpha > 0$, then $\forall a > 0$,
\[
	P(\vert x \vert > a) = P(\vert x \vert^\alpha > a^\alpha)
	                     \leq \frac{E[\vert X \vert^\alpha]}{a^\alpha}
\]
by applying Markov's inequality to the random variable $\vert X
\vert^\alpha$.
\end{lemma}
\note These are simple examples of what is called `tail-bounds' for random
variables.
\subsection{Chebyshev's Inequality}
\begin{theorem}
If $X$ is.a real-valued random variable, and $E[X] = \mu$ and $V[x] =
\sigma^2$, then
\[
	P(\vert X - \mu \vert \geq x \cdot \sigma) \leq \frac{1}{x^2} 
	\qquad \forall x > 0
\]
\end{theorem}
\noindent \textbf{Interpretation}: The theorem simply says that $X$ is
concentrated around its mean with high probability, i.e. with probability at
least $1 - \frac{1}{x^2}$, $X$ falls in the interval $(\mu - x \sigma,
\mu + x \sigma)$.
\begin{proof}
\[
	  P(\vert X - \mu \vert) \geq x \sigma
	= P(\vert X - \mu \vert^2 \geq x^2 \sigma^2)
	\leq \frac{E[x - \mu]^2}{x^2 \sigma^2}
	= \frac{\sigma^2}{x^2 \sigma^2}
	= \frac{1}{x^2}
\]
\end{proof}

\subsection{Chernoff Bound}
\begin{theorem}
If $X$ is a real-valued random variable with moment generating function
$m_X(\cdot)$, then
\[
	P(X \geq x) \leq e^{-tx} \cdot m_X(t)
\]
for any $t > 0$, $x \in \mathbb{R}$. Consequently,
\[
	P(X \geq x) \leq \inf_{t > 0} \left[ e^{-tx} m_X(t) \right]
\]
\end{theorem}
\begin{proof}
For any $t>0$,
\[
	  P(X \geq x)
	= P(e^{tX} \geq e^{tx}) 
	\leq \frac{E[e^{tX}]}{e^{tx}}
	= e^{-tx} m_X(t)
\]
\end{proof}
\begin{example}
$X \sim \poissondist{\lambda}$, for any $c > 1$,
\[
	  P(X \geq c \lambda)
	= P(X - \lambda \geq (c - 1) \lambda)
	\leq \frac{E[(X - \lambda)^2]}{(c-1)^2 \lambda^2}
	= \frac{\lambda}{(c-1)^2 \lambda^2}
	= \frac{1}{(c-1)^2 \lambda}
\] 
Thus, Chebyshev's Inequality yields a simple polynomial tail bound. A stronger
bound can be obtained using Chernoff type arguments.
\end{example}

%% pending section
%% section3/pending/section3_sub7_problem_on_linearity_of_expectation.tex
\end{document}

%
% TODO:
%   - in 01-31-2017, some standard formulas & standard limits