\section{Cumulative Distribution Function}
\begin{definition}
For any real-valued random variable $Y$, the function $F_y : \mathbb{R}
\rightarrow [0, 1]$ defined by,
\[
	F_Y(y) = P(Y \leq y)
\]
is called the cumulative distribution function (or cdf, or distribution
function) of $Y$.
\end{definition}
\begin{example}
If $Y$ is a real-valued discrete random variable taking the values $y_1, y_2,
\dots$, then,
\[
    F_Y(y) = P(Y \leq y) 
           = \sum_{j : y_j \leq y} P(Y = y_j)
\]
\end{example}
\begin{example}
The pmf of $Y$ is given by,
\[
    p_Y(0) = \frac{1}{6} \qquad
    p_Y(1) = \frac{1}{3} \qquad
    p_Y(1.5) = \frac{1}{2}
\]
Find the cumulative distribution function of $Y$.
\end{example}
\begin{solution}
\[
    F_Y(y) = P(Y \leq y)
           = \sum_{y_j \leq y} P_Y(y_j)
\]
Hence,
\[
    F_Y(y) = \begin{cases}
        0                                              & y < 0          \\
        \frac{1}{6}                                    & 0 \leq y < 1   \\
        \frac{1}{6}+\frac{1}{3} =\frac{1}{2}           & 1 \leq y < 1.5 \\
        \frac{1}{6}+\frac{1}{3}+\frac{1}{2} = 1        & 1.5 \leq y
    \end{cases}
\]
\begin{figure}[H]
    \centering
    \includesvg[./section4/figure/]{sec3-fig1}
    \caption{Graph of $F_Y(\cdot)$}
\end{figure}
\end{solution}
\begin{example}
$Y \sim \geometricdist{p}$, find the cumulative distribution function of $Y$. 
\end{example}
\begin{solution}
\begin{align*}
    & P(Y = k) = p(1 - p)^{k - 1} \qquad k = 1, 2, \dots                     \\
    & \Rightarrow F_Y(y) = \sum_{k = 1}^{\lfloor y \rfloor} p (1 - p)^{k - 1}\\
    & \Rightarrow F_Y(y) = \begin{cases}
        0                  & y < 1                                           \\
        1 - (1 - p)^k      & k \leq y < k + 1, k = 1, 2, 3, \dots
    \end{cases}
\end{align*}
\end{solution}
\begin{theorem}
If $F : \mathbb{R} \rightarrow [0, 1]$ is the cumulative distribution function
of a real-valued random variable, then the following holds:
\begin{enumerate}[noitemsep, topsep=0em]
\item \label{ref-sec4-thm1-1}
    $F$ is non-decreasing on $\mathbb{R}$, i.e. if $y_1 \leq y_2$, then
$F(y_1) \leq F(y_2)$.
\item \label{ref-sec4-thm1-2}
    $F(\infty) \coloneqq \lim_{x \rightarrow \infty} F(x) = 1$.
\item \label{ref_sec4-thm1-3}
    $F(-\infty) \coloneqq \lim{x \rightarrow -\infty} F(x) = 0$.
\item \label{ref_sec4-thm1-4}
    $F$ is a right continuous function, i.e. $\forall y \in \mathbb{R}$ and
sequence $\lbrace y_n \rbrace$ such that $y_1 \geq y_2 \geq y_3 \geq
\dots$ and $\lim_{n \rightarrow \infty} y_n = y$, $\lim_{n \rightarrow \infty}
F(y_n) = F(y)$.
\end{enumerate}
\end{theorem}
\begin{proof} \quad                                                          \\
\begin{enumerate}[noitemsep, topsep=0em]
\item
    If $y_1 \leq y_2$, then $\lbrace y \leq y_1 \rbrace \subseteq \lbrace y
\leq y_2 \rbrace \Rightarrow P(Y \leq y_1) \leq P(Y \leq y_2) \Rightarrow
F(y_1) \leq F(y_2)$.
\item
    Take any sequence $x_n \uparrow \infty$. Let $A_0 = \lbrace Y \leq x_0
\rbrace$ and for all $n \geq 1$, let $A_n = \lbrace x_{n - 1} \leq Y \leq x_n
\rbrace$. Then,
\[
    \lbrace x \in \mathbb{R} \rbrace = \bigcup_{n = 0}^\infty A_n
\]
and the events $A_n$ are mutually disjoint. Hence,
\[
    1 = P(x \in \mathbb{R})
      = P(\bigcup_{n = 0}^\infty A_n)
      = \sum_{n = 0}^\infty P(A_n)
      = \lim_{n \rightarrow \infty} \left[ \sum_{k = 0}^n P(A_k) \right]
      = \lim_{n \rightarrow \infty} P(Y \leq x_n)
      = \lim_{n \rightarrow \infty} F(x_n)
\]
\item
    Note that, for any $y \in \mathbb{R}$, $1 - F(y) = P(Y > y)$. Consider any
decreasing sequence $y_n \downarrow -\infty$, and use arguments similar to the
ones used in \ref{ref-sec4-thm1-2} to show that, 
\[
                \lim_{n \rightarrow \infty} P(Y_1 > y_n) \ 1
    \Rightarrow 1 - F(y_n) \xrightarrow{n \rightarrow \infty} 1
    \Rightarrow F(y_n) \xrightarrow{n \rightarrow \infty} 0
\]
\item
    Fix $y \in \mathbb{R}$ and a sequence $y_n \downarrow y$. Let $A_0 =
\lbrace y_1 < Y \rbrace$, and for any $n \geq 1$, let $A_n = \lbrace y_{n + 1}
< Y \leq y_n \rbrace$. Then $A_0, A_1, A_2, \dots$ are mutually disjoint and,
\begin{align*}
    &\bigcup_{n = 0}^\infty A_n = \lbrace y < Y \rbrace                      \\
    \Rightarrow &
    P(y < Y) = P(\bigcup_{n = 0}^\infty A_n)
             = \sum_{k = 0}^\infty P(A_k)
             = \lim_{n \rightarrow \infty} \left[ \sum_{k = 0}^n P(A_k) \right]
             = \lim_{n \rightarrow \infty} P(Y > y_{n + 1})                  \\
    \Rightarrow &
    1 - F(y) = \lim_{n \rightarrow \infty} [1 - F(y_{n + 1})]                \\
    \Rightarrow &
    F(y) = \lim_{n \rightarrow \infty} F(y_n)
\end{align*}
\end{enumerate}
\end{proof}

\begin{theorem}[without proof]
If $F : \mathbb{R} \rightarrow [0, 1]$ satisfies \ref{ref-sec4-thm1-1},
\ref{ref-sec4-thm1-2}, \ref{ref_sec4-thm1-3}, and \ref{ref_sec4-thm1-4} of
previous theorem, then there exists a real-valued random variable $X$ on some
sample space such that $F$ is the cumulative distribution function of $X$.
\end{theorem}

\begin{theorem}
For $x \in \mathbb{R}$, let 
\[
    F_X(x-) \coloneqq \lim_{y \uparrow x} F_X(y)
\]
Then $F_X(x-) = P(X < x)$. In particular, $P(X = x) = F_X(x) - F_X(x-)$.
\begin{proof}
Imitate the arguments used in the argument earlier.
\end{proof}
\end{theorem}

The cumulative distribution function of a discrete real-valued random variable
has jumps.

\begin{definition}
A real-valued random variable $Y$ with cumulative distribution function
$F_Y(\cdot)$ is said to be a continuous random variable if $F_Y$ is a
continuous function on $\mathbb{R}$.
\end{definition}

The cumulative distribution function of a continuous real-valued random
variable $Y$ does not have any jumps. In particular, $\forall y \in
\mathbb{R}$,
\[
    P(Y = y) = P(Y \leq y) - P(Y \leq y)
             = F_Y(y) - F_Y(y-)
             = 0
\]
\begin{definition}
A function $f : \mathbb{R} \rightarrow [0, \infty)$ is called a density
function if,
\[
    \int_{-\infty}^\infty f(x) dx = 1
\]
\end{definition}
\begin{definition}
A real-valued continuous random variable $X$ has density function $f_X(\cdot)$
if $\forall x \in \mathbb{R}$, 
\[
    F_X(x) = P(X \leq x) = \int_{-\infty}^x f_x(u) du
\]
\end{definition}
\note From the definition, one can see that $F_X^\prime(x) = f_X(x)$ at every
$x$ that is a continuity point of $f_X$.

If a continuous random variable $X$ has a density then it can be computed as
follows. At every $x \in \mathbb{R}$ where the cumulative distribution
function $F_X$ is differentiable, set
\[
    f_x(x) \coloneqq F_X^\prime(x) = 0
\]
At every other $x$, set $f_X(x) = 0$.
\begin{example} \quad                                                        \\
\[
    F_X(x) = \begin{cases}
        0               & x < 0                                              \\
        1 - e^{-x}      & x \geq 0
    \end{cases}
\]
is a continuous cumulative distribution function. The corresponding density is,
\[
    f_X(x) = \begin{cases}
        0               & x \leq 0                                           \\
        e^{-x}          & x > 0
    \end{cases}
\]
\end{example}

This is analogous to the case of discrete random variables. If $X$ is a
real-valued discrete random variable with cumulative distribution function
$F_X$, then its pmf can be computed as follows. At every $x \in \mathbb{R}$
where $F_X$ has a jump, set
\[
    p_X(x) = F_X(x) - F_X(x-)
\]
and define $p_X(x) = 0$ at all other $x \in \mathbb{R}$.

\begin{theorem}
Assume that $X$ is a real-valued random variable with density $f$. Then for any
$a, b \in \mathbb{R}$ with $a < b$, 
\[
    P(X \in (a, b]) = P(X \in [a, b]) = P(X \in (a, b))
                    = F(b) - F(a)
                    = \int_{a}^b f(u) du
\]
\begin{proof}
\[
    P(X = a) = 0 = P(X = b)
    \Rightarrow
    P(X \in (a, b]) = P(X = a) + P(X \in (a, b]) = P(X \in [a, b])
\]
First four equalities follow similarly,
\[
    P(X \in (a, b]) = P(X \leq b) - P(X \leq a)
                    = F_X(b) - F_X(a)
                    = \int_{-\infty}^b f(u) du - \int_{-\infty}^a f(u) du 
                    = \int_{a}^b f(u) du
\]
\end{proof}
\end{theorem}

\subsection{Distribution of a Function of an Random Variable}
If $X$ is a real-valued random variable having a density function and $g :
\mathbb{R} \rightarrow \mathbb{R}$, then the cumulative distribution function
of $g(x)$ can be computed in a straightforward manner. Let $Y = g(X)$, then the
cumulative distribution function of $Y$ is, 
\[
    F_Y(y) = P(g(X) \leq y) 
           = P(X \in g^{-1}(-\infty, y])
\]
The density of $Y$, if it exists, can be computed just by differentiating
$F_Y$.

\begin{example}
The density of $X$ is given by,
\[
    f(x) = \begin{cases}
        c \cdot \vert x \vert       & x \in (-1, 1)                          \\
        0                           & \text{otherwise}
    \end{cases}
\]
\begin{enumerate}[noitemsep, topsep=0em]
\item Find the value of $c$.
\item Find the cumulative distribution function and the density of $Y = X^2$.
\end{enumerate}
\end{example}
\begin{solution} \quad                                                       \\
\begin{enumerate}[noitemsep, topsep=0em]
\item
\[
    \int_{-\infty}^\infty f(x) \quad dx = 1
    \Rightarrow \int_{-1}^1 c \vert x \vert \quad dx = 1
    \Rightarrow 2c \cdot \int_{0}^1 x \quad dx = 1
    \Rightarrow c = 1
\] 
\item
For $y < 0$
\[
    P(Y \leq y) = P(X^2 \leq y) = P(\emptyset) = 0
\]
And for $1 \geq y \geq 0$, then
\[
    P(X^2 \leq y) = P(-\sqrt{y} \leq X \leq \sqrt{y})
                  = \int_{-\sqrt{y}}^{\sqrt{y}} f(u) \quad du
                  = \int_{-\sqrt{y}}^{\sqrt{y}} \vert u \vert \quad du
                  = 2 \int_{0}^{\sqrt{y}} u \quad du
                  = y
\]
If $y > 1$, then
\[
    P(X^2 \leq y) = \int_{-\sqrt{y}}^{\sqrt{y}} f(u) \quad du
                  = \int_{-1}^{1} \vert u \vert \quad du +
                    \int_{1}^{\sqrt{y}} 0 \quad du +
                    \int_{-\sqrt{y}}^{-1} 0 \quad du
                  = 1
\]
Hence the cumulative distribution function of $Y$ is given by,
\[
    F_Y(y) = \begin{cases}
        0         & y < 0                                                    \\
        u         & 0 \leq y \leq 1                                          \\
        1         & y < 1         
    \end{cases}
\]
The density of $Y$ is given by,
\[
    f_Y(y) = \begin{cases}
        0         & y \leq 0                                                 \\
        1         & 0 < y < 1                                                \\
        0         & 1 \leq y
    \end{cases}
\]
\end{enumerate}
\end{solution}
\subsection{Expectation}
\begin{definition}
If $X$ is real-valued random variable with a density $f$, and either 
\[
    \int_0^\infty f(x) \quad dx < \infty
    \qquad \text{or} \qquad
    \int_{-\infty}^0 \vert x \vert f(x) \quad dx < \infty
\]
then the expectation of $X$ is defined as,
\[
    E[X] = \int_{-\infty}^{\infty} x f(x) \quad dx
\]
If both $\int_{0}^{\infty} x f(x) \quad dx$ and $\int_{-\infty}^0 \vert x
\vert f(x) \quad dx$ are infinite, then we say that $E[X]$ does not exist.
\end{definition}
\note the similarity with discrete random variables essentially we are
replacing pmf by density and sum by integral.

\begin{definition}
For any real-valued random variable $X$ with cumulative distribution function
$F_X$, then expectation of $X$ is given by,
\[
    E[X] = \int_{-\infty}^{\infty} x \quad dF_X(x)
\]
where the integral is interpreted in the Riemann-Stieltjes sense. 
\end{definition}

\begin{theorem}[without proof]
Let $X_1$, $X_2$, $\dots$, $X_k$ be continuous random variables defined on the
same sample space having densities $f_1$, $f_2$, $\dots$, $f_k$ respectively.
\begin{enumerate}[noitemsep, topsep=0em]
\item
    Let $c \in \mathbb{R}$ and $g : \mathbb{R} \rightarrow \mathbb{R}$. Then, 
\[
    E[c \cdot g(X_1)] = c \cdot E[g(X_1)] 
                      = c \cdot \int_{-\infty}^{\infty} g(u)f_1(u) \quad du
\]
\item
\[
    E[X_1 + \dots + X_k] = E[X_1] + \dots + E[X_k]
                         = \int_{-\infty}^{\infty} x f_1(x) \quad dx +
                           \dots +
                           \int_{-\infty}^{\infty} x f_k(x) \quad dx
\]
\end{enumerate}
\end{theorem}
\note Even if $X_1$, $X_2$ are continuous random variables with densities,
$g(X_1)$ or $X_1 + X_2$ may not be continuous random variables. In that case, 
$E[g(X_1)]$ or $E[X_1 + X_2]$ is defined through the Riemann-Stielljes
integral. But the above theorem says that the expected values computed using
that definition coincides with,
\[
    \int_{-\infty}^{\infty} g(x) f(x) \quad dx
    \qquad \text{and} \qquad
    \int_{-\infty}^{\infty} x f_1(x) \quad dx + 
    \int_{-\infty}^{\infty} x f_2(x) \quad dx
\]
respectively.

\begin{definition}
If $X$ has density $f$,  then the variance of $X$ is given by, 
\[
    V[X] = E[(x - \mu)^2]
         = \int_{-\infty}^{\infty} (x - \mu)^2 f(x) \quad dx
\]
where $\mu = E[X]$. Consequently,
\[
    V[X] = E[X^2] - \mu^2
         = \int_{-\infty}^{\infty} x^2 f(x) \quad dx -
           \left( \int_{-\infty}^{\infty} x f(x) \quad dx \right)^2
\]
\end{definition}

\begin{example} \quad                                                        \\
\[
    F_X(x) = \begin{cases}
        0                    & x \leq 0                                      \\
        \frac{x}{8}          & 0 < x < 2                                     \\
        \frac{x^2}{16}       & 2 \leq 4 < 4                                  \\
        1                    & 4 \leq x 
    \end{cases}
\]
\begin{enumerate}[noitemsep, topsep=0em]
\item Find $P(1 \leq X \leq 3)$.
\item Find the density of $X$.
\item Find $E[X]$, $V[X]$.
\end{enumerate}
\end{example}
\begin{solution} \quad \linebreak                                
\begin{enumerate}[noitemsep, topsep=0em]
\item
\[
    P(1 \leq X \leq 3) = F_X(3) - F_X(1)
                       = \frac{9}{16} - \frac{1}{8}
                       = \frac{7}{16}
\]
\item
Density $f_x(\cdot)$ is given by,
\[
    f_X(x) = \begin{cases}
        0                  & x \leq 0                                       \\
        \frac{1}{8}        & 0 < x < 2                                      \\
        0                  & x = 2                                          \\
        \frac{x}{8}        & 2 < x < 5                                      \\
        0                  & 4 \leq x
    \end{cases}
\]
\end{enumerate}
\begin{figure}[H]
    \centering
    \begin{multicols}{2}
    \def\svgwidth{\linewidth}
    \includesvg[./section4/figure/]{sec3-fig2}
    \columnbreak
    \def\svgwidth{\linewidth}
    \includesvg[./section4/figure/]{sec3-fig3}
    \end{multicols}
\end{figure}
\note If instead of $F_X$, you were given $f_X$, then you could compute $P(1
\leq X \leq 3)$ using $\int_{1}^{3} f_x(u) \quad du$.
\item
\[
    E[X] = \int_{-\infty}^{\infty} x f(x) \quad dx
         = \int_{0}^2 \frac{x}{8} \quad dx + 
           \int_{2}^4 \frac{x^2}{8} \quad dx
      %% = TODO: finish computation
\]
\[
    E[X^2] = \int_{-\infty}^{\infty} x^2 f(x) \quad dx
           = \int_{0}^{2} \frac{x^2}{8} \quad dx +
             \int_{2}^{4} \frac{x^3}{8} \quad dx
        %% = TODO: finish computation
\]
\[
    V[X] = E[X^2] - (E[X])^2 %% = TODO: finish computation
\]
\end{solution}

\begin{example}
Suppose $X$ is a continuous random variable with density $f(\cdot)$ and
$E[\vert X \vert] < \infty$. Show that, 
\[
    \lim_{a \rightarrow \infty} a P(\vert X \vert \geq a) = 0
\]
\end{example}
\begin{solution}
\begin{equation} \label{ref-sec4-thm2-1}
    E[\vert X \vert] = \int_{-\infty}^{\infty} \vert x \vert f(x) \quad dx
                     < \infty
    \Rightarrow
    \lim_{a \rightarrow \infty} \left[
        \int_{a}^{\infty} x f(x) \quad dx +
        \int_{-\infty}^{-a} \vert x \vert f(x) \quad dx
    \right] = 0
\end{equation}
Now,
\[
    \int_{a}^{\infty} f(x) \quad dx \geq a \int_{a}^{\infty} f(x) \quad dx
                                    =    a P(X \geq a)
\]
Similarly,
\[
         \int_{-\infty}^{-a} \vert x \vert f(x) \quad dx 
    \geq a \int_{-\infty}^{-a} f(x) dx
    =    a P(x \leq -a)
\]
Hence \ref{ref-sec4-thm2-1} implies that,
\[
    \lim_{a \rightarrow \infty} [
        a P(X \geq a) +
        a P(X \leq -a)
    ] = 0
    \Rightarrow
    \lim{a \rightarrow \infty} a P(\vert X \vert \geq a) = 0
\]
\end{solution}


\subsection{Uniform Distribution}
\begin{definition}
Let $\theta_1 \leq \theta_2$. An random variable $X$ is said to have a uniform
distribution on the interval $[\theta_1, \theta_2]$, or simply $X \sim
\uniformdist{\theta_1}{\theta_2}$, if and only if the function $f$ given by,
\[
    f(x) = \begin{cases}
        \frac{1}{\theta_2 - \theta_1}      & x \in [\theta_1, \theta_2].     \\
        0                                  & \text{otherwise}
    \end{cases}
\]
is a density for $X$. In particular, the cumulative distribution function of
$X$ is given by,
\[
    F_X(x) = \begin{cases}
        0                                        & x < \theta_1              \\
        \frac{x - \theta_1}{\theta_2 - \theta_1} & \theta_1 \leq x < \theta_2\\
        1                                        & \theta_2 \leq x
    \end{cases}
\]
\begin{figure}[H]
    \centering
    \def\svgwidth{0.5\textwidth}
    \includesvg[./section4/figure/]{sec3-fig4}    
\end{figure}
\end{definition}

\begin{theorem} \quad                                                        \\
\begin{enumerate}[noitemsep, topsep=0em]
\item
    If $X \sim \uniformdist{0}{1}$, then $\forall a \in \mathbb{R}, b > 0$,
\[ 
    Y = a + bX \sim \uniformdist{a}{a + b}
\]
\item
    If $X \sim \uniformdist{\theta_1}{\theta_2}$ and $\theta_1 < \theta_2$,
then $\forall \theta_3, \theta_4 \in [\theta_1, \theta_2]$ with $\theta_3 <
\theta_4$, the random variable $Z$ whose distribution is given by,
\[
    P(Z \leq \mathbb{Z}) \coloneqq P(X \leq \mathbb{Z} \vert 
                                     X \in [\theta_3, \theta_4])
\]
is uniformly distributed over $[\theta_3, \theta_4]$, i.e., $Z \sim
\uniformdist{\theta_3}{\theta_4}$.
\end{enumerate}
\end{theorem}
\begin{proof} \quad                                                          \\
\begin{enumerate}[noitemsep, topsep=0em]
\item
\begin{align*}
    P(Y \leq y) &= P(a + bX \leq y)
                 = P(X \leq \frac{y - a}{b})
                 = F_X \left( \frac{y - a}{b} \right)                        \\
    &\Rightarrow F_Y(y) = \begin{cases}
        0                   & y < a                                          \\
        \frac{y - a}{b}     & a \leq y < a + b                               \\
        1                   & b \leq y
    \end{cases}                                                              \\
    &\Rightarrow Y \sim \uniformdist{a}{a + b}
\end{align*}   
\item
\[
    P(Z \leq \mathbb{Z}) = \frac
        {P(X \in (-\infty, \mathbb{Z}] \cap [\theta_3, \theta_4]))}
        {P(X \in [\theta_3, \theta_4])}
\]
Now,
\[
    (-\infty, \mathbb{Z}] \cap [\theta_3, \theta_4] = \begin{cases}
        \emptyset               & \mathbb{Z} < \theta_3                      \\
        [\theta_3, \mathbb{Z}]  & \theta_3 \leq \mathbb{Z} < \theta_4        \\
        [\theta_3, \theta_4]    & \theta_4 \leq \mathbb{Z}
    \end{cases}
\]
and, $P(X \in [\theta_3, \theta_4]) = \frac{\theta_4 - \theta_3}{\theta_2 -
\theta_1}$. Hence,
\[
    P(Z \leq \mathbb{Z}) = \begin{cases}
        0                       & \mathbb{Z} < \theta_3                      \\
        \frac{\mathbb{Z}-\theta_3}{\theta_4 - \theta_3} 
            & \theta_3 \leq \mathbb{Z} < \theta_4                            \\
        1                       & \theta_4 \leq \mathbb{Z}
    \end{cases}
    \Rightarrow
    Z \sim \uniformdist{\theta_3}{\theta_4}
\]
\end{enumerate}
\end{proof}
\subsection{Normal Distribution / Gaussian Distribution}
Normal distribution `approximates' many other probability distributions. This
is a classical result called the central limit theorem (CLT), which we will
discuss later.
\begin{definition}
Suppose $\mu \in \mathbb{R}$ and $\sigma > 0$. A random variable $X$ follows a
`normal distribution' with parameters $\mu$ and $\sigma^2$, or in short $X
\sim \normaldist{\mu}{\sigma^2}$, if and only if,
\[
    f_{\mu, \sigma^2}(x) = \frac{1}{\sigma \sqrt{2\pi}} 
        \exp \left( -\frac{(x - \mu)^2}{2\sigma^2} \right)
        \qquad x \in \mathbb{R} 
\]
is a density for $X$.
\end{definition}

\begin{theorem} \quad                                                        \\
\[
    \int_{0}^{\infty} e^{-\frac{x^2}{2}} \quad dx = \sqrt{\frac{\pi}{2}}
\]
In particular,
\[
    \int_{-\infty}^{\infty} f_{\mu, \sigma^2}(x) \quad dx = 1
\]
\end{theorem}
\begin{proof}
Let $I = \int_{0}^{\infty} e^{-\frac{x^2}{2}} \quad dx$. Then,
\[
    I^2 = (\int_{0}^{\infty} e^{-\frac{x^2}{2}} \quad dx) \cdot
          (\int_{0}^{\infty} e^{-\frac{x^2}{2}} \quad dx)
        = \int_{0}^{\infty} \int_{0}^{\infty}
              e^{-\frac{x^2 + y^2}{2}}
          \quad dxdy
\]
Switch to polar coordinates,
\[
    x = r \cdot \cos{\theta} \qquad y = r \cdot \sin{\theta}
\]
The Jacobian of this transformation is
\[
    \mathcal{J}(r, \theta) = \det \left\vert \begin{array}{cc}
          \frac{\delta}{\delta r} r \cdot \cos{\theta}
        & \frac{\delta}{\delta \theta} r \cdot \cos{\theta}                  \\
          \frac{\delta}{\delta r} r \cdot \sin{\theta}
        & \frac{\delta}{\delta \theta} r \cdot \sin{\theta}
    \end{array} \right\vert
    = \det \left\vert \begin{array}{cc}
          \cos{\theta}         & -r \cdot \sin{\theta}                       \\
          \sin{\theta}         &  r \cdot \cos{\theta}
    \end{array} \right\vert
    = r \cdot \cos^2{\theta} + r \cdot \sin^2{\theta}
    = r
\]
Hence,
\begin{align*}
    I^2 &= \int_{r = 0}^{\infty} \int_{\theta = 0}^{\frac{\pi}{2}} 
           \exp \left(
               -\frac{r^2 \cos^2{\theta} + r^2 \sin^2{\theta}}{2}
           \right)
           \vert \mathcal{J}(r, \theta) \vert \quad drd\theta                \\
        &= \int_{r = 0}^{\infty} \int_{\theta = 0}^{\frac{\pi}{2}}
               \exp \left( -\frac{r^2}{2} \right) \cdot r 
           \quad drd\theta                                                   \\
        &= \left(
               \int_{0}^{\infty} e^{-\frac{r^2}{2}} r \quad dr
           \right) \cdot
           \left(
               \int_{0}^{\frac{\pi}{2}} 1 \quad d\theta                 
           \right)                                                           \\
        &= \left(
               \int_{0}^{\infty} e^{-u} \quad du
           \right)
           \cdot \frac{\pi}{2}
         = \frac{\pi}{2}
\end{align*}
And we have,
\[
    I = \sqrt{\frac{\pi}{2}}
\]
To evaluate $\int_{\infty}^{\infty} f_{\mu, \sigma^2}(x) \quad dx$, subsitute
$y = \frac{x - \mu}{\sigma}$. Then,
\begin{align*}
    \int_{-\infty}^{\infty} 
         \frac{1}{\sigma \sqrt{2\pi}}
         \exp \left( - \frac{(x - \mu)^2}{2 \sigma^2} \right)
    \quad du
    &= \int_{-\infty}^{\infty}
           \frac{1}{\sigma \sqrt{2\pi}}
           \exp \left( - \frac{y^2}{2} \right) \sigma
        \quad dy                                                             \\
    &= \frac{1}{\sqrt{2 \pi}}
       \int_{-\infty}^{\infty}
           \exp \left( - \frac{y^2}{2} \right) 
       \quad dy                                                              \\
    &= \frac{1}{\sqrt{2 \pi}} \cdot 2 \cdot
       \int_{0}^{\infty} e^{-\frac{y^2}{2}} \quad dy                         \\
    &= \frac{1}{\sqrt{2 \pi}} \cdot 2 \cdot \sqrt{\frac{\pi}{2}}        
     = 1
\end{align*}
\end{proof}

\begin{definition}
The function,
\[
    \phi(x) = \frac{1}{\sqrt{2 \pi}} e^{-\frac{x^2}{2}} \qquad x \in \mathbb{R}
\]
is called the standard normal density. $\phi$ is the density of a random
variable that follows a $\normaldist{0}{1}$ distribution. Its cumulative
distribution function is given by,
\[
    \Phi(x) = \int_{-\infty}^{\infty} \phi(u) \quad du
\]
There is no closed form expression for $\Phi(\cdot)$. Only approximate values
of $\Phi(x)$ can be calculated.
\begin{figure}[H]
    \centering
    \def\svgwidth{0.5\linewidth}
    \includesvg[./section4/figure/]{sec3-fig5}    
\end{figure}
\end{definition}

\begin{theorem}
If $X \sim \normaldist{0}{1}$, then $\mu + \sigma X \sim
\normaldist{\mu}{\sigma^2}$ for all $\mu \in \mathbb{R}$, $\sigma > 0$.
Conversely, if $Y \sim \normaldist{\mu}{\sigma^2}$, then,
\[
    \frac{y - \mu}{\sigma} \sim \normaldist{0}{1}
\]
\end{theorem}
\begin{proof}\quad                                                           \\
\begin{align*}
      P(\mu + \sigma X \leq y) = P(X \leq \frac{y - \mu}{\sigma})
   &= \int_{-\infty}^{\frac{y - \mu}{\sigma}}
          \frac{1}{\sqrt{2\pi}} e^{\frac{x^2}{2}}
      \quad dx                                                               \\
   &= \int_{-\infty}^{y} 
          \frac{1}{\sigma \sqrt{2 \pi}} e^{-\frac{(t - \mu)^2}{2\sigma^2}}
      \quad dt
      \qquad \left( x = \frac{t - u}{\sigma} \right)                         \\
   &= \int_{-\infty}^{y}
          f_{\mu, \sigma^2}(t)
      \quad dt                                                               \\
   &\Rightarrow \mu + \sigma X \sim \normaldist{\mu}{\sigma^2}
\end{align*}
The other part follows similarly.
\end{proof}

\begin{theorem}
If $X \sim \normaldist{\mu}{\sigma^2}$, then
\begin{enumerate}[noitemsep, topsep=0em]
\item
\[
    E[X] = \mu
\]
\item
\[
    V[X] = \sigma^2
\]
\item
\[
    m_X(t) = \exp \left( \mu t + \sigma^2 \frac{t^2}{2} \right)
\]    
\end{enumerate}
\end{theorem}
\begin{proof}
Assume that $Y \sim \normaldist{0}{1}$. Then,
\[
    X \overset{\text{d}}{=} \mu + \sigma Y
    \Rightarrow \begin{cases}
        E[X] = \mu + \sigma E[Y]                                           \\
        V[X] = \sigma^2 V[Y]                                               \\
        m_X(t) = E [\exp(t(\mu + \sigma Y))] = E[e^{tu} \cdot e^{t \sigma Y}]
               = e^{t\mu} E[\exp(t \sigma Y)]
               = e^{t\mu} m_Y(t \sigma)
    \end{cases}
\]
Now,
\begin{align*}
    E[X] &= \int_{-\infty}^{\infty} \frac{y}{\sqrt{2 \pi}} e^{\frac{-y^2}{2}}
            \quad dy                                                       \\
         &= \int_{-\infty}^{0} \frac{y}{\sqrt{2 \pi}} e^{\frac{-y^2}{2}} \quad
            dy +
            \int_{0}^{\infty} \frac{y}{\sqrt{2 \pi}} e^{\frac{-y^2}{2}} \quad
            dy                                                             \\
         &= -\int_{0}^{\infty} \frac{u}{\sqrt{2 \pi}} e^{\frac{-u^2}{2}}
            \quad du +
            \int_{0}^{\infty} \frac{y}{\sqrt{2 \pi}} e^{\frac{-y^2}{2}} \quad
            dy                                                             \\
         &= 0                                                              \\
    \Rightarrow E[X] &= \mu
\end{align*}
\begin{align*}
    m_Y(t) &= \int_{-\infty}^{\infty} e^{ty} \frac{1}{\sqrt{2\pi}}
              e^{-\frac{y^2}{2}} \quad dy                                  \\
           &= \int_{-\infty}^{\infty} \frac{1}{\sqrt{2\pi}}
              \exp \left(
                  -\frac{y^2 - 2ty}{2}
              \right) \quad dy                                             \\
           &= e^{\frac{t^2}{2}}
              \int_{-\infty}^{\infty} \frac{1}{\sqrt{2\pi}}
              \exp \left(
                  -\frac{y^2 - 2ty + t^2}{2}
              \right) \quad dy                                             \\
           &= e^{\frac{t^2}{2}}
              \int_{-\infty}^{\infty} \frac{1}{\sqrt{2\pi}}
              \exp \left(
                  -\frac{(y - t)^2}{2}
              \right) \quad dy                                             \\
           &= e^{\frac{t^2}{2}}                                            \\
    \Rightarrow m_X(t) &= \exp (\mu t + \sigma^2 \frac{t^2}{2})
\end{align*}
\[
    \begin{cases}
        m_Y^\prime(t) = t e^{\frac{t^2}{2}}                                \\
        m_Y^{\prime\prime}(t) = t^2 e^{\frac{t^2}{2}} + e^{\frac{t^2}{2}}  \\
    \end{cases}
    \Rightarrow m_Y^{\prime\prime}(0) = 1
    \Rightarrow E[Y^2] = 1
    \Rightarrow V[Y] = E[Y^2] - (E[Y])^2 = 1 - 0 = 1
    \Rightarrow V[X] = \sigma^2
\]
\end{proof}

\begin{definition}
If $Y$ is a real-valued random variable with cumulative distribution function
$F$ and $p \in (0, 1)$, then the $p$-th quantile of $Y$, denoted by
$\mathcal{Q}_p$ is given by,
\[
    \mathcal{Q}_p \coloneqq \min \lbrace y : F(y) \geq p \rbrace
\]
Thus, $F(\mathcal{Q}_p) \geq p$. If $F$ is continuous (i.e., if $Y$ is a
continuous random variable), then $F(\mathcal{Q}_p) = p$.
\end{definition}
\note The quantiles for standard normal distribution are available from
standard tables (or using any software).

\begin{example}
A soft-drink machine can be regulated so that it discharges an average of
$\mu$ ounces per cup. The ounces of fill are normally distributed with standard
distribution of $0.3$ ounce. Give the setting for $\mu$ so that $8$oz cups will
overflow only $1\%$ of the time.
\end{example}
\begin{solution}
Let $\mu_0$ be the required value. Let $X$ denote the amount discharged. Then
$X \sim \normaldist{\mu_0}{(0.3)^2}$. 
\begin{align*}
    P(X \geq 8) = 0.01 \Rightarrow & P(X \leq 8) = 0.99                    \\
    \Rightarrow & P \left(
        \frac{x - \mu_0}{0.3} \leq \frac{8 - \mu_0}{0.3}
    \right) = 0.99                                                         \\
    \Rightarrow & \Phi \left( \frac{8 - \mu_0}{0.3} \right) = 0.99 
        \qquad \text{since } \frac{X - \mu_0}{0.3} \sim \normaldist{0}{1}  \\
    \Rightarrow & \frac{8 - \mu_0}{0.3} = \mathcal{Q}_{0.99}               \\
    \Rightarrow & \mu_0 = 8 - 0.3 \mathcal{Q}_{0.99}
\end{align*}
\end{solution}

\begin{example}
Scores on an examination are assumed to be normally distributed with mean $78$
and variance $36$. What is the probability that a person taking the exam score
higher than $72$?
\end{example}
\begin{solution}
Let $X$ denote his/her score. Then $X \sim \normaldist{78}{36}$.
\begin{align*}
    P(X \geq 72) &= P \left(
                        \frac{X - 78}{6} \geq \frac{72 - 78}{6}
                    \right)                                                \\
                 &= 1 - \Phi{-1} \qquad 
                    \text{since } \frac{X - 78}{6} \sim \normaldist{0}{1}  \\
                 &= \Phi(1) 
\end{align*}
\end{solution}

\begin{example}
Show that $1 - \Phi(-x) = \Phi(x)$.
\end{example}
%% TODO

\subsection{Gamma Distribution}
\begin{definition}
The Gamma function $\Gamma : (0, \infty) \rightarrow \mathbb{R}$ is defined as,
\[
    \Gamma(\alpha) = \int_{0}^{\infty} x^{\alpha - 1} e^{-x} \quad dx
\]
\end{definition}
\note It is easy to show that this integral is finite when $\alpha > 0$.

\begin{theorem} \quad                                                        \\
\begin{enumerate}[noitemsep, topsep=0em]
\item \[ \forall \alpha \geq 0, \Gamma(\alpha + 1) = \alpha \Gamma(\alpha) \]
\item \[ \Gamma(n + 1) = n!, n = 1, 2, 3, \dots                            \]
\item \[ \Gamma(\frac{1}{2}) = \sqrt{\pi}                                  \]
\end{enumerate}
\end{theorem}
\begin{proof} \quad                                                          \\
\begin{enumerate}[noitemsep, topsep=0em]
\item Using integration by parts,
\[
      \Gamma(\alpha + 1) = \int_0^{\infty} x^\alpha e^{-x} \quad dx
    = \left. x^\alpha e^{-x} \right\vert_{\infty}^0 +
      \int_0^{\infty} \alpha x^{\alpha - 1} e^{-x} \quad dx
    = 0 + \alpha \Gamma(\alpha)
    = \alpha \Gamma(\alpha)
\]
\item
\[
    \Gamma(1) = \int_{0}^{\infty} e^{-x} \quad dx = 1
\]
\[
    \Gamma(n + 1) = n \Gamma(n)
                  = n (n - 1) \Gamma(n - 1)
                  = \dots
                  = n \times (n - 1) \times \dots \times 3 \times 2 \times 1
                  = n!
\]
\item Substitute $x = \frac{u^2}{2}$, 
\[
    \Gamma(\frac{1}{2}) = \int_{0}^{\infty} \frac{1}{\sqrt{x}} e^{-x} \quad dx
                        = \int_{0}^{\infty} \sqrt{2} e^{-\frac{u^2}{2}} \quad
                          du
                        = \sqrt{2} \cdot \sqrt{\frac{\pi}{2}}
                        = \sqrt{\pi}
\]
\end{enumerate}
\end{proof}

\begin{definition}
A real-valued random variable $Y$ is said to have a gamma distribution with
shake parameter $\alpha > 0$ and scale parameter $\beta > 0$, if,
\[
    f(y) = \begin{cases}
        \frac{y^{\alpha - 1} e^{-\frac{y}{\beta}}}{\beta^\alpha \Gamma(\alpha)}
            & y \geq 0                                                       \\
        0
            & y < 0                                     
    \end{cases}
\]
is a density for $Y$.
\end{definition}
This is indeed a probability distribution function, as (take $u =
\frac{y}{\beta}$)
\[
      \int_{-\infty}^{\infty} f(y) \quad dy
    = \int_{0}^{\infty} 
          \frac{y^{\alpha - 1}e^{-\frac{y}{\beta}}}{\beta^\alpha\Gamma(\alpha)}
      \quad dy
    = \int_{0}^{\infty} 
          \frac
          {\beta^{\alpha - 1} u^{\alpha - 1} e^{-u} \beta}
          {\beta^{\alpha} \Gamma(\alpha)}
      \quad du
    = \int_{0}^{\infty}
          \frac{u^{\alpha - 1}e^{-u}}{\Gamma(\alpha)}
      \quad du
    = 1
\]
There is no closed form expression for the cumulative distribution function,
\[
    \int_{0}^{x}
        \frac{y^{\alpha - 1}e^{-\frac{y}{\beta}}}{\beta^\alpha \Gamma(\alpha)}
    \quad dy
\]
\begin{theorem}
If $Y \sim \gammadist{\alpha}{\beta}$, then,
\begin{enumerate}[noitemsep, topsep=0em]
\item \[
    E[Y] = \alpha \beta
\]
\item \[
    V[Y] = \alpha \beta^2
\]
\item For $t < \frac{1}{\beta}$,
\[
    m_Y(t) = (1 - \beta t)^{-\alpha}
\]
For $t \geq \frac{1}{\beta}$, $E[e^{ty}] = \infty$.
\end{enumerate}
\end{theorem}
\begin{proof}
Let $X = \frac{Y}{\beta}$. Then $X \sim \Gamma(\alpha, 1)$, and $E[Y] =
\beta E[X]$, $V(Y) = \beta^2 V[X]$ and $m_Y(t) = m_X(t\beta)$.
\begin{enumerate}[noitemsep, topsep=0em]
\item
\[
    E[X] = \int_{0}^{\infty}
               x \cdot \frac{x^{\alpha - 1}e^{-x}}{\Gamma(\alpha)}
           \quad dx
         = \int_{0}^{\infty}
               \frac{x^\alpha e^{-x}}{\Gamma(\alpha)}
           \quad dx
         = \frac{\Gamma(\alpha + 1)}{\Gamma(\alpha)}
         = \alpha
\]
\item
\[
    E[X^2] = \int_{0}^{\infty} \frac
                 {x^2 x^{\alpha - 1} e^{-x}}
                 {\Gamma(\alpha)}
             \quad dx
           = \int_{0}^{\infty} \frac
                 {x^{\alpha + 1} e^{-x}}
                 {\Gamma(\alpha)}
             \quad dx
           = \frac{\Gamma(\alpha + 1)}{\Gamma(\alpha)}
           = (\alpha + 1) \alpha
\]
\[
    \Rightarrow
    V[X] = \alpha(\alpha + 1) - (E[X])^2
         = \alpha(\alpha + 1) - \alpha^2
         = \alpha
\]
\item
\[
    m_X(t) = \int_{0}^{\infty}
                 e^{tx} \cdot \frac
                 {x^{\alpha - 1} e^{-x}}
                 {\Gamma(\alpha)}
             \quad dx
           = \int_{0}^{\infty} \frac 
                 {x^{\alpha - 1} e^{-(1 - t)x}}
                 {\Gamma(\alpha)}
             \quad dx
\]
This integral is finite only when $t < 1$. In that case,
\[
    m_X(t) = \int_{0}^{\infty} \frac 
                 {x^{\alpha - 1} e^{-(1 - t)x}}
                 {\Gamma(\alpha)}
             \quad dx
           = \frac{1}{(1 - t)^\alpha}
             \int_{0}^{\infty} \frac 
                 {(1 - t)^\alpha x^{\alpha - 1} e^{-(1 - t)x}}
                 {\Gamma(\alpha)}
             \quad dx
           = \frac{1}{(1 - t)^\alpha}
\]
\[
    \Rightarrow
    m_Y(t) = (1 - t\beta)^\alpha
\]
when $t\beta < 1$.
\end{enumerate}
\end{proof}