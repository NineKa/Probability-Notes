\section{Cumulative Distribution Function}
\begin{definition}
For any real-valued random variable $Y$, the function $F_y : \mathbb{R}
\rightarrow [0, 1]$ defined by,
\[
	F_Y(y) = P(Y \leq y)
\]
is called the cumulative distribution function (or cdf, or distribution
function) of $Y$.
\end{definition}
\begin{example}
If $Y$ is a real-valued discrete random variable taking the values $y_1, y_2,
\dots$, then,
\[
    F_Y(y) = P(Y \leq y) 
           = \sum_{j : y_j \leq y} P(Y = y_j)
\]
\end{example}
\begin{example}
The pmf of $Y$ is given by,
\[
    p_Y(0) = \frac{1}{6} \qquad
    p_Y(1) = \frac{1}{3} \qquad
    p_Y(1.5) = \frac{1}{2}
\]
Find the cumulative distribution function of $Y$.
\end{example}
\begin{solution}
\[
    F_Y(y) = P(Y \leq y)
           = \sum_{y_j \leq y} P_Y(y_j)
\]
Hence,
\[
    F_Y(y) = \begin{cases}
        0                                              & y < 0          \\
        \frac{1}{6}                                    & 0 \leq y < 1   \\
        \frac{1}{6}+\frac{1}{3} =\frac{1}{2}           & 1 \leq y < 1.5 \\
        \frac{1}{6}+\frac{1}{3}+\frac{1}{2} = 1        & 1.5 \leq y
    \end{cases}
\]
\begin{figure}[H]
    \centering
    \includesvg[./section4/figure/]{sec3-fig1}
    \caption{Graph of $F_Y(\cdot)$}
\end{figure}
\end{solution}
\begin{example}
$Y \sim \geometricdist{p}$, find the cumulative distribution function of $Y$. 
\end{example}
\begin{solution}
\begin{align*}
    & P(Y = k) = p(1 - p)^{k - 1} \qquad k = 1, 2, \dots                     \\
    & \Rightarrow F_Y(y) = \sum_{k = 1}^{\lfloor y \rfloor} p (1 - p)^{k - 1}\\
    & \Rightarrow F_Y(y) = \begin{cases}
        0                  & y < 1                                           \\
        1 - (1 - p)^k      & k \leq y < k + 1, k = 1, 2, 3, \dots
    \end{cases}
\end{align*}
\end{solution}
\begin{theorem}
If $F : \mathbb{R} \rightarrow [0, 1]$ is the cumulative distribution function
of a real-valued random variable, then the following holds:
\begin{enumerate}[noitemsep, topsep=0em]
\item \label{ref-sec4-thm1-1}
    $F$ is non-decreasing on $\mathbb{R}$, i.e. if $y_1 \leq y_2$, then
$F(y_1) \leq F(y_2)$.
\item \label{ref-sec4-thm1-2}
    $F(\infty) \coloneqq \lim_{x \rightarrow \infty} F(x) = 1$.
\item \label{ref_sec4-thm1-3}
    $F(-\infty) \coloneqq \lim{x \rightarrow -\infty} F(x) = 0$.
\item \label{ref_sec4-thm1-4}
    $F$ is a right continuous function, i.e. $\forall y \in \mathbb{R}$ and
sequence $\lbrace y_n \rbrace$ such that $y_1 \geq y_2 \geq y_3 \geq
\dots$ and $\lim_{n \rightarrow \infty} y_n = y$, $\lim_{n \rightarrow \infty}
F(y_n) = F(y)$.
\end{enumerate}
\end{theorem}
\begin{proof} \quad                                                          \\
\begin{enumerate}[noitemsep, topsep=0em]
\item
    If $y_1 \leq y_2$, then $\lbrace y \leq y_1 \rbrace \subseteq \lbrace y
\leq y_2 \rbrace \Rightarrow P(Y \leq y_1) \leq P(Y \leq y_2) \Rightarrow
F(y_1) \leq F(y_2)$.
\item
    Take any sequence $x_n \uparrow \infty$. Let $A_0 = \lbrace Y \leq x_0
\rbrace$ and for all $n \geq 1$, let $A_n = \lbrace x_{n - 1} \leq Y \leq x_n
\rbrace$. Then,
\[
    \lbrace x \in \mathbb{R} \rbrace = \bigcup_{n = 0}^\infty A_n
\]
and the events $A_n$ are mutually disjoint. Hence,
\[
    1 = P(x \in \mathbb{R})
      = P(\bigcup_{n = 0}^\infty A_n)
      = \sum_{n = 0}^\infty P(A_n)
      = \lim_{n \rightarrow \infty} \left[ \sum_{k = 0}^n P(A_k) \right]
      = \lim_{n \rightarrow \infty} P(Y \leq x_n)
      = \lim_{n \rightarrow \infty} F(x_n)
\]
\item
    Note that, for any $y \in \mathbb{R}$, $1 - F(y) = P(Y > y)$. Consider any
decreasing sequence $y_n \downarrow -\infty$, and use arguments similar to the
ones used in \ref{ref-sec4-thm1-2} to show that, 
\[
                \lim_{n \rightarrow \infty} P(Y_1 > y_n) \ 1
    \Rightarrow 1 - F(y_n) \xrightarrow{n \rightarrow \infty} 1
    \Rightarrow F(y_n) \xrightarrow{n \rightarrow \infty} 0
\]
\item
    Fix $y \in \mathbb{R}$ and a sequence $y_n \downarrow y$. Let $A_0 =
\lbrace y_1 < Y \rbrace$, and for any $n \geq 1$, let $A_n = \lbrace y_{n + 1}
< Y \leq y_n \rbrace$. Then $A_0, A_1, A_2, \dots$ are mutually disjoint and,
\begin{align*}
    &\bigcup_{n = 0}^\infty A_n = \lbrace y < Y \rbrace                      \\
    \Rightarrow &
    P(y < Y) = P(\bigcup_{n = 0}^\infty A_n)
             = \sum_{k = 0}^\infty P(A_k)
             = \lim_{n \rightarrow \infty} \left[ \sum_{k = 0}^n P(A_k) \right]
             = \lim_{n \rightarrow \infty} P(Y > y_{n + 1})                  \\
    \Rightarrow &
    1 - F(y) = \lim_{n \rightarrow \infty} [1 - F(y_{n + 1})]                \\
    \Rightarrow &
    F(y) = \lim_{n \rightarrow \infty} F(y_n)
\end{align*}
\end{enumerate}
\end{proof}

\begin{theorem}[without proof]
If $F : \mathbb{R} \rightarrow [0, 1]$ satisfies \ref{ref-sec4-thm1-1},
\ref{ref-sec4-thm1-2}, \ref{ref_sec4-thm1-3}, and \ref{ref_sec4-thm1-4} of
previous theorem, then there exists a real-valued random variable $X$ on some
sample space such that $F$ is the cumulative distribution function of $X$.
\end{theorem}

\begin{theorem}
For $x \in \mathbb{R}$, let 
\[
    F_X(x-) \coloneqq \lim_{y \uparrow x} F_X(y)
\]
Then $F_X(x-) = P(X < x)$. In particular, $P(X = x) = F_X(x) - F_X(x-)$.
\begin{proof}
Imitate the arguments used in the argument earlier.
\end{proof}
\end{theorem}

The cumulative distribution function of a discrete real-valued random variable
has jumps.

\begin{definition}
A real-valued random variable $Y$ with cumulative distribution function
$F_Y(\cdot)$ is said to be a continuous random variable if $F_Y$ is a
continuous function on $\mathbb{R}$.
\end{definition}

The cumulative distribution function of a continuous real-valued random
variable $Y$ does not have any jumps. In particular, $\forall y \in
\mathbb{R}$,
\[
    P(Y = y) = P(Y \leq y) - P(Y \leq y)
             = F_Y(y) - F_Y(y-)
             = 0
\]
\begin{definition}
A function $f : \mathbb{R} \rightarrow [0, \infty)$ is called a density
function if,
\[
    \int_{-\infty}^\infty f(x) dx = 1
\]
\end{definition}
\begin{definition}
A real-valued continuous random variable $X$ has density function $f_X(\cdot)$
if $\forall x \in \mathbb{R}$, 
\[
    F_X(x) = P(X \leq x) = \int_{-\infty}^x f_x(u) du
\]
\end{definition}
\note From the definition, one can see that $F_X^\prime(x) = f_X(x)$ at every
$x$ that is a continuity point of $f_X$.

If a continuous random variable $X$ has a density then it can be computed as
follows. At every $x \in \mathbb{R}$ where the cumulative distribution
function $F_X$ is differentiable, set
\[
    f_x(x) \coloneqq F_X^\prime(x) = 0
\]
At every other $x$, set $f_X(x) = 0$.
\begin{example} \quad                                                        \\
\[
    F_X(x) = \begin{cases}
        0               & x < 0                                              \\
        1 - e^{-x}      & x \geq 0
    \end{cases}
\]
is a continuous cumulative distribution function. The corresponding density is,
\[
    f_X(x) = \begin{cases}
        0               & x \leq 0                                           \\
        e^{-x}          & x > 0
    \end{cases}
\]
\end{example}

This is analogous to the case of discrete random variables. If $X$ is a
real-valued discrete random variable with cumulative distribution function
$F_X$, then its pmf can be computed as follows. At every $x \in \mathbb{R}$
where $F_X$ has a jump, set
\[
    p_X(x) = F_X(x) - F_X(x-)
\]
and define $p_X(x) = 0$ at all other $x \in \mathbb{R}$.

\begin{theorem}
Assume that $X$ is a real-valued random variable with density $f$. Then for any
$a, b \in \mathbb{R}$ with $a < b$, 
\[
    P(X \in (a, b]) = P(X \in [a, b]) = P(X \in (a, b))
                    = F(b) - F(a)
                    = \int_{a}^b f(u) du
\]
\begin{proof}
\[
    P(X = a) = 0 = P(X = b)
    \Rightarrow
    P(X \in (a, b]) = P(X = a) + P(X \in (a, b]) = P(X \in [a, b])
\]
First four equalities follow similarly,
\[
    P(X \in (a, b]) = P(X \leq b) - P(X \leq a)
                    = F_X(b) - F_X(a)
                    = \int_{-\infty}^b f(u) du - \int_{-\infty}^a f(u) du 
                    = \int_{a}^b f(u) du
\]
\end{proof}
\end{theorem}

\subsection{Distribution of a Function of an Random Variable}
If $X$ is a real-valued random variable having a density function and $g :
\mathbb{R} \rightarrow \mathbb{R}$, then the cumulative distribution function
of $g(x)$ can be computed in a straightforward manner. Let $Y = g(X)$, then the
cumulative distribution function of $Y$ is, 
\[
    F_Y(y) = P(g(X) \leq y) 
           = P(X \in g^{-1}(-\infty, y])
\]
The density of $Y$, if it exists, can be computed just by differentiating
$F_Y$.

\begin{example}
The density of $X$ is given by,
\[
    f(x) = \begin{cases}
        c \cdot \vert x \vert       & x \in (-1, 1)                          \\
        0                           & \text{otherwise}
    \end{cases}
\]
\begin{enumerate}[noitemsep, topsep=0em]
\item Find the value of $c$.
\item Find the cumulative distribution function and the density of $Y = X^2$.
\end{enumerate}
\end{example}
\begin{solution} \quad                                                       \\
\begin{enumerate}[noitemsep, topsep=0em]
\item
\[
    \int_{-\infty}^\infty f(x) \quad dx = 1
    \Rightarrow \int_{-1}^1 c \vert x \vert \quad dx = 1
    \Rightarrow 2c \cdot \int_{0}^1 x \quad dx = 1
    \Rightarrow c = 1
\] 
\item
For $y < 0$
\[
    P(Y \leq y) = P(X^2 \leq y) = P(\emptyset) = 0
\]
And for $1 \geq y \geq 0$, then
\[
    P(X^2 \leq y) = P(-\sqrt{y} \leq X \leq \sqrt{y})
                  = \int_{-\sqrt{y}}^{\sqrt{y}} f(u) \quad du
                  = \int_{-\sqrt{y}}^{\sqrt{y}} \vert u \vert \quad du
                  = 2 \int_{0}^{\sqrt{y}} u \quad du
                  = y
\]
If $y > 1$, then
\[
    P(X^2 \leq y) = \int_{-\sqrt{y}}^{\sqrt{y}} f(u) \quad du
                  = \int_{-1}^{1} \vert u \vert \quad du +
                    \int_{1}^{\sqrt{y}} 0 \quad du +
                    \int_{-\sqrt{y}}^{-1} 0 \quad du
                  = 1
\]
Hence the cumulative distribution function of $Y$ is given by,
\[
    F_Y(y) = \begin{cases}
        0         & y < 0                                                    \\
        u         & 0 \leq y \leq 1                                          \\
        1         & y < 1         
    \end{cases}
\]
The density of $Y$ is given by,
\[
    f_Y(y) = \begin{cases}
        0         & y \leq 0                                                 \\
        1         & 0 < y < 1                                                \\
        0         & 1 \leq y
    \end{cases}
\]
\end{enumerate}
\end{solution}
\subsection{Expectation}
\begin{definition}
If $X$ is real-valued random variable with a density $f$, and either 
\[
    \int_0^\infty f(x) \quad dx < \infty
    \qquad \text{or} \qquad
    \int_{-\infty}^0 \vert x \vert f(x) \quad dx < \infty
\]
then the expectation of $X$ is defined as,
\[
    E[X] = \int_{-\infty}^{\infty} x f(x) \quad dx
\]
If both $\int_{0}^{\infty} x f(x) \quad dx$ and $\int_{-\infty}^0 \vert x
\vert f(x) \quad dx$ are infinite, then we say that $E[X]$ does not exist.
\end{definition}
\note the similarity with discrete random variables essentially we are
replacing pmf by density and sum by integral.

\begin{definition}
For any real-valued random variable $X$ with cumulative distribution function
$F_X$, then expectation of $X$ is given by,
\[
    E[X] = \int_{-\infty}^{\infty} x \quad dF_X(x)
\]
where the integral is interpreted in the Riemann-Stieltjes sense. 
\end{definition}

\begin{theorem}[without proof]
Let $X_1$, $X_2$, $\dots$, $X_k$ be continuous random variables defined on the
same sample space having densities $f_1$, $f_2$, $\dots$, $f_k$ respectively.
\begin{enumerate}[noitemsep, topsep=0em]
\item
    Let $c \in \mathbb{R}$ and $g : \mathbb{R} \rightarrow \mathbb{R}$. Then, 
\[
    E[c \cdot g(X_1)] = c \cdot E[g(X_1)] 
                      = c \cdot \int_{-\infty}^{\infty} g(u)f_1(u) \quad du
\]
\item
\[
    E[X_1 + \dots + X_k] = E[X_1] + \dots + E[X_k]
                         = \int_{-\infty}^{\infty} x f_1(x) \quad dx +
                           \dots +
                           \int_{-\infty}^{\infty} x f_k(x) \quad dx
\]
\end{enumerate}
\end{theorem}
\note Even if $X_1$, $X_2$ are continuous random variables with densities,
$g(X_1)$ or $X_1 + X_2$ may not be continuous random variables. In that case, 
$E[g(X_1)]$ or $E[X_1 + X_2]$ is defined through the Riemann-Stielljes
integral. But the above theorem says that the expected values computed using
that definition coincides with,
\[
    \int_{-\infty}^{\infty} g(x) f(x) \quad dx
    \qquad \text{and} \qquad
    \int_{-\infty}^{\infty} x f_1(x) \quad dx + 
    \int_{-\infty}^{\infty} x f_2(x) \quad dx
\]
respectively.

\begin{definition}
If $X$ has density $f$,  then the variance of $X$ is given by, 
\[
    V[X] = E[(x - \mu)^2]
         = \int_{-\infty}^{\infty} (x - \mu)^2 f(x) \quad dx
\]
where $\mu = E[X]$. Consequently,
\[
    V[X] = E[X^2] - \mu^2
         = \int_{-\infty}^{\infty} x^2 f(x) \quad dx -
           \left( \int_{-\infty}^{\infty} x f(x) \quad dx \right)^2
\]
\end{definition}

\begin{example} \quad                                                        \\
\[
    F_X(x) = \begin{cases}
        0                    & x \leq 0                                      \\
        \frac{x}{8}          & 0 < x < 2                                     \\
        \frac{x^2}{16}       & 2 \leq 4 < 4                                  \\
        1                    & 4 \leq x 
    \end{cases}
\]
\begin{enumerate}[noitemsep, topsep=0em]
\item Find $P(1 \leq X \leq 3)$.
\item Find the density of $X$.
\item Find $E[X]$, $V[X]$.
\end{enumerate}
\end{example}
\begin{solution} \quad \linebreak                                
\begin{enumerate}[noitemsep, topsep=0em]
\item
\[
    P(1 \leq X \leq 3) = F_X(3) - F_X(1)
                       = \frac{9}{16} - \frac{1}{8}
                       = \frac{7}{16}
\]
\item
Density $f_x(\cdot)$ is given by,
\[
    f_X(x) = \begin{cases}
        0                  & x \leq 0                                       \\
        \frac{1}{8}        & 0 < x < 2                                      \\
        0                  & x = 2                                          \\
        \frac{x}{8}        & 2 < x < 5                                      \\
        0                  & 4 \leq x
    \end{cases}
\]
\end{enumerate}
\begin{figure}[H]
    \centering
    \begin{multicols}{2}
    \def\svgwidth{\linewidth}
    \includesvg[./section4/figure/]{sec3-fig2}
    \columnbreak
    \def\svgwidth{\linewidth}
    \includesvg[./section4/figure/]{sec3-fig3}
    \end{multicols}
\end{figure}
\note If instead of $F_X$, you were given $f_X$, then you could compute $P(1
\leq X \leq 3)$ using $\int_{1}^{3} f_x(u) \quad du$.
\item
\[
    E[X] = \int_{-\infty}^{\infty} x f(x) \quad dx
         = \int_{0}^2 \frac{x}{8} \quad dx + 
           \int_{2}^4 \frac{x^2}{8} \quad dx
      %% = TODO: finish computation
\]
\[
    E[X^2] = \int_{-\infty}^{\infty} x^2 f(x) \quad dx
           = \int_{0}^{2} \frac{x^2}{8} \quad dx +
             \int_{2}^{4} \frac{x^3}{8} \quad dx
        %% = TODO: finish computation
\]
\[
    V[X] = E[X^2] - (E[X])^2 %% = TODO: finish computation
\]
\end{solution}

\begin{example}
Suppose $X$ is a continuous random variable with density $f(\cdot)$ and
$E[\vert X \vert] < \infty$. Show that, 
\[
    \lim_{a \rightarrow \infty} a P(\vert X \vert \geq a) = 0
\]
\end{example}
\begin{solution}
\begin{equation} \label{ref-sec4-thm2-1}
    E[\vert X \vert] = \int_{-\infty}^{\infty} \vert x \vert f(x) \quad dx
                     < \infty
    \Rightarrow
    \lim_{a \rightarrow \infty} \left[
        \int_{a}^{\infty} x f(x) \quad dx +
        \int_{-\infty}^{-a} \vert x \vert f(x) \quad dx
    \right] = 0
\end{equation}
Now,
\[
    \int_{a}^{\infty} f(x) \quad dx \geq a \int_{a}^{\infty} f(x) \quad dx
                                    =    a P(X \geq a)
\]
Similarly,
\[
         \int_{-\infty}^{-a} \vert x \vert f(x) \quad dx 
    \geq a \int_{-\infty}^{-a} f(x) dx
    =    a P(x \leq -a)
\]
Hence \ref{ref-sec4-thm2-1} implies that,
\[
    \lim_{a \rightarrow \infty} [
        a P(X \geq a) +
        a P(X \leq -a)
    ] = 0
    \Rightarrow
    \lim{a \rightarrow \infty} a P(\vert X \vert \geq a) = 0
\]
\end{solution}


\subsection{Uniform Distribution}
\begin{definition}
Let $\theta_1 \leq \theta_2$. An random variable $X$ is said to have a uniform
distribution on the interval $[\theta_1, \theta_2]$, or simply $X \sim
\uniformdist{\theta_1}{\theta_2}$, if and only if the function $f$ given by,
\[
    f(x) = \begin{cases}
        \frac{1}{\theta_2 - \theta_1}      & x \in [\theta_1, \theta_2].     \\
        0                                  & \text{otherwise}
    \end{cases}
\]
is a density for $X$. In particular, the cumulative distribution function of
$X$ is given by,
\[
    F_X(x) = \begin{cases}
        0                                        & x < \theta_1              \\
        \frac{x - \theta_1}{\theta_2 - \theta_1} & \theta_1 \leq x < \theta_2\\
        1                                        & \theta_2 \leq x
    \end{cases}
\]
\begin{figure}[H]
    \centering
    \def\svgwidth{0.5\textwidth}
    \includesvg[./section4/figure/]{sec3-fig4}    
\end{figure}
\end{definition}

\begin{theorem} \quad                                                        \\
\begin{enumerate}[noitemsep, topsep=0em]
\item
    If $X \sim \uniformdist{0}{1}$, then $\forall a \in \mathbb{R}, b > 0$,
\[ 
    Y = a + bX \sim \uniformdist{a}{a + b}
\]
\item
    If $X \sim \uniformdist{\theta_1}{\theta_2}$ and $\theta_1 < \theta_2$,
then $\forall \theta_3, \theta_4 \in [\theta_1, \theta_2]$ with $\theta_3 <
\theta_4$, the random variable $Z$ whose distribution is given by,
\[
    P(Z \leq \mathbb{Z}) \coloneqq P(X \leq \mathbb{Z} \vert 
                                     X \in [\theta_3, \theta_4])
\]
is uniformly distributed over $[\theta_3, \theta_4]$, i.e., $Z \sim
\uniformdist{\theta_3}{\theta_4}$.
\end{enumerate}
\end{theorem}
\begin{proof} \quad                                                          \\
\begin{enumerate}[noitemsep, topsep=0em]
\item
\begin{align*}
    P(Y \leq y) &= P(a + bX \leq y)
                 = P(X \leq \frac{y - a}{b})
                 = F_X \left( \frac{y - a}{b} \right)                        \\
    &\Rightarrow F_Y(y) = \begin{cases}
        0                   & y < a                                          \\
        \frac{y - a}{b}     & a \leq y < a + b                               \\
        1                   & b \leq y
    \end{cases}                                                              \\
    &\Rightarrow Y \sim \uniformdist{a}{a + b}
\end{align*}   
\item
\[
    P(Z \leq \mathbb{Z}) = \frac
        {P(X \in (-\infty, \mathbb{Z}] \cap [\theta_3, \theta_4]))}
        {P(X \in [\theta_3, \theta_4])}
\]
Now,
\[
    (-\infty, \mathbb{Z}] \cap [\theta_3, \theta_4] = \begin{cases}
        \emptyset               & \mathbb{Z} < \theta_3                      \\
        [\theta_3, \mathbb{Z}]  & \theta_3 \leq \mathbb{Z} < \theta_4        \\
        [\theta_3, \theta_4]    & \theta_4 \leq \mathbb{Z}
    \end{cases}
\]
and, $P(X \in [\theta_3, \theta_4]) = \frac{\theta_4 - \theta_3}{\theta_2 -
\theta_1}$. Hence,
\[
    P(Z \leq \mathbb{Z}) = \begin{cases}
        0                       & \mathbb{Z} < \theta_3                      \\
        \frac{\mathbb{Z}-\theta_3}{\theta_4 - \theta_3} 
            & \theta_3 \leq \mathbb{Z} < \theta_4                            \\
        1                       & \theta_4 \leq \mathbb{Z}
    \end{cases}
    \Rightarrow
    Z \sim \uniformdist{\theta_3}{\theta_4}
\]
\end{enumerate}
\end{proof}