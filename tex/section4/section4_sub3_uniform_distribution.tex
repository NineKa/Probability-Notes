\subsection{Uniform Distribution}
\begin{definition}
Let $\theta_1 \leq \theta_2$. An random variable $X$ is said to have a uniform
distribution on the interval $[\theta_1, \theta_2]$, or simply $X \sim
\uniformdist{\theta_1}{\theta_2}$, if and only if the function $f$ given by,
\[
    f(x) = \begin{cases}
        \frac{1}{\theta_2 - \theta_1}      & x \in [\theta_1, \theta_2].     \\
        0                                  & \text{otherwise}
    \end{cases}
\]
is a density for $X$. In particular, the cumulative distribution function of
$X$ is given by,
\[
    F_X(x) = \begin{cases}
        0                                        & x < \theta_1              \\
        \frac{x - \theta_1}{\theta_2 - \theta_1} & \theta_1 \leq x < \theta_2\\
        1                                        & \theta_2 \leq x
    \end{cases}
\]
\begin{figure}[H]
    \centering
    \def\svgwidth{0.5\textwidth}
    \includesvg[./section4/figure/]{sec3-fig4}    
\end{figure}
\end{definition}

\begin{theorem} \quad                                                        \\
\begin{enumerate}[noitemsep, topsep=0em]
\item
    If $X \sim \uniformdist{0}{1}$, then $\forall a \in \mathbb{R}, b > 0$,
\[ 
    Y = a + bX \sim \uniformdist{a}{a + b}
\]
\item
    If $X \sim \uniformdist{\theta_1}{\theta_2}$ and $\theta_1 < \theta_2$,
then $\forall \theta_3, \theta_4 \in [\theta_1, \theta_2]$ with $\theta_3 <
\theta_4$, the random variable $Z$ whose distribution is given by,
\[
    P(Z \leq \mathbb{Z}) \coloneqq P(X \leq \mathbb{Z} \vert 
                                     X \in [\theta_3, \theta_4])
\]
is uniformly distributed over $[\theta_3, \theta_4]$, i.e., $Z \sim
\uniformdist{\theta_3}{\theta_4}$.
\end{enumerate}
\end{theorem}
\begin{proof} \quad                                                          \\
\begin{enumerate}[noitemsep, topsep=0em]
\item
\begin{align*}
    P(Y \leq y) &= P(a + bX \leq y)
                 = P(X \leq \frac{y - a}{b})
                 = F_X \left( \frac{y - a}{b} \right)                        \\
    &\Rightarrow F_Y(y) = \begin{cases}
        0                   & y < a                                          \\
        \frac{y - a}{b}     & a \leq y < a + b                               \\
        1                   & b \leq y
    \end{cases}                                                              \\
    &\Rightarrow Y \sim \uniformdist{a}{a + b}
\end{align*}   
\item
\[
    P(Z \leq \mathbb{Z}) = \frac
        {P(X \in (-\infty, \mathbb{Z}] \cap [\theta_3, \theta_4]))}
        {P(X \in [\theta_3, \theta_4])}
\]
Now,
\[
    (-\infty, \mathbb{Z}] \cap [\theta_3, \theta_4] = \begin{cases}
        \emptyset               & \mathbb{Z} < \theta_3                      \\
        [\theta_3, \mathbb{Z}]  & \theta_3 \leq \mathbb{Z} < \theta_4        \\
        [\theta_3, \theta_4]    & \theta_4 \leq \mathbb{Z}
    \end{cases}
\]
and, $P(X \in [\theta_3, \theta_4]) = \frac{\theta_4 - \theta_3}{\theta_2 -
\theta_1}$. Hence,
\[
    P(Z \leq \mathbb{Z}) = \begin{cases}
        0                       & \mathbb{Z} < \theta_3                      \\
        \frac{\mathbb{Z}-\theta_3}{\theta_4 - \theta_3} 
            & \theta_3 \leq \mathbb{Z} < \theta_4                            \\
        1                       & \theta_4 \leq \mathbb{Z}
    \end{cases}
    \Rightarrow
    Z \sim \uniformdist{\theta_3}{\theta_4}
\]
\end{enumerate}
\end{proof}

\begin{theorem}
If $X \sim \uniformdist{\theta_1}{\theta_2}$, then
\begin{enumerate}[noitemsep, topsep=0em]
\item  \[  E[X] = \frac{\theta_1 + \theta_2}{2}                              \]
\item  \[  V[X] = \frac{(\theta_2 - \theta-1)^2}{12}                         \]
\item 
\[
    m_X(t) = \begin{cases}
        \frac{(e^{t\theta_2} - e^{t\theta_1)}}{t(\theta_2 - \theta_1)} 
            & t \neq 0                                                       \\
        1
            & t = 0
    \end{cases}
\]    
\end{enumerate}
\end{theorem}
\begin{proof}\quad                                                           \\
\begin{enumerate}[noitemsep, topsep=0em]
\item
\[
    E[X] = \int_{-\infty}^{\infty} x f_X(x) \quad dx
         = \int_{\theta_1}^{\theta_2} \frac{x}{\theta_2 - \theta_1} \quad dx   
         = \frac{1}{2(\theta_2 - \theta_1)}(\theta_2^2 - \theta_1^2)
         = \frac{\theta_1 + \theta_2}{2}
\]
\item
    Suppose $Y \sim \uniformdist{0}{1}$. Then $X \overset{\text{d}}{=} a +
bY$, where $a = \theta_1$, $b = \theta_2 - \theta_1$. Now,
\[
    E[Y] = \frac{1}{2}
    \qquad \text{and} \qquad
    E[Y^2] = \int_{0}^{1} y^2 \quad dy = \frac{1}{3}
    \Rightarrow
    V[Y] = E[Y^2] - (E[Y])^2
         = \frac{1}{3} - \frac{1}{4}
         = \frac{1}{12}
\]
Hence,
\[
    V[X] = V[a + bY] = b^2 V[Y]
         = \frac{(\theta_2 - \theta_1)^2}{12}
\]
\item
    Note that $m_X(t) = E[e^{tX}] = 1$ if $t = 0$. If $t \neq 0$, then,
\[
    m_X(t) = E[e^{tX}] 
           = \int_{-\infty}^{\infty} e^{tX} \cdot f_X(x) \quad dx
           = \int_{\theta_1}^{\theta_2} \frac{e^{tX}}{\theta_2 - \theta_1} 
             \quad dx
           = \frac{e^{t\theta_2} - e^{t\theta_1}}{t(\theta_2 - \theta_1)}
\]
\end{enumerate}
\end{proof}

\begin{example}
The radius of a circle is an random variable that has a $\uniformdist{0}{1}$
distribution. What is the probability that its area is bigger than $1$?
\end{example}
\begin{solution}
Let $A$ be random variable equals to the size of the circle. Note that $A =
\pi R^2$, where $R \sim \uniformdist{0}{1}$,
\[
    P(A \geq 1) = P(\pi R^2 \geq 1)
                = P(R \geq \frac{1}{\sqrt{\pi}})
                = \int_{\frac{1}{\sqrt{\pi}}}^{1} 1 \quad dx
                = 1 - \frac{1}{\sqrt{\pi}}
\]
\end{solution}

