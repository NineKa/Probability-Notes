\subsection{Expectation}
\begin{definition}
If $X$ is real-valued random variable with a density $f$, and either 
\[
    \int_0^\infty f(x) \quad dx < \infty
    \qquad \text{or} \qquad
    \int_{-\infty}^0 \vert x \vert f(x) \quad dx < \infty
\]
then the expectation of $X$ is defined as,
\[
    E[X] = \int_{-\infty}^{\infty} x f(x) \quad dx
\]
If both $\int_{0}^{\infty} x f(x) \quad dx$ and $\int_{-\infty}^0 \vert x
\vert f(x) \quad dx$ are infinite, then we say that $E[X]$ does not exist.
\end{definition}
\note the similarity with discrete random variables essentially we are
replacing pmf by density and sum by integral.

\begin{definition}
For any real-valued random variable $X$ with cumulative distribution function
$F_X$, then expectation of $X$ is given by,
\[
    E[X] = \int_{-\infty}^{\infty} x \quad dF_X(x)
\]
where the integral is interpreted in the Riemann-Stieltjes sense. 
\end{definition}

\begin{theorem}[without proof]
Let $X_1$, $X_2$, $\dots$, $X_k$ be continuous random variables defined on the
same sample space having densities $f_1$, $f_2$, $\dots$, $f_k$ respectively.
\begin{enumerate}[noitemsep, topsep=0em]
\item
    Let $c \in \mathbb{R}$ and $g : \mathbb{R} \rightarrow \mathbb{R}$. Then, 
\[
    E[c \cdot g(X_1)] = c \cdot E[g(X_1)] 
                      = c \cdot \int_{-\infty}^{\infty} g(u)f_1(u) \quad du
\]
\item
\[
    E[X_1 + \dots + X_k] = E[X_1] + \dots + E[X_k]
                         = \int_{-\infty}^{\infty} x f_1(x) \quad dx +
                           \dots +
                           \int_{-\infty}^{\infty} x f_k(x) \quad dx
\]
\end{enumerate}
\end{theorem}
\note Even if $X_1$, $X_2$ are continuous random variables with densities,
$g(X_1)$ or $X_1 + X_2$ may not be continuous random variables. In that case, 
$E[g(X_1)]$ or $E[X_1 + X_2]$ is defined through the Riemann-Stielljes
integral. But the above theorem says that the expected values computed using
that definition coincides with,
\[
    \int_{-\infty}^{\infty} g(x) f(x) \quad dx
    \qquad \text{and} \qquad
    \int_{-\infty}^{\infty} x f_1(x) \quad dx + 
    \int_{-\infty}^{\infty} x f_2(x) \quad dx
\]
respectively.

\begin{definition}
If $X$ has density $f$,  then the variance of $X$ is given by, 
\[
    V[X] = E[(x - \mu)^2]
         = \int_{-\infty}^{\infty} (x - \mu)^2 f(x) \quad dx
\]
where $\mu = E[X]$. Consequently,
\[
    V[X] = E[X^2] - \mu^2
         = \int_{-\infty}^{\infty} x^2 f(x) \quad dx -
           \left( \int_{-\infty}^{\infty} x f(x) \quad dx \right)^2
\]
\end{definition}

\begin{example} \quad                                                        \\
\[
    F_X(x) = \begin{cases}
        0                    & x \leq 0                                      \\
        \frac{x}{8}          & 0 < x < 2                                     \\
        \frac{x^2}{16}       & 2 \leq 4 < 4                                  \\
        1                    & 4 \leq x 
    \end{cases}
\]
\begin{enumerate}[noitemsep, topsep=0em]
\item Find $P(1 \leq X \leq 3)$.
\item Find the density of $X$.
\item Find $E[X]$, $V[X]$.
\end{enumerate}
\end{example}
\begin{solution} \quad \linebreak                                
\begin{enumerate}[noitemsep, topsep=0em]
\item
\[
    P(1 \leq X \leq 3) = F_X(3) - F_X(1)
                       = \frac{9}{16} - \frac{1}{8}
                       = \frac{7}{16}
\]
\item
Density $f_x(\cdot)$ is given by,
\[
    f_X(x) = \begin{cases}
        0                  & x \leq 0                                       \\
        \frac{1}{8}        & 0 < x < 2                                      \\
        0                  & x = 2                                          \\
        \frac{x}{8}        & 2 < x < 5                                      \\
        0                  & 4 \leq x
    \end{cases}
\]
\end{enumerate}
\begin{figure}[H]
    \centering
    \begin{multicols}{2}
    \def\svgwidth{\linewidth}
    \includesvg[./section4/figure/]{sec3-fig2}
    \columnbreak
    \def\svgwidth{\linewidth}
    \includesvg[./section4/figure/]{sec3-fig3}
    \end{multicols}
\end{figure}
\note If instead of $F_X$, you were given $f_X$, then you could compute $P(1
\leq X \leq 3)$ using $\int_{1}^{3} f_x(u) \quad du$.
\item
\[
    E[X] = \int_{-\infty}^{\infty} x f(x) \quad dx
         = \int_{0}^2 \frac{x}{8} \quad dx + 
           \int_{2}^4 \frac{x^2}{8} \quad dx
      %% = TODO: finish computation
\]
\[
    E[X^2] = \int_{-\infty}^{\infty} x^2 f(x) \quad dx
           = \int_{0}^{2} \frac{x^2}{8} \quad dx +
             \int_{2}^{4} \frac{x^3}{8} \quad dx
        %% = TODO: finish computation
\]
\[
    V[X] = E[X^2] - (E[X])^2 %% = TODO: finish computation
\]
\end{solution}

\begin{example}
Suppose $X$ is a continuous random variable with density $f(\cdot)$ and
$E[\vert X \vert] < \infty$. Show that, 
\[
    \lim_{a \rightarrow \infty} a P(\vert X \vert \geq a) = 0
\]
\end{example}
\begin{solution}
\begin{equation} \label{ref-sec4-thm2-1}
    E[\vert X \vert] = \int_{-\infty}^{\infty} \vert x \vert f(x) \quad dx
                     < \infty
    \Rightarrow
    \lim_{a \rightarrow \infty} \left[
        \int_{a}^{\infty} x f(x) \quad dx +
        \int_{-\infty}^{-a} \vert x \vert f(x) \quad dx
    \right] = 0
\end{equation}
Now,
\[
    \int_{a}^{\infty} f(x) \quad dx \geq a \int_{a}^{\infty} f(x) \quad dx
                                    =    a P(X \geq a)
\]
Similarly,
\[
         \int_{-\infty}^{-a} \vert x \vert f(x) \quad dx 
    \geq a \int_{-\infty}^{-a} f(x) dx
    =    a P(x \leq -a)
\]
Hence \ref{ref-sec4-thm2-1} implies that,
\[
    \lim_{a \rightarrow \infty} [
        a P(X \geq a) +
        a P(X \leq -a)
    ] = 0
    \Rightarrow
    \lim{a \rightarrow \infty} a P(\vert X \vert \geq a) = 0
\]
\end{solution}

