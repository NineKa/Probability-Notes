\subsection{Conditional Probability}
The knowledge that an event has occurred affects the odds of occurrence of a
correlated event. This is formalized in the notion of conditional probability.

\begin{example}
    Roll a fair die. Given the information that the outcome is even, what
probability should be assigned to the event $\lbrace 6 \rbrace$?
\end{example}
\begin{solution}
    As $S^\prime = \lbrace 2, 4, 6 \rbrace$. The conditional probability of the
event $\lbrace 6 \rbrace$ should be $\frac{1}{3}$. (We are using the fact that
the die is fair).
\end{solution}

\begin{definition}
    If $A$ and $B$ are events and $P(B) > 0$, then the "conditional probability
of $A$ given $B$" is defined as, 
\begin{equation*}
    P(A \vert B) = \frac{P(A \cap B)}{P(B)}
\end{equation*}
\end{definition}

\begin{definition}
    Two events $A$ and $B$ are independent if $P(A \cap B) = P(A) \cdot P(B)$.
In general $A_1, A_2, \dots, A_k$ are independent events if for any $1 \leq
i_1 < i_2 < \dots < i_r \leq k$,
\begin{equation*}
    P(A_{i_1} \cap A_{i_2} \cap \dots \cap A_{i_r}) = 
    P(A_{i_1} \cdot A_{i_2} \cdot \dots \cdot P(A_{i_r})
\end{equation*}
\end{definition}

\noindent \textbf{Note}: If $A$ and $B$ are independent and $P(B) > 0$, then 
\begin{equation*}
    P(A \vert B) = \frac{P(A \cap B)}{P(B)}
                 = \frac{P(A) \cdot P(B)}{P(B)}
                 = P(A)
\end{equation*}
In other words, if $A$ and $B$ are independent, then knowledge about occurrence
of $B$ does not affect the chances of occurrence of $A$.

\begin{example}
    In a biased die, the probability of face $i$ is proportional to $i$, $1
\leq i \leq 6$. Given the information that a number less than or equal to $4$
came up when the die was rolled, what is the probability that the number was
less than or equal to $2$?
\end{example}
\begin{solution}
    As, $P(\lbrace i \rbrace) = c \cdot i, \forall 1 \leq i \leq 6$. Let $A =
\lbrace 1, 2, 3, 4 \rbrace$ and $B = \lbrace 1, 2 \rbrace$. Then, 
\begin{equation*}
    P(B \vert A) = \frac{P(A \cap B)}{P(A)}
                 = \frac{P(B)}{P(A)}
                 = \frac{c \cdot (1 + 2)}{c \cdot (1 + 2 + 3 + 4)}
                 = \frac{3}{10}
\end{equation*}
\end{solution}