\subsection{Independent Experiments}
\begin{definition}
    Two experiments $E_1$ and $E_2$ conducted jointly are said to have been
conducted independently if the following holds. If $A_1$ is any event whose
occurrence depends only on the outcome of $E_1$ and $A_2$ is any event whose
occurrence depends only of $E_2$, then $A_1$ and $A_2$ are independent.
\end{definition}
Can be generalized to countably many experiments.

\begin{example}
    Toss a fair coin three times independently. $P(\text{three tails}) =
(\frac{1}{2})^3 = \frac{1}{8}$.
\end{example}

\begin{example}
    Toss a fair coin and roll a fair die independently. What is
$P(\lbrace \text{tails} \rbrace \cap \lbrace \text{even numbers from
die} \rbrace)$?
\end{example}
\begin{solution}
    \begin{equation*}
        P(\lbrace \text{tails} \rbrace \cap \lbrace \text{even numbers from
          die} \rbrace)
        = P(\lbrace \text{tails} \rbrace) \cdot
          P(\lbrace \text{even numbers from die} \rbrace)
        = \frac{1}{2} \cdot \frac{1}{2}
        = \frac{1}{4}
    \end{equation*}
\end{solution}

\begin{example}[exercise 2.119 of textbook]
    Two fair dice are tossed repeatedly and the sum of the two uppermost faces
is determined on each toss. What is the probability that we obtain a sum of $3$
before obtaining a sum of $7$?
\end{example}
\begin{solution}
    The experiment is rolling two dice independently countably infinitely many
times, and notice the sum of the two uppermost faces. Let $E_{3, i}$ be the
event that we get a sum of $3$ in the $i$-th toss. Let $E_{7, i}$ be the event
that we get a sum of $7$ in the $i$-th toss. Let $F_i$ be the event that we get
a sum that is not $3$ nor $7$ in the $i$-th toss. Then, 
\begin{align*}
    P(E_{3, i}) &= \frac{2}{36} = \frac{1}{18} \quad \forall i \geq 1        \\
    P(E_{7, i}) &= \frac{6}{36} = \frac{1}{6}  \quad \forall i \geq 1        \\
    P(F_i)     &= 1 - P(F_{3, i}) - P(F_{7, i}) = \frac{7}{9}
\end{align*}
\begin{align*}
       P(\text{sum of $3$ before sum of $7$})
    &= P(E_{3, 1}) + P(F_1 \cap E_{3, 2}) + P(F_1 \cap F_2 \cap E_{3, 3}) +
       \dots                                                                 \\
    &= P(E_{3, 1}) + P(F_1) \cdot P(E_{3, 2}) + P(F_1) \cdot P(F_2) \cdot
       P(E_{3, 3}) + \dots                                                   \\
    &= \sum_{i = 0}^\infty (\frac{7}{9})^i \cdot \frac{1}{18}                \\
    &= \frac{1}{18} \cdot \frac{1}{1 - \frac{7}{9}}
     = \frac{1}{18} \cdot \frac{9}{2}
     = \frac{1}{4}
\end{align*}
\end{solution}

\begin{example}[exercise 2.154(b) of textbook]
    A drawer contains $n$ different pairs of socks. The two socks making up a
matching pair are distinguishable. A person selects $2r$ many socks at random,
where $2r < n$. What is the probability that there is no matching pair in the
sample?
\end{example}
\begin{solution}
    As, $\vert S \vert = {{2n} \choose {2r}}$, and $\vert E \vert = {{n}
\choose {2r}} \cdot 2^{2r}$ (choose $2r$ pairs and then select one of the two
socks in each pair). Thus, the required probability is,
\begin{equation*}
       P(\text{no matching pair})
    = \frac{{{n} \choose {2r}} \cdot 2^{2r}}{{{2n} \choose {2r}}}
\end{equation*}
\end{solution}

\begin{example}[exercise 2.181 of textbook]
    Suppose that $n$ indistinguishable balls are to be arranged in $N$
distinguishable boxes. If $n \geq N$ and all arrangements are equally likely,
then show that the probability of no box being empty is given by
$\frac{{{n-1} \choose {N - 1}}}{{{N + n -1} \choose {N - 1}}}$
\end{example}
\begin{solution}
    Since the ball are indistinguishable. The sample space $S$ is given by,
    \begin{equation*}
        S = \lbrace (x_1, x_2, \dots, x_N) \vert x_i \text{ are nonnegative
                    integers such that } \sum_{i = 1}^N x_i = n \rbrace
    \end{equation*}
    Here $x_i$ represents the number of balls in box $i$. Hence, $\vert S
\vert = {{N + n - 1} \choose {N - 1}}$. If $E$ is the event that no box is
empty, then $\vert E \vert$ equals the number of positive integer valued
solutions to,
    \begin{equation}
        \label{thm-counting-tech-5-1}
        x_1 + x_2 + \dots + x_N = n
    \end{equation}
    Let $y_i = x_i - 1$. Then $y_i \geq 0$. The number of positive
integer-valued solutions to \ref{thm-counting-tech-5-1} above is same as the
number of nonnegative integer-valued solution to $y_1 + y_2 + \dots + y_N = n
- N$. Hence, $\vert E \vert = {{n -N + N - 1} \choose {N - 1}} = {{n - 1}
\choose {N - 1}}$. Thus the required probability is, 
    \begin{equation*}
          P(\text{no box being empty}) 
        = \frac{\vert E \vert}{\vert S \vert}
        = \frac{{{n - 1}\choose {N - 1}}}{{{N + n - 1} \choose {N - 1}}}
    \end{equation*}
\end{solution}