\subsection{Counting Techniques}
Suppose you have to complete two tasks. If Task 1 can be completed in $n_1$
ways, and for each of these $n_1$ possibilities, Task 2 can be completed in
$n_2$ ways, then the total number of ways the two tasks can be completed is
$n_1 \cdot n_2$.
\begin{example}
Toss a coin and roll a die simultaneously. The former can have two outcomes,
and for each of these, the latter can have six outcomes. Thus, the total number
of possible outcomes is $2 \times 6 = 12$. 
\end{example}

\noindent \textbf{Recall} for any $n \in \mathbb{N}$, $n! = 1 \times 2 \times
\dots \times n$, and $0! = 1$ by definition. 

\begin{theorem}
    If we are given $n$ distinct objects and $m$ distinct boxes each of which
can hold one object and $n \leq m$, then the number of ways of filling up the
boxes is, 
\begin{equation*}
    \perm[n]{m} = \frac{n!}{(n - m)!}
\end{equation*}
\end{theorem}
\begin{proof}
    The first box can be filled up in $n$ ways. For each of these $n$ ways, the
second box can be filled up in $n - 1$ ways (by using any of the remaining $n
- 1$ objects). Thus the first two boxes can be filled up in $n \cdot (n - 1)$
ways. For each of these $n \cdot (n - 1)$ ways, the third box can be filled up
in $(n - 2)$ ways. Thus, the first $3$ boxes can be filled up in $n \cdot (n -
1) \cdot (n - 2)$ ways. Continuing like this, the $m$ boxes can be filled up
in, 
\begin{equation*}
    n \cdot (n-1) \cdot \dots \cdot (n - m + 1) = \frac{n!}{(n - m)!}
\end{equation*}
\end{proof}

\begin{example}
The papers from $3$ different exams need to be graded. $5$ graders are
available for this job. Each exam requires only one grader. Then the number of
ways of assigning graders to different exams is, 
\begin{equation*}
    \perm[5]{3} = \frac{5!}{(5 - 3)!} = 60
\end{equation*}
\end{example}

\begin{theorem}
    If $r \leq n$, the number of ways of selecting an unordered sample of $r$
objects from a total of $n$ distinct objects is,
\begin{equation*}
    \comb[n]{r} = \frac{n!}{(n - r)!r!}
\end{equation*}
This is also denoted by $n \choose r$.
\end{theorem}
\begin{proof}
    Suppose this can be done in $m$ ways. Now if we had $r$ distinct boxes,
then we could fill them up in two steps:
\begin{enumerate}[noitemsep,topsep=0pt]
\item 
    select an unordered sample of size $r$ from the $n$ distinct objects.
    This can be done in $m$ ways.
\item
    use these $r$ objects to fill up the $r$ boxes by putting one object in
    each box. This can be done in $\perm[r]{r}$ ways.
\end{enumerate}
Thus the total number of ways of filling up those $r$ boxes is $m \cdot
r!$. Hence,
\begin{equation*}
    m \cdot r! = \perm[n]{r} = \frac{n!}{(n - r)!}
    \Rightarrow
    m = \frac{n!}{(n - r)!r!}
\end{equation*}
\end{proof}

\begin{example}
    A committee of $3$ is to be formed from a group of $10$ people. How many
different committees are possible?
\end{example}
\begin{solution}
    Number of different committees is,
    \begin{equation*}
        {10 \choose 3} = \frac{10 \cdot 9 \cdot 8}{3 \cdot 2 \cdot 1} = 120
    \end{equation*}
\end{solution}

\noindent \textbf{Recall} ${n \choose r}$ is the coefficient of $x^r$ in the
expansion of $(1+x)^n$,
\begin{equation*}
    (1+x)^n = \sum_{r = 0}^n {n \choose r} x^r
\end{equation*}
These are called binomial coefficients. 

\begin{theorem}
    The number of ways of partitioning $n$ distinct objects into $k$ distinct
groups containing $n_1, n_2, \dots, n_k$ object respectively where $n_1 +
\dots + n_k = n$ is,
\begin{equation*}
    {n \choose {n_1, n_2, \dots, n_k}} = 
    \frac{n!}{n_1! \cdot n_2! \cdot \cdots \cdot n_k!}
\end{equation*}
\end{theorem}
\begin{proof}
    The first group can be formed in $n \choose n_1$ ways. For each of these
$n \choose n_1$ possibilities, the second group can be formed in ${n - n_1}
\choose n_2$ many ways. Continuing like this, the total number of ways of
partitioning the $n$ objects into $k$ groups is, 
\begin{align*}
    &{n \choose n_1} \cdot {{n - n_1} \choose n_2} \cdot
    {{n - n_1 - n_2} \choose n_3} \cdot \dots \cdot 
    {{n - n_1 - n_2 - \dots - n_{k-1}} \choose n_k}                          \\
    =& \frac{n!}{n_1!(n-n_1)!} \cdot \frac{(n - n_1)!}{n_2! (n - n_1 - n_2)!}
       \cdot \frac{(n - n_1 - n_2)!}{n_3!(n - n_1 - n_2 - n_3)!} \cdot
       \dots \cdot \frac{(n - n_1 - \dots - n_{k-1})!}{n_k! (n - \sum_{i=1}^k
       n_i)!}                                                                \\
    =& \frac{n!}{n_1! \cdot n_2! \cdot \dots \cdot n_k!}
\end{align*}
\end{proof}

\begin{example}
    $4$ people get into an elevator, and each of them can get one of $10$
floors. Assuming each person gets off at a random floor, what is the
probability that no two get off at the same floor?
\end{example}

\begin{solution}
    Let $S$ be the sample space and $E$ denote the event that no two get off at
the same floor. The first person can get off at any of the $10$ floors. For
each of these $10$ possibilities, the second person can get off at any of the
$10$ floors and so on. Hence,
    \begin{equation*}
        \vert S \vert = 10 \times 10 \times 10 \times 10 = 10^4
    \end{equation*}
    Each of these outcomes is equally likely. And, 
    \begin{equation*}
        \vert E \vert = \perm[10]{4} = 10 \times 9 \times 8 \times 7
    \end{equation*}
    Hence required probability is, 
    \begin{equation*}
        P(E) = \frac{\vert E \vert}{\vert S \vert} 
             = \frac{10 \times 9 \times 8 \times 7}{10^4}
             = \frac{63}{125}
    \end{equation*}
\end{solution}

\textbf{Note} This is similar to the birthday problem (Example 2.7 of
textbook).

\begin{example}
    A committee of $5$ is to be selected from a group of $6$ men and $9$ women.
If the selection is made randomly, what is the probability that the committee
consists of $3$ men and $2$ women?
\end{example}
\begin{solution}
    Let $S$ be the sample space and $E$ denote the event that the committee
consists of $3$ men and $2$ women. Then $\vert S \vert = {15 \choose 5}$. Each
of these outcomes is equally likely. And, $\vert E \vert = {6 \choose 3} \cdot
{9 \choose 2}$. Hence, required possibility is, 
\begin{equation*}
    P(E) = \frac{\vert E \vert}{\vert S \vert}
         = \frac{{6 \choose 3} \cdot {9 \choose 2}}{{15 \choose 5}}
         = \frac{240}{1001}
\end{equation*}
\end{solution}

\begin{example}
    A $5$-card poker hand is said to be a full house if it consists of $3$ card
s of the same denomination and $2$ other cards of the same denomination (the
two denominations are necessarily different). If a $5$-card hand is dealt out
randomly, what is the probability that it is a full house?
\end{example}
\begin{solution}
    Obviously, $\vert S \vert = {52 \choose 5}$. Each of these outcomes is
equally likely. In order to get a full house, the two denominations can be
chosen in $13 \cdot 12$ ways, and then the suits can be chosen in ${4 \choose
3} \cdot {4 \choose 2}$ ways. Hence,
\begin{equation*}
    \vert E \vert = 13 \cdot 12 \cdot {4 \choose 3} \cdot {4 \choose 2}
\end{equation*}
And the required probability, 
\begin{equation*}
    P(E) = \frac{\vert E \vert}{\vert S \vert}
         = \frac{13 \cdot 12 \cdot {4 \choose 3} \cdot {4 \choose 2}}{{52
           \choose 5}}
         = \frac{6}{4165}
\end{equation*}
\end{solution}

\begin{example}
    \emph{(Exercise 2.64 of textbook)} A balanced die is tossed six times, and
the numbers on the uppermost face is recorded each time. What is the
probability that the numbers recorded are $1, 2, 3, 4, 5$ and $6$ in any order?
\end{example}
\begin{solution}
    The first toss can result in any of the $6$ faces. For each of them, the
second toss can result in any of the $6$ faces and so on. Hence, $\vert S
\vert = 6^6$, and each of the outcomes is equally likely. And, $\vert E \vert
= 6!$. Hence, the required probability is, 
\begin{equation*}
    P(E) = \frac{\vert E \vert}{\vert S \vert} 
         = \frac{6!}{6^6} 
         = \frac{5}{324}
\end{equation*}
\end{solution}

\begin{theorem}
    Suppose $k \geq 1$ and $r \geq 0$ are integers, then the number of
nonnegative integer solutions $(x_1, x_2, \dots, x_k)$ to
\begin{equation}
    \label{thm-counting-tech-1-1}
    x_1 + x_2 + \dots + x_k = r
\end{equation}
is ${{r + k - 1} \choose {k - 1}} = {{r + k - 1} \choose {n}}$.
\end{theorem}
\begin{proof}
    If we expand $(1 + x + x^2 + \dots)^k$ into a power series, then each term
will be a monomial of the form $x^{n_1} \cdot x^{n_2} \cdot \dots \cdot
x^{n_k} = x^{n_1 + \dots + n_k}$. Hence the numbers of non-negative integer
solutions to \ref{thm-counting-tech-1-1} above is same as the coefficient of
$x^r$ in the expansion of $(1 + x + x^2 + \dots)^k = (1 - x)^{-k}$, which
equals ${{r + k - 1} \choose {k - 1}}$ (recall the negative binomial series).
\end{proof}

\begin{example}
    If $P(A) = 0.8$, and $P(B) = 0.7$, then
    \begin{enumerate}[noitemsep,topsep=0pt]
        \item What is the largest possible value of $P(A \cap B)$?
        \item What is the smallest possible value of $P(A \cap B)$?
    \end{enumerate}
\end{example}
\begin{solution} \quad                                                       \\
    \begin{enumerate}[noitemsep,topsep=0pt]
        \item 
        $P(A \cap B) \leq P(B)$, (since $A \cap B \subseteq B$) and similarly, 
        $P(A \cap B) \leq P(A)$. Hence $P(A \cap B) \leq \min{\lbrace P(A),
        P(B) \rbrace} = 0.7$.
        \item
        \begin{align*}
                        & P(A \cup B) = P(A) + P(B) - P(A \cap B)            \\
            \Rightarrow & P(A) + P(B) - P(A \cap B) \leq 1                   \\
            \Rightarrow & P(A \cap B) \geq P(A) + P(B) - 1 = 0.5
        \end{align*}
    \end{enumerate}
\end{solution}

\begin{example}
    A deck of $52$ cards is shuffled, and the cards are turned up one at a
time. What is the probability that a card with face value $4$ appears before
any card with face value $5$?
\end{example}
\begin{solution}
    Let $C$ denote the set of cards. The sample space $S$ consists of all $52!$
arrangements of the cards. Further let $E_1$ be the event that a card with face
value $4$ appears before a card with face value $5$, and $E_2$ be the event
that a card with face value $5$ appears before a card with face value $4$.
Then, 
\begin{equation}
    \label{thm-counting-tech-2-1}
    E_1 \cup E_2 = S \text{ and } E_1 \cap E_2 = \emptyset
\end{equation}
Let $\phi : C \rightarrow C$ be the map that takes $4$ of any suit to $5$ of
that suit, and $5$ of any suit to $4$ of that suit, and leaves the reset of
cards fixed. Then $\phi$ is a bijection between $E_1$ and $E_2$. Hence $\vert
E_1 \vert = \vert E_2 \vert \Rightarrow P(E_1) = \frac{\vert E_1 \vert}{\vert
S \vert} = \frac{\vert E_2 \vert}{\vert S \vert} = P(E_2)$. Since,
\begin{align*}
    1 &= P(S)                                                                \\
      &= P(E_1 \cup E_2)                                                     \\
      &= P(E_1) + P(E_1)     &\text{(because of \ref{thm-counting-tech-2-1})}
\end{align*}
We have, $2 \cdot P(E_1) = 1 \Rightarrow P(E_1) = \frac{1}{2}$.
\end{solution}

\begin{example}
    Mary will go to a parity of either Tim, Jim, or Pim goes.
    \begin{align*}
        P(\text{Tim going}) &= 0.15                                          \\
        P(\text{Jim going}) &= 0.10                                          \\
        P(\text{Pim going}) &= 0.05
    \end{align*}
    Would you say Mary will go to the party with probability at least
    $\frac{1}{2}$?
\end{example}
\begin{solution}
    Let $T$ denote the event that Tim goes to the party, $J$ denote the event
that $Jim$ goes to the party, and $P$ denote the event that Pim goes to the
party. Then, 
\begin{align*}
    P(\text{Mary goes to the party}) &= P(T \cup J \cup P)                   \\
                                     &\leq P(T) + P(J) + P(P)                \\
                                     &= 0.30
\end{align*}
The answer is no.
\end{solution}

\begin{example}
    A football team consists of $20$ offensive and $20$ defensive players. THe
players are to be paired in groups of $2$ for the purpose of determining
roommates. If the pairing is done at random, what is the probability that there
are probability that there are no defensive-offensive roommate pairs?
\end{example}
\begin{solution}
    Let $S$ be the set of ways of pairing $40$ players into $20$ unordered
groups of size $2$ each, then 
\begin{equation*}
    \vert S \vert 
    = \frac{{40 \choose {2, 2, \dots, 2}}}{20!}
    = \frac{40!}{2^{20} \cdot 20!}
\end{equation*}
In order for no offensive-defensive pairing to happen, the defensive players
need to be paired among themselves, and the offensive players need to be paired
among themselves. This can be done in,
\begin{equation*}
    \frac{{20 \choose {2, 2, \dots, 2}}}{10!} \cdot \frac{{20 \choose {2, 2,
    \dots, 2}}}{10!}
    = \frac{(20!)^2}{(2^{10} \cdot 10!)^2}
    = \frac{(20!)^2}{2^{20} \cdot (10!)^2}
\end{equation*}
\end{solution}
\begin{solution}
    Rank the players $1$ through $40$. To pair up $40$ players into $20$
groups, the first players can be paired up in $39$ ways. Once this has
happened, the players with the smallest rank among the remaining $38$ players
can be paired up in $37$ ways. In the next step, the player with the smallest
rank among the remaining $36$ players, can be paired up in $35$ ways and so on.
Hence, 
\begin{equation*}
    \vert S \vert = 39 \cdot 37 \cdot 35 \cdot \dots \cdot 3 \cdot 1
\end{equation*}
    By the same argument, players can be paired up so that there are no
offensive-defensive pairs in $(19 \cdot 17 \cdot \dots \cdot 3 \cdot 1)^2$ many
ways. Thus, the required probability is, 
\begin{equation*}
    P(\text{no defensive-offensive pairs}) = 
    \frac{(19 \cdot 17 \cdot \dots \cdot 3 \cdot 1)^2}
         {39 \cdot 37 \cdot 35 \cdot \dots \cdot 3 \cdot 1}
\end{equation*}
\end{solution}

Now suppose the problem had been posed like this:
\begin{example}
    A football team consists of $20$ offensive and $20$ defensive players. A
travel agency has booked $20$ rooms in a hotel for the football team. The
players are to be paired in groups of $2$ and each pair will share one hotel
room. If the players are paired and each pair is assigned a hotel room all at
random, then what is the probability that there are no defensive-offensive
roommate pairs?
\end{example}
\noindent \textbf{Note}: for this problem, because the hotel room, a roommate
pair checks into, is taken into account, the sample space will consist of
elements each of which is an ordered collection of $20$ roommate pairs. So the
sample space would be different. However, the probability of the required event
is still the same. This is because of the following reason:

An ordered group of $20$ roommate pairs can be formed in two steps:
\begin{enumerate}[noitemsep,topsep=0pt]
    \item form an unordered group of $20$ roommate pairs
    \item order them in $20!$ possible ways. 
\end{enumerate}

Thus, when computing the required probability, the factor of $20!$ will cancel
out in the numerators and the denominators, leading to the same answer. But one
can also use a slightly different counting technique to get to the same answer.
\begin{solution}
    Let $S$ denotes the set of ways of pairing $40$ players into $20$ ordered
groups of size $2$ each. Then,
\begin{equation*}
    \vert S \vert = {{40} \choose {2, 2, \dots, 2}} = \frac{40!}{2^{20}}
\end{equation*}

    The offensive players can be paired among themselves into $10$ ordered
groups in $\frac{20!}{2^{10}}$ ways, and same for defensive players. Now we
would like to merge the ordered collection of $10$ offensive-offensive pairs
with the ordered collection of $10$ defensive-defensive pairs, to get one
ordered collection of $20$ roommate pairs. 

    Pictorially this is same as having $10$ numbered sticks and $10$ numbered
balls, and we want to insert the sticks in to $11$ possible places marked in
the picture. Keeping the relative position of the sticks unchanged (i.e., stick
$2$ will be inserted in a place to the right of stick $1$ etc.)
\begin{figure}[H]
    \centering
    \includesvg[./section2/figure/]{sec2-sub1-fig1}
\end{figure}
letting $x_k$ be the number of sticks inserted at mark $k$, we see that the
number of ways done is same as the number of nonnegative integer solutions to
$x_1 + x_2 + \dots + x_{11} = 10$, which equals ${{10 + 11 - 1} \choose {11 -
1}} = {20 \choose 10}$. Hence the required probability is,
\begin{equation*}
    P(\text{no defensive-offensive pairs}) 
    = \frac{\frac{20!}{2^{10}} \cdot
            \frac{20!}{2^{10}} \cdot
            {20 \choose 10}}
           {\frac{40!}{2^{20}}}
    = \frac{(20!)^3}{(10!)^2 \cdot 40!}
\end{equation*}

    Same as before.
\end{solution}