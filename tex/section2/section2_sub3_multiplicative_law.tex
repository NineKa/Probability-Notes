\subsection{Multiplicative Law of Probability}
If $P(B) > 0$, then $P(A \cap B) = P(B) \cdot P(A \vert B)$. Similarly, if
$P(A) > 0$, then $P(A \cap B) = P(A) \cdot P(B \vert A)$. Using this
repeatedly,
\begin{align*}
    P(A_1 \cap A_2 \cap \dots \cap A_k) 
    &= P(A_k \vert A_1 \cap A_2 \cap \dots \cap A_{k-1}) \cdot
       P(A_1 \cap \dots \cap A_{k-1})                                        \\
    &= P(A_k \vert A_1 \cap A_2 \cap \dots \cap A_{k-1}) \cdot
       P(A_{k-1} \vert A_1 \cap A_2 \cap \dots \cap A_{k-2}) \cdot
       P(A_1 \cap \dots \cap A_{k-2})                                        \\
    &\dots                                                                   \\
    &= P(A_k \vert A_1 \cap A_2 \cap \dots \cap A_{k-1}) \cdot
       P(A_{k-1} \vert A_1 \cap A_2 \cap \dots \cap A_{k-2}) \cdot
       \dots \cdot
       P(A_2 \vert A_1) \cdot
       P(A_1)
\end{align*}
provided with $P(A_1 \cap \dots \cap A_{k-1}) > 0$.

\begin{example}
    A student estimates his probability of receiving an A grade to be
$\frac{1}{2}$ in a French elective course and $\frac{2}{3}$ in a chemistry
course. He has to take exactly one of these courses. If he decides to make his 
decision on the flip of a fair coin, what is the probability that he takes
chemistry and gets an A?
\end{example}
\begin{solution}
    Let $C$ be the events that the student takes chemistry course, and $A$ be
the events that the student is getting an A. Then, 
\begin{equation*}
    P(A \cap C) = P(C) \cdot P(A \vert C)
                = \frac{1}{2} \cdot \frac{2}{3}
                = \frac{1}{3}
\end{equation*}
\end{solution}

\begin{example}
    If $P(B) > 0$, show that $P(A^\complement \vert B) = 1 - P(A \vert B)$.
\end{example}
\begin{solution}
    \begin{equation*}
        P(A^\complement \vert B) 
        = \frac{P(A^\complement \cap B)}{P(B)}                              
        = \frac{P(B) - P(A \cap B)}{P(B)}                                   
        = 1 - \frac{P(A \cap B)}{P(B)}                                      
        = 1 - P(A \vert B)
    \end{equation*}
\end{solution}


\begin{example}
    If $P(B) > 0$, show that $P(A_1 \cup A_2 \vert B) = P(A_1 \vert B) + P(A_2
\vert B) - P(A_1 \cap A_2 \vert B)$.
\end{example}
\begin{solution}
    \begin{align*}
           P(A_1 \cup A_2 \vert B)
        &= \frac{P((A_1 \cup A_2) \cap B)}{P(B)}                             \\
        &= \frac{P((A_1 \cap B) \cup (A_2 \cap B))}{P(B)}                    \\
        &= \frac{P(A_1 \cap B) + P(A_2 \cap B) - P(A_1 \cap A_2 \cap B)}
                {P(B)}                                                       \\
        &= P(A_1 \vert B) + P(A_2 \cap B) - P(A_1 \cap A_2 \vert B)
    \end{align*}
\end{solution}

\begin{example}
    Show that if $A_1 \subseteq A_2$, then $P(A_1 \vert B) \leq P(A_2 \vert
B)$.
\end{example}
\begin{solution}
    \begin{align*}
           P(A_2 \vert B)
        &= \frac{P(A_2 \cap B)}{P(B)}
         = \frac{P((A_2 \cap (A_1 \cup A_1^\complement)) \cap B)}{P(B)}      \\
        &= \frac{P(((A_2 \cap A_1) \cup (A_2 \cap A_1^\complement)) \cap
           B)}{P(B)}                                                         \\
        &= \frac{P((A_1 \cap B) \cup ((A_2 \cap A_1^\complement) \cap B))}
           {P(B)}                                                            \\
        &= \frac{P(A_1 \cap B) + P((A_2 \cap A_1^\complement) \cap B)}{P(B)} \\
        &= P(A_1 \vert B) + P(A_2 \cap A_1^\complement \vert B)              
    \end{align*}
    As $P(A_2 \cap A_1^\complement \vert B) \geq 0$, $P(A_1 \vert B) \leq P(A_2
\vert B)$.
\end{solution}

\begin{definition}
    If $B_1, B_2, \dots, B_k$ are subsets of a set $S$, such that,
    \begin{enumerate}[noitemsep, topsep=0em]
        \item $\bigcup_{i=1}^k B_i = S$ (exhaustive), and,
        \item $B_i \cap B_j = \emptyset$ whenever $i \neq j$ 
              (mutually disjoint)
    \end{enumerate}
    then, $\lbrace B_1, B_2, \dots, B_k \rbrace$ is said to be a partition of
$S$.

    \noindent
    If instead we had a countably infinite sequence $B_1, B_2, \dots$ of
subsets of $S$, such that,
    \begin{enumerate}[noitemsep, topsep=0em]
        \item $\bigcup_{i=1}^\infty B_i = S$, and, 
        \item $B_i \cap B_j = \emptyset \quad \forall i \neq j$
    \end{enumerate}
    then, $\lbrace B_1, B_2, \dots, \rbrace$ is a partition of $S$.
\end{definition}

\noindent \textbf{Note}: for any event $B$, $\lbrace B, B^\complement \rbrace$
is always a partition of $S$.