\section{Counting Techniques}
Suppose you have to complete two tasks. If Task 1 can be completed in $n_1$
ways, and for each of these $n_1$ possibilities, Task 2 can be completed in
$n_2$ ways, then the total number of ways the two tasks can be completed is
$n_1 \cdot n_2$.
\begin{example}
Toss a coin and roll a die simultaneously. The former can have two outcomes,
and for each of these, the latter can have six outcomes. Thus, the total number
of possible outcomes is $2 \times 6 = 12$. 
\end{example}

\paragraph{Recall} for any $n \in \mathbb{N}$, $n! = 1 \times 2 \times
\dots \times n$, and $0! = 1$ by definition. 

\begin{theorem}
    If we are given $n$ distinct objects and $m$ distinct boxes each of which
can hold one object and $n \leq m$, then the number of ways of filling up the
boxes is, 
\begin{equation*}
    \perm[n]{m} = \frac{n!}{(n - m)!}
\end{equation*}
\end{theorem}
\begin{proof}
    The first box can be filled up in $n$ ways. For each of these $n$ ways, the
second box can be filled up in $n - 1$ ways (by using any of the remaining $n
- 1$ objects). Thus the first two boxes can be filled up in $n \cdot (n - 1)$
ways. For each of these $n \cdot (n - 1)$ ways, the third box can be filled up
in $(n - 2)$ ways. Thus, the first $3$ boxes can be filled up in $n \cdot (n -
1) \cdot (n - 2)$ ways. Continuing like this, the $m$ boxes can be filled up
in, 
\begin{equation*}
    n \cdot (n-1) \cdot \dots \cdot (n - m + 1) = \frac{n!}{(n - m)!}
\end{equation*}
\end{proof}

\begin{example}
The papers from $3$ different exams need to be graded. $5$ graders are
available for this job. Each exam requires only one grader. Then the number of
ways of assigning graders to different exams is, 
\begin{equation*}
    \perm[5]{3} = \frac{5!}{(5 - 3)!} = 60
\end{equation*}
\end{example}

\begin{theorem}
    If $r \leq n$, the number of ways of selecting an unordered sample of $r$
objects from a total of $n$ distinct objects is,
\begin{equation*}
    \comb[n]{r} = \frac{n!}{(n - r)!r!}
\end{equation*}
This is also denoted by $n \choose r$.
\end{theorem}
\begin{proof}
    Suppose this can be done in $m$ ways. Now if we had $r$ distinct boxes,
then we could fill them up in two steps:
\begin{enumerate}[noitemsep,topsep=0pt]
\item 
    select an unordered sample of size $r$ from the $n$ distinct objects.
    This can be done in $m$ ways.
\item
    use these $r$ objects to fill up the $r$ boxes by putting one object in
    each box. This can be done in $\perm[r]{r}$ ways.
\end{enumerate}
Thus the total number of ways of filling up those $r$ boxes is $m \cdot
r!$. Hence,
\begin{equation*}
    m \cdot r! = \perm[n]{r} = \frac{n!}{(n - r)!}
    \Rightarrow
    m = \frac{n!}{(n - r)!r!}
\end{equation*}
\end{proof}

\begin{example}
    A committee of $3$ is to be formed from a group of $10$ people. How many
different committees are possible?
\end{example}
\begin{solution}
    Number of different committees is,
    \begin{equation*}
        {10 \choose 3} = \frac{10 \cdot 9 \cdot 8}{3 \cdot 2 \cdot 1} = 120
    \end{equation*}
\end{solution}

\paragraph{Recall} ${n \choose r}$ is the coefficient of $x^r$ in the expansion
of $(1+x)^n$,
\begin{equation*}
    (1+x)^n = \sum_{r = 0}^n {n \choose r} x^r
\end{equation*}
These are called binomial coefficients. 

\begin{theorem}
    The number of ways of partitioning $n$ distinct objects into $k$ distinct
groups containing $n_1, n_2, \dots, n_k$ object respectively where $n_1 +
\dots + n_k = n$ is,
\begin{equation*}
    {n \choose {n_1, n_2, \dots, n_k}} = 
    \frac{n!}{n_1! \cdot n_2! \cdot \cdots \cdot n_k!}
\end{equation*}
\end{theorem}
\begin{proof}
    The first group can be formed in $n \choose n_1$ ways. For each of these
$n \choose n_1$ possibilities, the second group can be formed in ${n - n_1}
\choose n_2$ many ways. Continuing like this, the total number of ways of
partitioning the $n$ objects into $k$ groups is, 
\begin{align*}
    &{n \choose n_1} \cdot {{n - n_1} \choose n_2} \cdot
    {{n - n_1 - n_2} \choose n_3} \cdot \dots \cdot 
    {{n - n_1 - n_2 - \dots - n_{k-1}} \choose n_k}                          \\
    =& \frac{n!}{n_1!(n-n_1)!} \cdot \frac{(n - n_1)!}{n_2! (n - n_1 - n_2)!}
       \cdot \frac{(n - n_1 - n_2)!}{n_3!(n - n_1 - n_2 - n_3)!} \cdot
       \dots \cdot \frac{(n - n_1 - \dots - n_{k-1})!}{n_k! (n - \sum_{i=1}^k
       n_i)!}                                                                \\
    =& \frac{n!}{n_1! \cdot n_2! \cdot \dots \cdot n_k!}
\end{align*}
\end{proof}

\begin{example}
    $4$ people get into an elevator, and each of them can get one of $10$
floors. Assuming each person gets off at a random floor, what is the
probability that no two get off at the same floor?
\end{example}

\begin{solution}
    Let $S$ be the sample space and $E$ denote the event that no two get off at
the same floor. The first person can get off at any of the $10$ floors. For
each of these $10$ possibilities, the second person can get off at any of the
$10$ floors and so on. Hence,
    \begin{equation*}
        \vert S \vert = 10 \times 10 \times 10 \times 10 = 10^4
    \end{equation*}
    Each of these outcomes is equally likely. And, 
    \begin{equation*}
        \vert E \vert = \perm[10]{4} = 10 \times 9 \times 8 \times 7
    \end{equation*}
    Hence required probability is, 
    \begin{equation*}
        P(E) = \frac{\vert E \vert}{\vert S \vert} 
             = \frac{10 \times 9 \times 8 \times 7}{10^4}
             = \frac{63}{125}
    \end{equation*}
\end{solution}

\textbf{Note} This is similar to the birthday problem (Example 2.7 of
textbook).

\begin{example}
    A committee of $5$ is to be selected from a group of $6$ men and $9$ women.
If the selection is made randomly, what is the probability that the committee
consists of $3$ men and $2$ women?
\end{example}
\begin{solution}
    Let $S$ be the sample space and $E$ denote the event that the committee
consists of $3$ men and $2$ women. Then $\vert S \vert = {15 \choose 5}$. Each
of these outcomes is equally likely. And, $\vert E \vert = {6 \choose 3} \cdot
{9 \choose 2}$. Hence, required possibility is, 
\begin{equation*}
    P(E) = \frac{\vert E \vert}{\vert S \vert}
         = \frac{{6 \choose 3} \cdot {9 \choose 2}}{{15 \choose 5}}
         = \frac{240}{1001}
\end{equation*}
\end{solution}

\begin{example}
    A $5$-card poker hand is said to be a full house if it consists of $3$ card
s of the same denomination and $2$ other cards of the same denomination (the
two denominations are necessarily different). If a $5$-card hand is dealt out
randomly, what is the probability that it is a full house?
\end{example}
\begin{solution}
    Obviously, $\vert S \vert = {52 \choose 5}$. Each of these outcomes is
equally likely. In order to get a full house, the two denominations can be
chosen in $13 \cdot 12$ ways, and then the suits can be chosen in ${4 \choose
3} \cdot {4 \choose 2}$ ways. Hence,
\begin{equation*}
    \vert E \vert = 13 \cdot 12 \cdot {4 \choose 3} \cdot {4 \choose 2}
\end{equation*}
And the required probability, 
\begin{equation*}
    P(E) = \frac{\vert E \vert}{\vert S \vert}
         = \frac{13 \cdot 12 \cdot {4 \choose 3} \cdot {4 \choose 2}}{{52
           \choose 5}}
         = \frac{6}{4165}
\end{equation*}
\end{solution}

\begin{example}
    \emph{(Exercise 2.64 of textbook)} A balanced die is tossed six times, and
the numbers on the uppermost face is recorded each time. What is the
probability that the numbers recorded are $1, 2, 3, 4, 5$ and $6$ in any order?
\end{example}
\begin{solution}
    The first toss can result in any of the $6$ faces. For each of them, the
second toss can result in any of the $6$ faces and so on. Hence, $\vert S
\vert = 6^6$, and each of the outcomes is equally likely. And, $\vert E \vert
= 6!$. Hence, the required probability is, 
\begin{equation*}
    P(E) = \frac{\vert E \vert}{\vert S \vert} 
         = \frac{6!}{6^6} 
         = \frac{5}{324}
\end{equation*}
\end{solution}