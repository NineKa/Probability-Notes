\section{Counting Techniques}
Suppose you have to complete two tasks. If Task 1 can be completed in $n_1$
ways, and for each of these $n_1$ possibilities, Task 2 can be completed in
$n_2$ ways, then the total number of ways the two tasks can be completed is
$n_1 \cdot n_2$.
\begin{example}
Toss a coin and roll a die simultaneously. The former can have two outcomes,
and for each of these, the latter can have six outcomes. Thus, the total number
of possible outcomes is $2 \times 6 = 12$. 

\paragraph{Recall} for any $n \in \mathbb{N}$, $n! = 1 \times 2 \times
\dots \times n$, and $0! = 1$ by definition. 

\begin{theorem}
    If we are given $n$ distinct objects and $m$ distinct boxes each of which
can hold one object and $n \leq m$, then the number of ways of filling up the
boxes is, 
\begin{equation*}
    \perm[n]{m} = \frac{n!}{(n - m)!}
\end{equation*}
\end{theorem}
\begin{proof}
    The first box can be filled up in $n$ ways. For each of these $n$ ways, the
second box can be filled up in $n - 1$ ways (by using any of the remaining $n
- 1$ objects). Thus the first two boxes can be filled up in $n \cdot (n - 1)$
ways. For each of these $n \cdot (n - 1)$ ways, the third box can be filled up
in $(n - 2)$ ways. Thus, the first $3$ boxes can be filled up in $n \cdot (n -
1) \cdot (n - 2)$ ways. Continuing like this, the $m$ boxes can be filled up
in, 
\begin{equation*}
    n \cdot (n-1) \cdot \dots \cdot (n - m + 1) = \frac{n!}{(n - m)!}
\end{equation*}
\end{proof}
\end{example}
